\documentclass[fleqn]{article}

\usepackage[utf8]{inputenc}
\usepackage[bulgarian]{babel}
\usepackage{amsmath}
\usepackage{amssymb}
\usepackage{booktabs}

\newtheorem{theorem}{Tеорема}[subsection]
\newtheorem{example}{Пример}[subsection]
\newtheorem{definition}{Дефиниция}[subsection]
\newtheorem{law}{Закон}[subsection]

\title{Физика}
\author{Exonaut}


\begin{document}
\maketitle
\pagenumbering{gobble}
\newpage
\pagenumbering{arabic}

\tableofcontents
\newpage

\section{Лекция 1: Кинематика}
Механиката се дели на: 
\begin{itemize}
	\item Кинематика: описва движението, без да се интересува от причините, които го пораждат.
	\item Динамика: изучава законите за движение и причините, които го предизвикват.
	\item Статика: изучава условията за равновесие на телата.
\end{itemize}

\subsection{Основни понятия}

\begin{itemize}
	\item Материална точка: тяло, чиито форма и размери могат да се пренебрегнат при изучаване на движението му.
	\item Отправно тяло: тяло, спрямо което отчитаме движението.
	\item Отправна система: състои се от отправно тяло, координатна система и часовник.
	\item Радиус вектор: вектор от началото на отправната система до материалната точка. Означава се с $\vec{r}(t)$
	\item Траектория: линията, описвана от материалната точка при движението й.
	\item Път: дължината на траекторията от началното до крайното положение.
	\item Преместване: вектор от началното до крайното положение.
\end{itemize}

\subsection{Праволинейно движение}
Като начало ще разгледаме движението само по едно направление, например по оста x. Такова движение се нарича праволинейно.

\subsubsection{Средна скорост}
Средна скорост: преместването по $\Delta x$ разделена на интервала време $\Delta t$, или $V(t) = \dfrac{\Delta x}{\Delta t}$.

\subsubsection{Моментна скорост}
Ако интеравала е много малък $(\Delta t \rightarrow 0)$ скоростта се нарича моментна : $V(t) = \lim\limits_{\Delta t \rightarrow 0}\dfrac{\Delta x}{\Delta t} = \dfrac{dx}{dt}$. $dx$ е много малко преместване извършено в за много малък интервал от време $dt$.\\
\\
\textbf{Моментната скорост е първа производна на радиус-вектора по времето.} или 
$$V(t) = \dfrac{d \vec{r}}{dt}$$
$$V = \left[ \dfrac{m}{s}\right] = \left[ \dfrac{km}{h}\right], \quad 1\dfrac{m}{s} = 3,6 \dfrac{km}{h} $$

\subsubsection{Средно ускорение}
Средно ускорение наричаме изменението на скоростта $\Delta V$ , разделено на интервала време, за който е извършено това изменение: $a(t) =\dfrac{\Delta V}{\Delta t}$.

\subsubsection{Моментно ускорение}
Ако интеравала е много малък $(\Delta t \rightarrow 0)$ ускорението се нарича моментно : $a(t) = \lim\limits_{\Delta t \rightarrow 0}\dfrac{\Delta V}{\Delta t} = \dfrac{dV}{dt}$.\\
\\
\textbf{Моментното ускорение е първа производна на скоростта по времето и втора производна на радиус-вектора по времето: } или 
$$a(t) =  \dfrac{dV}{dt} = \dfrac{d^2 \vec{r}}{dt^2}$$
$$a = \left[ \dfrac{m}{s^2} \right]$$

\begin{example}
Тяло се движи по закон $x = 5t^3 + 2t^2 + 1$. Да се намери скоростта и ускорението в момента $t = 1s$. \\
Решение: \\
$
V(t) = \dfrac{dx}{dt}, \quad a(t) = \dfrac{dV}{dt} \\
V(t) = 5 \cdot 3 \cdot t^{3-1} + 2 \cdot 2 \cdot t^{2-1} + 0 = 15 \cdot t^2 + 4\cdot t \\
a(t) = 15\cdot 2 \cdot t^{2-1} + 4 = 30 \cdot t + 4 \\
V(1) = 15 \cdot 1^2 + 4 \cdot 1 = 15 + 4 = 19 \dfrac{m}{s} \\
a(1) = 30 \cdot 1 + 4 = 30 + 4 = 34 \dfrac{m}{s^2}
$
\end{example}

\subsection{Движение с постоянна скорост}
Нека материална точка се движи с начална скорост $V_0$ . В момента $t_0 = 0$ тя започва да се движи с постоянно ускорение $a = const$. В някакъв по-късен момент $t$ материалната точка се движи със скорост $V$. От дефиницията за ускорение  $ a =\dfrac{\Delta V}{\Delta t}$ можем да запишем 
$$a = \dfrac{V - V_0}{t - t_0} = \dfrac{V - V_0}{t}$$
$$a = \dfrac{V - V_0}{t} \implies V - V_0 = at \implies V = V_0 + at$$
Изразът за зависимостта на скоростта от времето($V = V_0 + at$) се нарича закон за скоростта. \\
\\
Нека материална точка започва да се движи в момента $t_0 - 0$ от положение с координата $x_0$ с постоянна скорост  $V_0 = const$. В някакъв по-късен момент $t$ материалната точка има координата $x$. От дефиницията за скорост  $V_0 = \dfrac{\Delta x}{\Delta t}$ можем да запишем $V_0 = \dfrac{x-x_0}{t - t_0}$ или $x = x_0 + V_0(t-t_0)$, но $t_0 = 0$ от където следва 
$$x = x_0 + V_0t$$
При движение с постоянно ускорение $a$ към горния израз се добавя още един член, отчитащ промяната в скоростта: 
$$x = x_0 + V_0t + \dfrac{at^2}{2}$$
Изразът, даващ зависимостта на радиус-вектора от времето се нарича закон за движение. \\
Знаците пред скоростта и ускорението в горните изрази могат да бъдат като положителни, така и отрицателни. Знакът е положителен, ако посоката на $V$ или а съвпада с посоката на оста х и отрицателен, ако посоката е противоположна на оста х.\\
Ако ускорението е константа и скоростта на тялото нараства с времето, движението се нарича \textit{равноускорително}. Ако скоростта на тялото намалява – \textit{равнозакъснително}, а ако ускорението е нула и скоростта на тялото не се променя, говорим за \textit{равномерно} движение.

\begin{example}
Кола се движи със скорост $V_0$ . След задействане на спирачката, колата започва да се движи равнозакъснително с ускорение $a$ и скоростта на колата намалява до $V$. Намерете спирачния път.\\
Решение:
$$V = V_0 - at$$
$$x = V_0t - \dfrac{at^2}{2}$$
Oт първото равенство имаме $t = \dfrac{V_0 - V}{a}$ и заместваме във второто равенство
$$x = V_0 \dfrac{V_0 - V}{a} - \dfrac{1}{2}a \left( \dfrac{V_0 - V}{a}\right)^2 $$
$$x = \dfrac{V_0 ^2 - VV_0}{a} -  \dfrac{a}{2}  \left( \dfrac{V_0 ^2 - 2VV_0 + V^2}{a^2}\right)$$ 
$$x =\dfrac{V_0 ^2 - VV_0}{a} - \dfrac{a(V_0 ^2 - 2VV_0 + V^2)}{2a^2}$$
$$x = \dfrac{2(V_0 ^2 - VV_0)}{2a} - \dfrac{V_0 ^2 - 2VV_0 + V^2}{2a}$$
$$x = \dfrac{2V_0 ^2 - 2VV_0 - V_0 ^2 + 2VV_0 - V^2}{2a}$$
$$x = \dfrac{V_0 ^2 - V^2}{2a}$$
\end{example}

\begin{example}
Тяло е хвърлено вертикално нагоре от височина $h_0 = 1m$ с начална скорост $V_0 = 10 \dfrac{m}{s}$. След колко време тялото ще достигне максимална височина? До каква максимална височина ще се издигне тялото? След колко време и с каква скорост тялото ще падне до $h = 0$. \\
Решение: \\
Всички тела в близост до земята се движат с ускорение $g = 9,8 \dfrac{m}{s^2} \approx 10 \dfrac{m}{s^2}$. \\
Записваме закона за скоростта и закона за движение по оста y:
$$V = V_0 - gt$$
$$y = h_0 + V_0t - \dfrac{gt^2}{2}$$
Знакът пред $V_0$ е положителен, защото посоката й съвпада с посоката на оста $y$, а знакът пред $g$ е отрицателен, защото посоката му е противоположна на оста $y$.\\
Когато тялото се издига, скоростта му намалява, в най-високата точка става нула, след което тялото започва да пада, скоростта му става отрицателна, понеже е насочена срещу оста $y$. В най-високата точка $V = 0$  или $0 = V_0 - gt$. От тук намираме времето, за което тялото ще достигне най-високата точка : $t = \dfrac{V_0}{g} = \dfrac{10}{10} = 1s$. Заместваме това време в израза за y за да получим максималната височина:
$$h_max = h_0 + V_0t -  \dfrac{gt^2}{2} = 1+ 10 \cdot 1 -  \dfrac{10 \cdot 1^2}{2} = 11 - 5 = 6m$$
Времето за падане до $h = 0$  намираме от условието $y = 0$.
$$0 = h_0 + V_0t - \dfrac{gt^2}{2} $$
$$0 = 1 + 10t -  \dfrac{10t^2}{2}$$
$$-5t^2 + 10t  + 1 = 0 $$
$$D = 10^2 - 4 \cdot (-5) \cdot 1 = 100 + 20 = 120 $$
$$t_{1,2} = \dfrac{-10 \pm \sqrt{120}}{2 \cdot (-5)} = \dfrac{-10 \pm 2\sqrt{30}}{-10} = \dfrac{-5 \pm \sqrt{30}}{-5}$$
$$t_1 = \dfrac{-5 -\sqrt{30}}{-5} \approx -0.1 s$$
$$t_2 = \dfrac{-5 +\sqrt{30}}{-5} \approx 2.1 s$$
Физичен смисъл има само положителното време. Заместваме го в израза за скоростта $V = V_0 - gt = 10 - 10 \cdot 2.1 = -11 m/s$ Това е скоростта, с която тялото пада на земята. Тя е отрицателна, защото е насочена срещу оста $y$.
\end{example}

\subsection{Движение при произволна форма на траекторията}
Когато движението не е праволинейно, скоростта и ускорението се записват за всяка от компонентите на радиус-вектора: 
$$V_x = \dfrac{dx}{dt}, \quad V_y = \dfrac{dy}{dt}, \quad V_z = \dfrac{dz}{dt} \qquad \vec{V} = \dfrac{d\vec{r}}{dt}$$
Скоростта $\vec{V}$ е векторна величина - тя се характеризира с големина и посока. Големината на скоростта се определя от координатите на скоростта (по Питагоровата теорема) $V = \sqrt{V_x^2 + V_y^2 + V_z^2}$ \\
Аналогично: 
$$\vec{a} = \dfrac{d\vec{V}}{dt} = \dfrac{d^2 \vec{r}}{dt^2}$$ 
Тъй като скоростта е вектор, тя може да се изменя поради промяна на големината и поради промяна на посоката си.\\
Ускорението, дължащо се на изменение на скоростта по големина се нарича тангенциално ускорение $\vec{a_\tau}$. То има посока, съвпадаща с направлението на скоростта.\\
Ускорението, дължащо се на изменение на скоростта по посока се нарича нормално ускорение $\vec{a_n}$. То има посока, перпендикулярна на направлението на скоростта. Може да се покаже, че $\vec{a_n} = \dfrac{V^2}{R}$, като R е радиусът на кривината на траекторията в разглежданата точка. \\
От казаното по-горе е ясно, че при праволинейно движение нормалното ускорение е винаги нула. При движение по крива, дори и с постоянна скорост, нормалното ускорение е различно от нула.
Пълното ускорение се получава като векторна сума от тангенциалното и нормалното ускорение:
$$\vec{a} =\vec{a_\tau} + \vec{a_n}$$
При постоянно ускорение законът за скоростта и за движение се записват във векторен вид: 
$$\vec{V} = \vec{V_0} + \vec{a}t$$
$$\vec{r} = \vec{r_0} + \vec{V_0}t + \dfrac{\vec{a}t^2}{2}$$
След това се записват уравненията за всяка от компонентите на векторите.

\begin{example}
Тяло е хвърлено под ъгъл $\alpha = 30^\circ $ спрямо хоризонта с начална скорост 
$V_0 = 10 \dfrac{m}{s}$. Намерете максималната височина, до която се издига тялото и разстоянието, което то прелита. \\
Решение: \\
$$\vec{V} = \vec{V_0} + \vec{g}t$$
$$\vec{r} = \vec{V_0}t + \dfrac{\vec{g}t^2}{2}$$
$\vec{r_0} = 0$. \\
Всеки от векторите може да бъде разложен на две компоненти $x$и $y$. \\
По оста $x$: $V_x = V_{0x}, \quad x = V_{0x}t$ \\
По оста $y$: $V_y = V_{0y} - gt, \quad y = V_{0y}t - \dfrac{\vec{g}t^2}{2}$ \\
Тук сме взели предвид, че по оста $x$ няма ускорение, а по оста $y$ ускорението е $g$, насочено надолу, в посока обратна на оста $y$(и затова с отрицателен знак). \\
$$V_{0x} = V_0 \cos{30^\circ} = \dfrac{\sqrt{3}}{2}V_0 \quad V_{0y} = V_0 \sin{30^\circ} = \dfrac{V_0}{2}$$
В най високата точка $V_y$ е равна на 0: $0 = V_{0y} - gt$ и времето за което тялото достига максимална височина е $t = \dfrac{V_{0y}}{g}$
Заместваме това време в израза за y и получаваме 
$$y_{max} = V_{0y} \dfrac{V_{0y}}{g} - \dfrac{g}{2} \cdot \left(  \dfrac{V_{0y}}{g} \right)^2 $$
$$y_{max} = \dfrac{V_{0y}^2}{g} - \dfrac{1}{2} \cdot \dfrac{V_{0y}^2}{g}$$ 
$$y_{max} = \dfrac{1}{2} \cdot \dfrac{V_{0y}^2}{g} =  \dfrac{1}{2} \cdot \dfrac{\left( \dfrac{10}{2} \right)^2}{10} = \dfrac{5^2}{20} = \dfrac{25}{20} = 1.25m$$
\end{example}

В общия случай, когато ускорението не е постоянно, законът за скоростта се получава с интегриране. 
$$\vec{a} =  \dfrac{d\vec{V}}{dt} \Leftrightarrow d\vec{V} = \vec{a}dt$$
Интегрираме и получаваме закона за скоростта в общия случай: 
$$\vec{V} = \int _0 ^ t  \vec{a}(t)dt + \vec{V_0}$$
Аналогично закона за движение : 
$$\vec{V} =  \dfrac{d\vec{r}}{dt} \Leftrightarrow d\vec{r} = \vec{V}dt$$
Интегрираме и получаваме закона за пътя в общия случай:
$$\vec{r} = \int _0 ^ t  \vec{V}(t)dt + \vec{r_0}$$
Ще използваме този резултат за да получим закона за движение при постоянно ускорение. При движение с постоянно ускорение  
$$\vec{V} = \int _0 ^ t  \vec{a}dt + \vec{V_0} = \vec{V_0} + \vec{a} \int _0 ^ t dt = \vec{V_0} + \vec{a}t$$
Заместваме този резултат в $\vec{r} = \int _0 ^ t  \vec{V}(t)dt + \vec{r_0}$ и получаваме 
$$\vec{r} = \int _0 ^ t (\vec{V_0} + \vec{a}t) dt + \vec{r_0} = \vec{r_0} + \int _0 ^ t \vec{V_0} dt + \int _0 ^ t \vec{a}t dt$$
$$ \vec{r} = \vec{r_0} + \vec{V_0}t +\dfrac{\vec{a}t^2}{2} $$

\newpage

\section{Лекция 2: Динамика на материална точка}

\subsection{Принципи на механиката (Принципи на Нютон)}
 
\subsubsection{Първи принцип}

\textbf{Всяко тяло запазва състоянието си на покой или на праволинейно равномерно движение, докато външно въздействие не го изведе от това състояние.}\\
Този принцип не е в сила за всички отправни системи. Отправните системи, за които той е в сила, се наричат \textit{инерциални отправни системи}. \\
Величината, която количествено характеризира взаимодействието между телата, се нарича сила. Силата е векторна величина – има големина, посока и приложна точка. Означава се с буквата F. Мерната единица за сила е нютон N.\\
Телата притежаватсвойството инертност – съпротивляват се на въздействие, което се стреми да ги извади от състоянието им на покой или на праволинейно равномерно движение. Това свойство се характеризира с величината маса m. Измерва се в килограми kg. Колкото по-голяма масата на едно тяло, толкова по-трудно е да изменим неговата скорост. \\
Произведението на масата и скоростта на едно тяло се нарича импулс $$\vec{p} = m \vec{V}$$
Mерната единица за импулс е  $ \dfrac{kg \cdot m}{s} $

\subsubsection{Втори принцип}
\textbf{Първата производна на импулса по времето е равна на силата, действаща на тялото.}
$$\vec{F} = \dfrac{d \vec{p}}{dt}$$
Ако масата не се променя можем да запишем
$$\vec{F} = \dfrac{d \vec{p}}{dt} = \dfrac{d m\vec{V}}{dt} = m \dfrac{d\vec{V}}{dt} = m \vec{a}$$
Mерната единица за сила $N = \dfrac{kg \cdot m}{s^2}$, Когато на тялото
действат няколко сили, $\vec{F}$ е векторната сума на тези сили.

\subsubsection{Трети принцип}
\textbf{Силите на взаимодействие между две тела са равни по големина и противоположни по посока.}

\subsection{Някои видове сили}

\subsubsection{Гравитационна сила}
Законът на Нютон за гравитацията гласи: \textit{Между всеки две материални точки действа сила на привличане, която е правопропорционална на произведението на масите им и обратно пропорционална на квадрата на разстоянието между тях.} \\
$$F = \gamma \dfrac{m_1 m_2}{r^2}$$
$\gamma = 6.67 \cdot 10^{-11} \dfrac{N \cdot m^2}{kg^2}$, $m_1, m_2$ - масите на двете материални точки, а r - разстоянието между тях. 

\begin{example}
Две тела с маси $m_1 =m_2 = 100 kg$ са разположени на разстояние r = 1 m.Намерете силата на привличане. \\
Решение: \\
$$F = \gamma \dfrac{m_1 m_2}{r^2} =  6.67 \cdot 10^{-11}  \dfrac{100 \cdot 100}{1^2} = 6.67 \cdot 10^{-7} N$$
\end{example}


\subsubsection{Сила на тежестта}
Разглеждаме тяло в близост до земната повърхност. Гравитационната сила, действаща на тялото в този случай ще означим с G, масата Земята с M, а разстоянието до центъра на Земята с R. Записваме закона за гравитацията:
$$G = \gamma \dfrac{mM}{R^2} = m \left( \gamma \dfrac{M}{R^2} \right)$$
$\gamma \dfrac{M}{R^2} $ е еднаква за всички тела величина, която се означава с g и се нарича земно ускорение. Стойността на $g \approx 9.8 \dfrac{m}{s^2}$ е измерена експериментално. Оттук може да запишем за силата на тежестта:
$$G = mg$$
При принципите на Нютон въведохме масата като мярка за инерчните свойства на телата. Тук даваме още едно определение за масата – тя характеризира гравитационните свойства на телата. Масата в  $\vec{F} = m \vec{a}$ се нарича инертна маса, а масата в $G = mg$ - тежка маса. Съгласно съвременната физика тежката и инертната маса са еквивалентни.

\subsubsection{Реакция на опората}
Разглеждаме книга поставена на един чин. Книгата действа на чина със силата на тежестта $G = mg$ , насочена надолу. Съгласно третия принцип на Нютон и чинът действа на книгата със същата по големина сила, но насочена нагоре. Тази сила се нарича реакция на опората и ще я означаваме с N. В крайна сметка на книгата действат две равни по големина сили G и N , насочени в противоположни посоки. Тяхната векторна сума е нула и затова книгата остава в покой. \\
Реакцията на опората винаги е перпендикулярна на повърхността, на която е поставено тялото.
 
\begin{example}
Тяло се спуска без триене по равнина, наклонена под ъгъл $\theta$. Определете ускорението на тялото и реакцията на опората. \\
Решение: \\
Избираме отправна система, при която оста х е успоредна на равнината, а оста у е перпендикулярна на равнината. Записваме втория принцип на Нютон $\vec{F} = m \vec{a}$. На тялото действат две сили: сила на тежестта и реакция на опората и следователно силата $\vec{F}$ е сума от тези две сили.
$$\vec{F} = \vec{G} + \vec{N} = m \vec{a} $$
Записваме това уравнение за всяка от осите
$$F_x = G_x = ma$$
$$F_y = N - G_y = 0$$
Тук сме отчели, че по оста у няма движение и ускорението е нула. В горните изрази: $G_x  =G \sin{\theta} = mg \sin{\theta}, G_y  =G \cos{\theta} = mg \cos{\theta}$\\
$$G_x = mg \sin{\theta} = ma$$
$$a =g\sin{\theta}$$
$$N = G_y = mg \cos{\theta}$$
\end{example}

\subsubsection{Сила на триене}
Разглеждаме тяло, поставено върху хоризонтална поставка. Между молекулите на тялото и на поставката възникват електромагнитни сили, които се противопоставят на движението на тялото спрямо поставката. Тези сили се наричат сили на триене. Силата на триене винаги е насочена срещу посоката на движение (на фигурата външната сила F движи тялото надясно, а силата на триене f е насочена наляво. Големината на силата на триене е пропорционална на реакцията на опората N.
$$f = kN$$
Коефициентът на пропорционалност k се нарича коефициент на триене и зависи от материала, от който са изработени триещите се повърхности, грапавините и други.

\begin{example}
Автомобил с маса $m = 1000 kg$ се движи по хоризонтален път със скорост $V_0 = 54 km/h$. След задействане на спирачкитеавтомобилът спира за време 5 s. Определете силата на триене и коефициента на триене.\\
Решение: \\
Като имаме предвид, че 1 m/s = 3,6 km/h, намираме, че v0 = 54 km/h = 15 m/s. \\
$$V = V_0 - at$$
В момента на спиране $V = 0$ и $0 = V_0 - at$ или $a = \dfrac{V_0}{t} = \dfrac{15}{5} = 3 \dfrac{m}{s^2}. $\\
От $f = ma$ получаваме $f =1000 \cdot 3 = 3000 N$ \\
f = kN Тук, понеже сме на хоризонтален път, N = mg и f = kmg. Последно:
$$k = \dfrac{f}{mg} = \dfrac{3000}{1000 \cdot 10} = 0.3$$
\end{example}

\begin{example}
Шейна се движи по хоризонтална повърхност, като коефициентът на триене между шейната и снега е k. Теглим шейната със сила Т насочена под ъгъл $\theta$. Напишете уравненията за силите, действащи на шейната \\
Решение: 
\end{example}

\newpage
\section{Лекция 3: Механика на идеално твърдо тяло}

\newpage
\section{Лекция 4: }

\newpage
\section{Формули}

\subsection{Лекция 1:}

\subsection{Лекция 2:}

\subsection{Лекция 3: }

\subsection{Лекция 4: }











\end{document}