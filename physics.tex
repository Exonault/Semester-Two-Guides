\documentclass[fleqn]{article}

\usepackage[utf8]{inputenc}
\usepackage[bulgarian]{babel}
\usepackage{amsmath}
\usepackage{amssymb}
\usepackage{booktabs}

\newtheorem{theorem}{Tеорема}[subsection]
\newtheorem{example}{Пример}[subsection]
\newtheorem{definition}{Дефиниция}[subsection]

\title{Физика}
\author{Exonaut}


\begin{document}
\maketitle
\pagenumbering{gobble}
\newpage
\pagenumbering{arabic}

\tableofcontents
\newpage

\section{Лекция 1: Кинематика}
Механиката се дели на: 
\begin{itemize}
	\item Кинематика: описва движението, без да се интересува от причините, които го пораждат.
	\item Динамика: изучава законите за движение и причините, които го предизвикват.
	\item Статика: изучава условията за равновесие на телата.
\end{itemize}

\subsection{Основни понятия}

\begin{itemize}
	\item Материална точка: тяло, чиито форма и размери могат да се пренебрегнат при изучаване на движението му.
	\item Отправно тяло: тяло, спрямо което отчитаме движението.
	\item Отправна система: състои се от отправно тяло, координатна система и часовник.
	\item Радиус вектор: вектор от началото на отправната система до материалната точка. Означава се с $\vec{r}(t)$
	\item Траектория: линията, описвана от материалната точка при движението й.
	\item Път: дължината на траекторията от началното до крайното положение.
	\item Преместване: вектор от началното до крайното положение.
\end{itemize}

\subsection{Праволинейно движение}
Като начало ще разгледаме движението само по едно направление, например по оста x. Такова движение се нарича праволинейно.

\subsubsection{Средна скорост}
Средна скорост: преместването по $\Delta x$ разделена на интервала време $\Delta t$, или $V(t) = \dfrac{\Delta x}{\Delta t}$.

\subsubsection{Моментна скорост}
Ако интеравала е много малък $(\Delta t \rightarrow 0)$ скоростта се нарича моментна : $V(t) = \lim\limits_{\Delta t \rightarrow 0}\dfrac{\Delta x}{\Delta t} = \dfrac{dx}{dt}$. $dx$ е много малко преместване извършено в за много малък интервал от време $dt$.\\
\\
\textbf{Моментната скорост е първа производна на радиус-вектора по времето.} или 
$$V(t) = \dfrac{d \vec{r}}{dt}$$
$$V = \left[ \dfrac{m}{s}\right] = \left[ \dfrac{km}{h}\right], \quad 1\dfrac{m}{s} = 3,6 \dfrac{km}{h} $$

\subsubsection{Средно ускорение}
Средно ускорение наричаме изменението на скоростта $\Delta V$ , разделено на интервала време, за който е извършено това изменение: $a(t) =\dfrac{\Delta V}{\Delta t}$.

\subsubsection{Моментно ускорение}
Ако интеравала е много малък $(\Delta t \rightarrow 0)$ ускорението се нарича моментно : $a(t) = \lim\limits_{\Delta t \rightarrow 0}\dfrac{\Delta V}{\Delta t} = \dfrac{dV}{dt}$.\\
\\
\textbf{Моментното ускорение е първа производна на скоростта по времето и втора производна на радиус-вектора по времето: } или 
$$a(t) =  \dfrac{dV}{dt} = \dfrac{d^2 \vec{r}}{dt^2}$$
$$a = \left[ \dfrac{m}{s^2} \right]$$

\begin{example}
Тяло се движи по закон $x = 5t^3 + 2t^2 + 1$. Да се намери скоростта и ускорението в момента $t = 1s$. \\
Решение: \\
$
V(t) = \dfrac{dx}{dt}, \quad a(t) = \dfrac{dV}{dt} \\
V(t) = 5 \cdot 3 \cdot t^{3-1} + 2 \cdot 2 \cdot t^{2-1} + 0 = 15 \cdot t^2 + 4\cdot t \\
a(t) = 15\cdot 2 \cdot t^{2-1} + 4 = 30 \cdot t + 4 \\
V(1) = 15 \cdot 1^2 + 4 \cdot 1 = 15 + 4 = 19 \dfrac{m}{s} \\
a(1) = 30 \cdot 1 + 4 = 30 + 4 = 34 \dfrac{m}{s^2}
$
\end{example}

\subsection{Движение с постоянна скорост}
Нека материална точка се движи с начална скорост $V_0$ . В момента $t_0 = 0$ тя започва да се движи с постоянно ускорение $a = const$. В някакъв по-късен момент $t$ материалната точка се движи със скорост $V$. От дефиницията за ускорение  $ a =\dfrac{\Delta V}{\Delta t}$ можем да запишем 
$$a = \dfrac{V - V_0}{t - t_0} = \dfrac{V - V_0}{t}$$
$$a = \dfrac{V - V_0}{t} \implies V - V_0 = at \implies V = V_0 + at$$
Изразът за зависимостта на скоростта от времето($V = V_0 + at$) се нарича закон за скоростта. \\
\\
Нека материална точка започва да се движи в момента $t_0 - 0$ от положение с координата $x_0$ с постоянна скорост  $V_0 = const$. В някакъв по-късен момент $t$ материалната точка има координата $x$. От дефиницията за скорост  $V_0 = \dfrac{\Delta x}{\Delta t}$ можем да запишем $V_0 = \dfrac{x-x_0}{t - t_0}$ или $x = x_0 + V_0(t-t_0)$, но $t_0 = 0$ от където следва 
$$x = x_0 + V_0t$$
При движение с постоянно ускорение $a$ към горния израз се добавя още един член, отчитащ промяната в скоростта: 
$$x = x_0 + V_0t + \dfrac{at^2}{2}$$
Изразът, даващ зависимостта на радиус-вектора от времето се нарича закон за движение. \\
Знаците пред скоростта и ускорението в горните изрази могат да бъдат като положителни, така и отрицателни. Знакът е положителен, ако посоката на $V$ или а съвпада с посоката на оста х и отрицателен, ако посоката е противоположна на оста х.\\
Ако ускорението е константа и скоростта на тялото нараства с времето, движението се нарича \textit{равноускорително}. Ако скоростта на тялото намалява – \textit{равнозакъснително}, а ако ускорението е нула и скоростта на тялото не се променя, говорим за \textit{равномерно} движение.

\begin{example}
Кола се движи със скорост $V_0$ . След задействане на спирачката, колата започва да се движи равнозакъснително с ускорение $a$ и скоростта на колата намалява до $V$. Намерете спирачния път.\\
Решение:
$$V = V_0 - at$$
$$x = V_0t - \dfrac{at^2}{2}$$
Oт първото равенство имаме $t = \dfrac{V_0 - V}{a}$ и заместваме във второто равенство
$$x = V_0 \dfrac{V_0 - V}{a} - \dfrac{1}{2}a \left( \dfrac{V_0 - V}{a}\right)^2 $$
$$x = \dfrac{V_0 ^2 - VV_0}{a} -  \dfrac{a}{2}  \left( \dfrac{V_0 ^2 - 2VV_0 + V^2}{a^2}\right)$$ 
$$x =\dfrac{V_0 ^2 - VV_0}{a} - \dfrac{a(V_0 ^2 - 2VV_0 + V^2)}{2a^2}$$
$$x = \dfrac{2(V_0 ^2 - VV_0)}{2a} - \dfrac{V_0 ^2 - 2VV_0 + V^2}{2a}$$
$$x = \dfrac{2V_0 ^2 - 2VV_0 - V_0 ^2 + 2VV_0 - V^2}{2a}$$
$$x = \dfrac{V_0 ^2 - V^2}{2a}$$
\end{example}

\begin{example}
Тяло е хвърлено вертикално нагоре от височина $h_0 = 1m$ с начална скорост $V_0 = 10 \dfrac{m}{s}$. След колко време тялото ще достигне максимална височина? До каква максимална височина ще се издигне тялото? След колко време и с каква скорост тялото ще падне до $h = 0$. \\
Решение: \\
Всички тела в близост до земята се движат с ускорение $g = 9,8 \dfrac{m}{s^2} \approx 10 \dfrac{m}{s^2}$. \\
Записваме закона за скоростта и закона за движение по оста y:
$$V = V_0 - gt$$
$$y = h_0 + V_0t - \dfrac{gt^2}{2}$$
Знакът пред $V_0$ е положителен, защото посоката й съвпада с посоката на оста $y$, а знакът пред $g$ е отрицателен, защото посоката му е противоположна на оста $y$.\\
Когато тялото се издига, скоростта му намалява, в най-високата точка става нула, след което тялото започва да пада, скоростта му става отрицателна, понеже е насочена срещу оста $y$. В най-високата точка $V = 0$  или $0 = V_0 - gt$. От тук намираме времето, за което тялото ще достигне най-високата точка : $t = \dfrac{V_0}{g} = \dfrac{10}{10} = 1s$. Заместваме това време в израза за y за да получим максималната височина:
$$h_max = h_0 + V_0t -  \dfrac{gt^2}{2} = 1+ 10 \cdot 1 -  \dfrac{10 \cdot 1^2}{2} = 11 - 5 = 6m$$
Времето за падане до $h = 0$  намираме от условието $y = 0$.
$$0 = h_0 + V_0t - \dfrac{gt^2}{2} $$
$$0 = 1 + 10t -  \dfrac{10t^2}{2}$$
$$-5t^2 + 10t  + 1 = 0 $$
$$D = 10^2 - 4 \cdot (-5) \cdot 1 = 100 + 20 = 120 $$
$$t_{1,2} = \dfrac{-10 \pm \sqrt{120}}{2 \cdot (-5)} = \dfrac{-10 \pm 2\sqrt{30}}{-10} = \dfrac{-5 \pm \sqrt{30}}{-5}$$
$$t_1 = \dfrac{-5 -\sqrt{30}}{-5} \approx -0.1 s$$
$$t_2 = \dfrac{-5 +\sqrt{30}}{-5} \approx 2.1 s$$
Физичен смисъл има само положителното време. Заместваме го в израза за скоростта $V = V_0 - gt = 10 - 10 \cdot 2.1 = -11 m/s$ Това е скоростта, с която тялото пада на земята. Тя е отрицателна, защото е насочена срещу оста $y$.
\end{example}

\subsection{Движение при произволна форма на траекторията}
Когато движението не е праволинейно, скоростта и ускорението се записват за всяка от компонентите на радиус-вектора: 
$$V_x = \dfrac{dx}{dt}, \quad V_y = \dfrac{dy}{dt}, \quad V_z = \dfrac{dz}{dt} \qquad \vec{V} = \dfrac{d\vec{r}}{dt}$$
Скоростта $\vec{V}$ е векторна величина - тя се характеризира с големина и посока. Големината на скоростта се определя от координатите на скоростта (по Питагоровата теорема) $V = \sqrt{V_x^2 + V_y^2 + V_z^2}$ \\
Аналогично: 
$$\vec{a} = \dfrac{d\vec{V}}{dt} = \dfrac{d^2 \vec{r}}{dt^2}$$ 
Тъй като скоростта е вектор, тя може да се изменя поради промяна на големината и поради промяна на посоката си.\\
Ускорението, дължащо се на изменение на скоростта по големина се нарича тангенциално ускорение $\vec{a_\tau}$. То има посока, съвпадаща с направлението на скоростта.\\
Ускорението, дължащо се на изменение на скоростта по посока се нарича нормално ускорение $\vec{a_n}$. То има посока, перпендикулярна на направлението на скоростта. Може да се покаже, че $\vec{a_n} = \dfrac{V^2}{R}$, като R е радиусът на кривината на траекторията в разглежданата точка. \\
От казаното по-горе е ясно, че при праволинейно движение нормалното ускорение е винаги нула. При движение по крива, дори и с постоянна скорост, нормалното ускорение е различно от нула.
Пълното ускорение се получава като векторна сума от тангенциалното и нормалното ускорение:
$$\vec{a} =\vec{a_\tau} + \vec{a_n}$$
При постоянно ускорение законът за скоростта и за движение се записват във векторен вид: 
$$\vec{V} = \vec{V_0} + \vec{a}t$$
$$\vec{r} = \vec{r_0} + \vec{V_0}t + \dfrac{\vec{a}t^2}{2}$$
След това се записват уравненията за всяка от компонентите на векторите.

\begin{example}
Тяло е хвърлено под ъгъл $\alpha = 30^\circ $ спрямо хоризонта с начална скорост 
$V_0 = 10 \dfrac{m}{s}$. Намерете максималната височина, до която се издига тялото и разстоянието, което то прелита. \\
Решение: \\
$$\vec{V} = \vec{V_0} + \vec{g}t$$
$$\vec{r} = \vec{V_0}t + \dfrac{\vec{g}t^2}{2}$$
$\vec{r_0} = 0$. \\
Всеки от векторите може да бъде разложен на две компоненти $x$и $y$. \\
По оста $x$: $V_x = V_{0x}, \quad x = V_{0x}t$ \\
По оста $y$: $V_y = V_{0y} - gt, \quad y = V_{0y}t - \dfrac{\vec{g}t^2}{2}$ \\
Тук сме взели предвид, че по оста $x$ няма ускорение, а по оста $y$ ускорението е $g$, насочено надолу, в посока обратна на оста $y$(и затова с отрицателен знак). \\
$$V_{0x} = V_0 \cos{30^\circ} = \dfrac{\sqrt{3}}{2}V_0 \quad V_{0y} = V_0 \sin{30^\circ} = \dfrac{V_0}{2}$$
В най високата точка $V_y$ е равна на 0: $0 = V_{0y} - gt$ и времето за което тялото достига максимална височина е $t = \dfrac{V_{0y}}{g}$
Заместваме това време в израза за y и получаваме 
$$y_{max} = V_{0y} \dfrac{V_{0y}}{g} - \dfrac{g}{2} \cdot \left(  \dfrac{V_{0y}}{g} \right)^2 $$
$$y_{max} = \dfrac{V_{0y}^2}{g} - \dfrac{1}{2} \cdot \dfrac{V_{0y}^2}{g}$$ 
$$y_{max} = \dfrac{1}{2} \cdot \dfrac{V_{0y}^2}{g} =  \dfrac{1}{2} \cdot \dfrac{\left( \dfrac{10}{2} \right)^2}{10} = \dfrac{5^2}{20} = \dfrac{25}{20} = 1.25m$$
\end{example}

В общия случай, когато ускорението не е постоянно, законът за скоростта се получава с интегриране. 
$$\vec{a} =  \dfrac{d\vec{V}}{dt} \Leftrightarrow d\vec{V} = \vec{a}dt$$
Интегрираме и получаваме закона за скоростта в общия случай: 
$$\vec{V} = \int _0 ^ t  \vec{a}(t)dt + \vec{V_0}$$
Аналогично закона за движение : 
$$\vec{V} =  \dfrac{d\vec{r}}{dt} \Leftrightarrow d\vec{r} = \vec{V}dt$$
Интегрираме и получаваме закона за пътя в общия случай:
$$\vec{r} = \int _0 ^ t  \vec{V}(t)dt + \vec{r_0}$$
Ще използваме този резултат за да получим закона за движение при постоянно ускорение. При движение с постоянно ускорение  
$$\vec{V} = \int _0 ^ t  \vec{a}dt + \vec{V_0} = \vec{V_0} + \vec{a} \int _0 ^ t dt = \vec{V_0} + \vec{a}t$$
Заместваме този резултат в $\vec{r} = \int _0 ^ t  \vec{V}(t)dt + \vec{r_0}$ и получаваме 
$$\vec{r} = \int _0 ^ t (\vec{V_0} + \vec{a}t) dt + \vec{r_0} = \vec{r_0} + \int _0 ^ t \vec{V_0} dt + \int _0 ^ t \vec{a}t dt$$
$$ \vec{r} = \vec{r_0} + \vec{V_0}t +\dfrac{\vec{a}t^2}{2} $$

\newpage

\section{Лекция 2: Динамика на материална точка}

\subsection{Принципи на механиката (Принципи на Нютон)}
 
\subsubsection{Първи принцип}

\textbf{Всяко тяло запазва състоянието си на покой или на праволинейно равномерно движение, докато външно въздействие не го изведе от това състояние.}\\
Този принцип не е в сила за всички отправни системи. Отправните системи, за които той е в сила, се наричат \textit{инерциални отправни системи}. \\
Величината, която количествено характеризира взаимодействието между телата, се нарича сила. Силата е векторна величина – има големина, посока и приложна точка. Означава се с буквата F. Мерната единица за сила е нютон N.\\
Телата притежаватсвойството инертност – съпротивляват се на въздействие, което се стреми да ги извади от състоянието им на покой или на праволинейно равномерно движение. Това свойство се характеризира с величината маса m. Измерва се в килограми kg. Колкото по-голяма масата на едно тяло, толкова по-трудно е да изменим неговата скорост. \\
Произведението на масата и скоростта на едно тяло се нарича импулс $$\vec{p} = m \vec{V}$$
Mерната единица за импулс е  $ \dfrac{kg \cdot m}{s} $

\subsubsection{Втори принцип}
\textbf{Първата производна на импулса по времето е равна на силата, действаща на тялото.}
$$\vec{F} = \dfrac{d \vec{p}}{dt}$$
Ако масата не се променя можем да запишем
$$\vec{F} = \dfrac{d \vec{p}}{dt} = \dfrac{d m\vec{V}}{dt} = m \dfrac{d\vec{V}}{dt} = m \vec{a}$$
Mерната единица за сила $N = \dfrac{kg \cdot m}{s^2}$, Когато на тялото
действат няколко сили, $\vec{F}$ е векторната сума на тези сили.

\subsubsection{Трети принцип}
\textbf{Силите на взаимодействие между две тела са равни по големина и противоположни по посока.}

\subsection{Някои видове сили}

\subsubsection{Гравитационна сила}
Законът на Нютон за гравитацията гласи: \textit{Между всеки две материални точки действа сила на привличане, която е правопропорционална на произведението на масите им и обратно пропорционална на квадрата на разстоянието между тях.} \\
$$F = \gamma \dfrac{m_1 m_2}{r^2}$$
$\gamma = 6.67 \cdot 10^{-11} \dfrac{N \cdot m^2}{kg^2}$, $m_1, m_2$ - масите на двете материални точки, а r - разстоянието между тях. 

\begin{example}
Две тела с маси $m_1 =m_2 = 100 kg$ са разположени на разстояние r = 1 m.Намерете силата на привличане. \\
Решение: \\
$$F = \gamma \dfrac{m_1 m_2}{r^2} =  6.67 \cdot 10^{-11}  \dfrac{100 \cdot 100}{1^2} = 6.67 \cdot 10^{-7} N$$
\end{example}


\subsubsection{Сила на тежестта}
Разглеждаме тяло в близост до земната повърхност. Гравитационната сила, действаща на тялото в този случай ще означим с G, масата Земята с M, а разстоянието до центъра на Земята с R. Записваме закона за гравитацията:
$$G = \gamma \dfrac{mM}{R^2} = m \left( \gamma \dfrac{M}{R^2} \right)$$
$\gamma \dfrac{M}{R^2} $ е еднаква за всички тела величина, която се означава с g и се нарича земно ускорение. Стойността на $g \approx 9.8 \dfrac{m}{s^2}$ е измерена експериментално. Оттук може да запишем за силата на тежестта:
$$G = mg$$
При принципите на Нютон въведохме масата като мярка за инерчните свойства на телата. Тук даваме още едно определение за масата – тя характеризира гравитационните свойства на телата. Масата в  $\vec{F} = m \vec{a}$ се нарича инертна маса, а масата в $G = mg$ - тежка маса. Съгласно съвременната физика тежката и инертната маса са еквивалентни.

\subsubsection{Реакция на опората}
Разглеждаме книга поставена на един чин. Книгата действа на чина със силата на тежестта $G = mg$ , насочена надолу. Съгласно третия принцип на Нютон и чинът действа на книгата със същата по големина сила, но насочена нагоре. Тази сила се нарича реакция на опората и ще я означаваме с N. В крайна сметка на книгата действат две равни по големина сили G и N , насочени в противоположни посоки. Тяхната векторна сума е нула и затова книгата остава в покой. \\
Реакцията на опората винаги е перпендикулярна на повърхността, на която е поставено тялото.
 
\begin{example}
Тяло се спуска без триене по равнина, наклонена под ъгъл $\theta$. Определете ускорението на тялото и реакцията на опората. \\
Решение: \\
Избираме отправна система, при която оста х е успоредна на равнината, а оста у е перпендикулярна на равнината. Записваме втория принцип на Нютон $\vec{F} = m \vec{a}$. На тялото действат две сили: сила на тежестта и реакция на опората и следователно силата $\vec{F}$ е сума от тези две сили.
$$\vec{F} = \vec{G} + \vec{N} = m \vec{a} $$
Записваме това уравнение за всяка от осите
$$F_x = G_x = ma$$
$$F_y = N - G_y = 0$$
Тук сме отчели, че по оста у няма движение и ускорението е нула. В горните изрази: $G_x  =G \sin{\theta} = mg \sin{\theta}, G_y  =G \cos{\theta} = mg \cos{\theta}$\\
$$G_x = mg \sin{\theta} = ma$$
$$a =g\sin{\theta}$$
$$N = G_y = mg \cos{\theta}$$
\end{example}

\subsubsection{Сила на триене}
Разглеждаме тяло, поставено върху хоризонтална поставка. Между молекулите на тялото и на поставката възникват електромагнитни сили, които се противопоставят на движението на тялото спрямо поставката. Тези сили се наричат сили на триене. Силата на триене винаги е насочена срещу посоката на движение (на фигурата външната сила F движи тялото надясно, а силата на триене f е насочена наляво. Големината на силата на триене е пропорционална на реакцията на опората N.
$$f = kN$$
Коефициентът на пропорционалност k се нарича коефициент на триене и зависи от материала, от който са изработени триещите се повърхности, грапавините и други.

\begin{example}
Автомобил с маса $m = 1000 kg$ се движи по хоризонтален път със скорост $V_0 = 54 km/h$. След задействане на спирачкитеавтомобилът спира за време 5 s. Определете силата на триене и коефициента на триене.\\
Решение: \\
Като имаме предвид, че 1 m/s = 3,6 km/h, намираме, че v0 = 54 km/h = 15 m/s. \\
$$V = V_0 - at$$
В момента на спиране $V = 0$ и $0 = V_0 - at$ или $a = \dfrac{V_0}{t} = \dfrac{15}{5} = 3 \dfrac{m}{s^2}. $\\
От $f = ma$ получаваме $f =1000 \cdot 3 = 3000 N$ \\
f = kN Тук, понеже сме на хоризонтален път, N = mg и f = kmg. Последно:
$$k = \dfrac{f}{mg} = \dfrac{3000}{1000 \cdot 10} = 0.3$$
\end{example}

\begin{example}
Шейна се движи по хоризонтална повърхност, като коефициентът на триене между шейната и снега е k. Теглим шейната със сила Т насочена под ъгъл $\theta$. Напишете уравненията за силите, действащи на шейната \\
Решение: 
Записваме втория принцип на Нютон: $\vec{F} = m \vec{a}$, $\vec{F} = \vec{T} + \vec{G} + \vec{f} + \vec{N}$ е векторна сума от всички сили, действащи на шейната: силата Т , с която теглим, силата на тежестта $G = mg$, силата на триене $f = kN$и силата на реакция на опората N. \\
Записваме уравненията за силите по всяка ос: \\
по x: $T_x - f = ma $\\
по y: $ T_y + N - G = 0$\\
като: $ T_x = T \cos{\theta}, T_y = T \sin{\theta}$ \\
От второто уравнение $N = G - Ty$; т.е. реакцията на опората е по-малка от силата на тежестта, защото ние теглим нагоре. Като заместим тази стойност в силата на триене в първото уравнение, можем да намерим ускорението.
\end{example}

\subsection{Инерциални и неинерциални отправни системи. Класически принцип на относителността}
Нека инерциалната система К условно е неподвижна, а системата К' се движи спрямо нея праволинейно равномерно със скорост $\vec{V_0}$. Възниква въпросът: изменят ли се законите на класическата механика при преход от една инерциална система К в друга инерциална система К'? Получените резултати по този въпрос се формулират като \textit{класически принцип на относителността (принцип на Галилей за относителността)}. Той гласи: \textit{законите на класическата механика са еднакви във всички инерциални системи}. \\
По късно Айнщайн в теорията на относителността е допълнил принципа като \textit{всички закони на природата са еднакви във всички инерциални системи}. (Става дума не само за законите на механиката, а за всички закони.)\\
Дотук – системата К' се движи равномерно спрямо К. Нека системата К' се движи с ускорение $a_0$ спрямо К. В този случай К' вече не е инерциална отправна система. На всички тела в К' ще действа инерчна сила $F_e = -m a_0 $ насочена в посока обратна на $a_0$. Например автобус потегля с ускорение. Всички пътници политат назад (в посока обратна на ускорението). \\
Следващия случай, който ще разгледаме е неинерциална система, която се върти с постоянна скорост. В този случай инерчната сила се нарича центробежна сила. Тя е насочена перпендикулярно на оста на въртене и се стреми да отдалечи материалната точка от оста на въртене.
$$F_c = m a_n = \dfrac{mV^2}{r}$$

\begin{example}
С каква скорост се движи един изкуствен спътник на Земята?\\
Решение:  За да стане едно тяло изкуствен спътник трябва центробежната сила да компенсира
гравитационната сила.
$$\gamma \dfrac{mM}{R^2} = \dfrac{mV^2}{r}$$
m - маса на спътника, М - масата на земята, r - радиуса на орбитата. 
$$V = \sqrt{\gamma \dfrac{M}{r}} = \sqrt{6.67 \cdot 10^{-11} \dfrac{5.9 \cdot 10^{24}}{6700 \cdot 1000}} = 7664 \dfrac{m}{s} \approx 8 \dfrac{km}{s}$$
Тук сме приели, че радиусът на орбитата е 6700 km, т.е. спътникът се движи на около 330 km над земната повърхност. (От резултата се вижда, че при друг радиус на орбитата на спътника, неговата скорост ще е различна.)
\end{example}

\begin{example}
Автомобил с маса $m = 1000 kg$ се движи по завой с радиус $r = 100 m$ и коефициент на триене между гумите и асфалта $k =0.4$. С каква максимална скорост може да се движи колата без да излезе от пътя?\\
Решение: \\
Реакцията на опората $N = G = mg = 1000 \cdot 10 = 10 000 N$\\
Силата на триене $f = kN$ трябва да е по голяма или равна на центробежната сила, за да остане колата на пътя 
$$kN =  \dfrac{mV^2}{r}$$
$$V = \sqrt{\dfrac{kNr}{m}} = \sqrt{\dfrac{0.4 \cdot 10 000 \cdot 100}{1000}} = \sqrt{400} = 20 \dfrac{m}{s}$$
\end{example}


\subsection{Импулс. Закон за запазване на импулса}

Импулс на Сила: От втория принцип на Нютон: $\vec{F} = \dfrac{d \vec{p}}{dt}$, може да получим $d \vec{p} =\vec{F}dt =$. Изменението на импулса е равно на силата, умножена по времето, за което е станало това изменение. \\
Втория принцип на Нютон: $\vec{F} = \dfrac{d \vec{p}}{dt}$ е в сила и за системи, състоящи се от много тела. В този случай $\vec{p}$ е векторна сума на импулсите на всички тела, изграждащи системата, а $\vec{F}$ е векторна сума от всички действащи сили. Ако сумата от силите е нула, то
$$0 = \dfrac{d \vec{p}}{dt}$$
Щом производната на една величина е нула, то тази величина е константа.
$$\vec{p} = const$$
Закон за запазване на импулса (ЗЗИ): \textbf{Импулсът на затворена система от материални точки е постоянна величина}.\\ 
Затворена система в случая означава система, на която не действат външни сили (или сумата на външните сили е нула).

\begin{example}
Върху неподвижен ($V_{01} = 0 \dfrac{m}{s}$) скейтборд с маса $m1 = 5 kg$ скача дете с маса $m2 = 45 kg$ и хоризонтална скорост
$V_{02} = 2 \dfrac{m}{s}$ и остава върху него. Намерете скоростта на детето със скейтборда.\\
Решение: \\
Преди скока:\\
Импулс на скейтборда $p_{01} = m_1 V_{01} = 0 \dfrac{kg \cdot m}{s}$ \\
Импулс на детето $p_{02} = m_2 V_{02} = 45 \cdot 2 = 90 \dfrac{kg \cdot m}{s} $ \\
Общ импулс на системата преди скока:
$$p_0 = p_{01} + p_{02} = m_1 V_{01} + m_2 V_{02} = 0 + 90 = 90 \dfrac{kg \cdot m}{s}$$
След скока:  
Импулс на скейтборда $p_1 = m_1 V $ \\
Импулс на детето $p_2 = m_2 V $\\
Скоростта V е с която се пързаля и еднаква за детето и скейтборда. \\
Общ импулс на системата след скока: $p = p_1 + p_2 = m_1 V +  m_2 V  = (m_1 + m_2)V$ \\
Прилагаме закона за запазване на импулса: $p_0 = p$
$$m_1 V_{01} + m_2 V_{02} = (m_1 + m_2)V $$
$$V = \dfrac{m_1 V_{01} + m_2 V_{02}}{m_1 + m_2} = \dfrac{90}{50} = 1.8 \dfrac{m}{s}$$

\end{example}

\begin{example}
От пушка с маса $m_1 = 5 kg$ се изстрелва куршум с маса $m_2 = 5 g$ и скорост $V_2 = 100 \dfrac{m}{s}$. Намерете скоростта на отката на пушката. \\
Решение: \\
В началото преди изстрела всички тела са неподвижни и импулсът е $p_0 = 0$. След изстрела $p = p_1 + p_2 = m_1V_1 + m_2V_2$ Тук $V_1$ е неизвестната скорост на пушката след отката. 
ЗЗИ: $p_0 = p$ или $0 =m_1V_1 + m_2V_2$
$$V_1 = -\dfrac{m_2}{m_1} V_2 = -\dfrac{0.005}{5} 100 = -0,1 \dfrac{m}{s}$$
Знакът минус означава, че посоката на скоростта на отката е противоположна на посоката на куршума
\end{example}

\subsection{Работа и мощност}

\subsubsection{Работа}
Нека върху т.М действа сила $\vec{F}$ и тя извършва безкрайно малко преместване $d \vec{r}$. Скаларното произведение на силата и преместването се нарича елементарна работа. $dA = \vec{F} \cdot d \vec{r}$. \\
Съгласно свойствата на скаларното произведение на два вектора, изразът може да се представи във вида $dA = Fdr\cos{\alpha}, \alpha = \sphericalangle ( \vec{F};d \vec{r}) $ \\
Нека компонентите на силата $\vec{F}$ са $F_x , F_y, F_z$ , a компонентите на преместванeтo $d\vec{r}$ сa $dx , dy , dz$. Съгласно свойствата на скаларното произведение 
$$dA = F_xdx + F_ydy + F_zdz$$
\\
При произволно преместване от положение 1 с радиус-вектор $\vec{r_1}$ до положение 2 с радиус-вектор $\vec{r_2}$, работата и силата се определят с интегриране на елементарните работи. 
$$A = \int_{r_1} ^{r_2} \vec{F} d\vec{r} = \int_{r_1} ^{r_2} F \cos{\alpha} dr$$
Ако силата е постоянна и ъгълът не се мени
$$A = \int_{r_1} ^{r_2} F \cos{\alpha} dr = F \cos{\alpha} \int_{r_1} ^{r_2} dr =  F \cos{\alpha} (r_2 - r_1) =  F \cos{\alpha} \Delta r$$ 
\\
Единицата SI за величината работа е $N \cdot m$ = J (джаул). Работа A = 1 J е работата, извършена от сила 1 N при преместване на тялото на разстояние 1 m.

\subsubsection{Мощност}
Мощност на сила $\vec{F}$ се нарича отношението на елементарната работа dA, извършена от силата за интервал от време dt, към този интервал dt, т.е.
$$P = \dfrac{dA}{dt}$$
\\
Мощността е скаларна величина. В SI [P] =W (ват). Мощността на силата е 1 W, когато силата, извършва работа 1 J за време 1 s. 

\subsection{Енергия}

\subsubsection{Кинетична енергия}
Може да се покаже, че извършената работа А за промяна на скоростта на тяло от начална стойност $v_1$ до крайна стойност$v_2$ е равна на 
$$A_{12} = E_{k_2} - E_{k_1} $$
$E_k$ се нарича кинетична енергия и се дава с формулата 
$$E_k = \dfrac{mV^2}{2}$$
Този резултат показва, че механичната работа е равна на разликата между крайната и началната стойност на кинетичната енергия на материалната точка. 

\begin{example}
Камък с маса 2 kg е хвърлен вертикално надолу, като за даден период от време увеличава скоростта си от 5 на 10 $\dfrac{m}{s}$. Намерете работата, извършена от силата на тежестта.
$$A = E_{k_2} - E_{k_1} = \dfrac{mV_2^2}{2} - \dfrac{mV_1^2}{2} = \dfrac{2 \cdot 10^2}{2} \dfrac{2 \cdot 5^2}{2} = 75J$$
\end{example}

\subsubsection{Консервативни сили и потенциална енергия}

Консервативни сили: \textbf{Консервативни сили се наричат силите, работата на които не зависи от вида на траекторията, а се определя само от началното и крайното положение на материалната точка. Работата на консервативните сили, извършена за всяка затворена траектория винаги е равна на нула}. \\
\\
Потенциална енергия $E_p: A_{12} = E_{p_1} - E_{p_2}$
$$E_p = mgh$$

\subsubsection{Закон за запазване на енергията}
Работата, която извършват консервативните сили в затворена система, в която действат само консервативни сили може да се изрази:\\
чрез кинетичната енергия: $A_{12} = E_{k_2} - E_{k_1}$ \\
чрез потенциалната енергия: $A_{12} = E_{p_1} - E_{p_2}$ \\
Откъдето $E_{p_1} - E_{p_2} = E_{k_2} - E_{k_1}$ и $E_{k_1} + E_{p_1} = E_{k_2} + E_{p_2} $. $E = E_k + E_p$ се нарича пълна механична енергия. \\
Закон за запазване на енергията(ЗЗЕ): \textbf{В една затворена механична система, в коятодействат само консервативни сили, пълната механична енергия е константа.}

\begin{example}
Тяло пада от височина $h_0 = 20m$ без начална скорост. С каква скорост тялото ще достигне земята? \\
Решение:\\
В момента на хвърляне: \\
$E_{k_1} = 0, E_{p_1} = mgh_0 \implies E_1 = E_{k_1} + E_{p_1} = mgh_0$. \\
На земята: \\
$E_{k_2} = \dfrac{mV^2}{2}, E_{p_2} = 0 \implies E_2 = E_{k_2} + E_{p_2} = \dfrac{mV^2}{2}$. \\
ЗЗЕ: 
$$E_1 = E_2$$
$$mgh_0 = \dfrac{mV^2}{2}$$
$$V = \sqrt{2gh_0} = \sqrt{2\cdot 10 \cdot 20} = 20 \dfrac{m}{s}$$
\end{example}

\begin{example}
Тяло е хвърлено вертикално нагоре от височина $h_0 =  1m$ с начална скорост $v_0 = 10 \dfrac{m}{s}$. До каква максимална височина ще се издигне тялото?\\
Решение:\\
В момента на хвърляне: \\
$E_{k_1} = \dfrac{mV_0^2}{2}, E_{p_1} = mgh_0 \implies E = E_{k_1} + E_{p_1} =  \dfrac{mV_0^2}{2} + mgh_0 $ \\
На максимална височина: \\
$E_{k_2} = 0 , E_{p_2} = mgh_{max} \implies E_2 = E_{k_2} + E_{p_2} =  mgh_{max} $. \\
ЗЗЕ: 
$$E_1 = E_2$$
$$mgh_{max} = \dfrac{mV_0^2}{2} + mgh_0  $$
$$h_{max} = h_0 + \dfrac{V_0^2}{2g} = 1 + \dfrac{10^2}{2 \cdot 10} = 6m$$
\end{example}


\newpage
\section{Лекция 3: Механика на идеално твърдо тяло}

\subsection{Кинематика на въртеливо движение на материална точка}
При въртеливо движение траекторията на всяка точка от тялото е окръжност. Нека дадена точка се е завъртяла на ъгъл $\Delta \varphi$
за време $\Delta t$

\subsubsection{Ъглова скорост}
Ъглова скорост $\vec{\omega}$ се нарича векторната величина $\omega = \dfrac{\Delta \varphi}{\Delta t}$. Когато въртенето е в посока обратна на часовниковата стрела, посоката на вектора е от равнината на въртене нагоре. \\
$\left[ \omega \right] = \dfrac{rad}{s}$

\subsubsection{Моментна ъглова скорост}
Когато интервалът време е много малък, дефинираме моментна ъглова скорост $\omega = \lim\limits_{\Delta t \rightarrow 0} \dfrac{\Delta \varphi}{\Delta t} = \dfrac{d \varphi}{dt}$

\subsubsection{Ъглово ускорение}
Ъглово ускорение $\vec{\alpha}$ се нарича векторната величина: $\vec{\alpha} = \lim\limits_{\Delta t \rightarrow 0} \dfrac{\Delta \vec{\omega}}{\Delta t} = \dfrac{d \vec{\omega}}{dt} $ \\
$\left[ \alpha \right] = \dfrac{rad}{s^2}$\\
\\
При ускорително движение, посоката на $\vec{\alpha}$ и $\vec{\omega}$ съвпадат, а при закъснително - посоката на $\vec{\alpha}$ е обратна на $\vec{\omega}$. \\
По подобие на закона за скоростта и закона за движение при праволинейно движение и тук можем да напишем:
$$\omega = \omega_0 + \alpha t$$
$$\varphi = \varphi_0 + \omega_0t + \dfrac{\alpha t^2}{2}$$
При завъртане на ъгъл $\Delta \varphi$ материалната точка изминава път
$$\Delta s = R\Delta \varphi$$
При малък ъгъл на завъртане $\Delta s \approx \Delta r, \Delta r =  R\Delta \varphi$.  Делим двете страни на това равенство на $\Delta t$ и получаваме връзка между линейна скорост $\vec{V}$ и ъглова скорост $\vec{\omega}$
$$V = \lim\limits_{\Delta t \rightarrow 0} \dfrac{\Delta r}{\Delta t} = \lim\limits_{\Delta t \rightarrow 0} R \dfrac{\Delta \varphi}{\Delta t} = R \Delta \omega$$
Тангенциалното ускорение се получава: 
$$a_t = \dfrac{dV}{dt} = R \dfrac{\Delta \omega}{\Delta t} = R \alpha$$
А нормалното ускорение: 
$$a_n = \dfrac{V^2}{R} = R \omega^2$$

\begin{example}
Да се определи ъгловата скорост, скоростта и нормалното ускорение за точка от екватора на Земята. Радиусът на Земята е R = 6370 km.\\
Решение: \\
Знаем, че Земята прави едно завъртане около оста си за 24 часа. Т.е. времето за завъртане на ъгъл $\Delta \varphi = 2 \pi$ (едно пълно завъртане) е 
$$\Delta t = 24h = 24h \cdot 60m \cdot 60s = 86400 s$$
$$\omega =  \dfrac{\Delta \varphi}{\Delta t} =  \dfrac{2 \cdot 3.14}{86400} = 7,27 \cdot 10^{-5} \dfrac{rad}{s}$$
$$V = R\omega = 6370000 \cdot  7,27 \cdot 10^{-5} = 463 \dfrac{m}{s} = 1666 \dfrac{km}{h} $$
$$a_n = \dfrac{V^2}{R} = 0.03 \dfrac{m}{s^2}$$
\end{example}

\subsection{Момент на сила и момент на импулса. Основно уравнение}

При въртеливо движение вторият принцип на Нютон се записва с по-различна форма. Нека т.О е произволна неподвижна точка (начало на отправната система), а в т.А се намира частица с маса m, на коятодейства сила $\vec{F}$. Записваме втория принцип на Нютон: $\vec{F} = \dfrac{d \vec{p}}{dt}$. Умножаваме двете страни на това равенство векторно по $\vec{r}$, където $\vec{r}$ е радиус вектора от т.О до т.А. 
$$\vec{r} \times \vec{F} = \vec{r} \times \dfrac{d \vec{p}}{dt} $$
$\vec{M} = \vec{r} \times \vec{F} $ - момент на сила спряло т.О \\
$\vec{L} = \vec{r} \times \vec{p} = \vec{r} \times m\vec{V}$  - момент на импулса \\
От свойствата на векторното произведение може да се получи $\dfrac{d \vec{L}}{dt} =  \vec{r} \times \dfrac{d \vec{p}}{dt}$ или
$$\vec{M} = \dfrac{d \vec{L}}{dt}$$
За система от n материални точки (тяло): \\
Моментът на импулса е равен на векторната сума от моментите на импулсите 
$$\vec{L} = \sum_{i=1} ^n \vec{L_i} = \sum_{i=1} ^n \left( \vec{r_i} \times \vec{p_i} \right)  $$
Момента на силата е равен на векторната сума на действащите сили \\
\\
За затворена система от материални точки (т.е. не действат външни сили и сумата от силите на взаимодействие между частиците е нула) 
$$\vec{M} = 0 \implies \dfrac{d \vec{L}}{dt} = 0 \implies \vec{L} = const$$
Този резултат е израз на закона за запазване на момента на импулса. Той гласи: \textbf{моментът на импулса $\vec{L}$ на затворена система от материални точки (тяло) остава постоянен.}

\subsection{Инерчен момент}
$I = mr^2$ - инерчен момент на материална точка. \\
За система от n материални точки инерчният момент е сума от инерчните моменти на отделните точки 
$$I = \sum_{i=1} ^n m_i r_i^2$$
При тела разделяме мислено тялото на много малки части всяка с маса dm. Инерчният момент на всяка част е $dI = r^2dm$ Инерчният момент на тялото намираме чрез интегриране
$$I = \int dI = \int r^2dm $$

\subsection{Въртене около постоянна ос}
Разглеждаме материална точка, която се движи по окръжност с радиус r около постоянна ос. В този случай  $\vec{r} \perp \vec{V}, \, \vec{L} = \vec{r} \times \vec{p} = \vec{r} \times m\vec{V} \implies L = rmV $. От $V = r \omega \implies L =mr^2  \omega, \, I = mr^2 \implies $ 
$$L = I\omega$$
От $\vec{M} = \dfrac{d \vec{L}}{dt},\, L = I\omega \implies  M = \dfrac{d(I\omega)}{dt} = I \dfrac{d(\omega)}{dt} = I\alpha$
$$M = I\alpha $$

\subsection{Аналогия между величини при постъпателно и въртеливо движение}

\begin{center}
\begin{tabular}{ |c|c| } 
 \hline
 \textbf{Постъпателно движение} & \textbf{Въртеливо движение} \\ 
\hline
Радиус-вектор $\vec{r} $ & Ъгъл на завъртане $\varphi$ \\ 
\hline
 Скорост $\vec{V} = \dfrac{d \vec{r}}{dt}$ & Ъглова скорост  $\omega = \dfrac{d\varphi}{dt}$\\ 
 \hline
 Ускорение $\vec{a} = \dfrac{d \vec{V}}{dt}$ & Ъглова скорост  $\vec{\alpha} = \dfrac{d\vec{\omega}}{dt}$\\ 
 \hline
 Закон за скоростта $V = V_0 + at$ & Закон за ъгловата скорост $\omega = \omega_0 + \alpha t$ \\
\hline
Закон за движение $x = x_0 + V_0t + \dfrac{at^2}{2}$ & Закон за движение  $\varphi = \varphi_0 + \omega_0t + \dfrac{\alpha t^2}{2}$  \\
\hline
Маса m & Инерчен момент I\\
\hline
Сила $\vec{F}$ & Момент на сила $\vec{M} = \vec{r} \times \vec{F} $ \\
\hline
Импулс $\vec{p} = m\vec{V}$ & Момент на импулса $\vec{L} = \vec{r} \times \vec{p}, L = I\omega $\\
\hline
Основно уравнение $ \vec{F} = \dfrac{d \vec{p}}{dt}, \quad  \vec{F} = m\vec{a} $ & Основно уравнение $\vec{M} = \dfrac{d \vec{L}}{dt}, \quad  M = I\alpha$ \\
\hline
Работа $dA = \vec{F} d\vec{\varphi}$ & Работа $dA = Md\varphi$ \\
\hline
Кинетична енергия $E_k = \dfrac{mv^2}{2}$ &  Кинетична енергия $E_k = \dfrac{I \omega^2}{2}$\\
\hline
\end{tabular}
\end{center}

\subsection{Приложения и примери}

\subsubsection{Момент на сила}
В т. А от дадено тяло е приложена сила, която предизвиква въртене на тялото около т. О. От свойствата на векторното произведение $\vec{M} = \vec{r} \times \vec{F}$ се вижда, че в този случай векторът на момента на силата е насочен от листа към нас. Големината на момента на силата е $M = rF\sin{\alpha} = Fr\sin{\alpha} = Fl$. Величината $l = r\sin{\alpha}$ се нарича рамо на силата. Рамото на силата е винаги перпендикулярно към правата, по чиято дължина действа силата.

\begin{example}
$l_1 = 1 m,\, F_1 = 20,\, l_2 = 0.1m\, F_2 = ?\\
\alpha = \dfrac{F_1 l_1}{I} = \dfrac{F_2 l_2}{I} \implies F_1 l_1 = F_2 l_2 \\
F_2 = F_1 \dfrac{l_1}{l_2} = 20 \dfrac{1}{0.1} = 200 N \implies \text{силата нараства 10 пъти.}
$
\end{example}

\begin{example}
$l_1 = 1 m,\, F_1 = 20,\, l_2 = 0.1m,\, F_2 = 200N,\,\alpha = \dfrac{\pi}{2}, \\
A_1 =?,\, A_2 = ? \\
s_1 = \dfrac{2\pi l_1}{4} = 1.57m\\
s_2 = \dfrac{2\pi l_2}{4} = 0.157m\\
dA = Md\varphi \implies \\
A_1 = M_1 \dfrac{\pi}{2} = F_1r_1 \dfrac{\pi}{2} = 20 \cdot 1 \cdot \dfrac{3.14}{2} = 31.4J\\
A_2 = M_2 \dfrac{\pi}{2} = F_2r_2 \dfrac{\pi}{2} = 200 \cdot 0.1 \cdot \dfrac{3.14}{2} = 31.4J\\
\implies A_1 = A_2
$
\end{example}

\subsubsection{Закон за запазване на момента на импулса}
От $L = I\omega = const$ е ясно, че когато инерчният момент расте, ъгловата скорост намалява и обратно.

\begin{example}
Човек седи на въртящ се стол и се върти с ъглова скорост $ \omega_0 = 5 rad/s $ Инерчният момент на човека със стола е $I_h = 1 \dfrac{kg}{m^2}$. Човекът държи две гири всяка с маса $m = 2 kg$. В началото човекът държи гирите на оста на въртене, а после си разперва ръцете (приемете че дължината на ръцете на човека е $r = 1 m$).\\
Намерете ъгловата скорост.\\
В началото гирите са по оста и инерчният им момент е нула. Началният инерчен момент $I_0$ е равен на инерчният момент на човека $I_h$ и $L_0 = I_0 \omega_0$ След като човекът си разпери ръцете $I = I_h + 2I_w = I_h + 2mr^2 = 1 + 2 \cdot 2 \cdot 1 = 5 \dfrac{kg}{m^2}$ и $L = I \omega$. От закона за запазване на момента на импулса
$$L_0 = L \Leftrightarrow I_0 \omega_0 = I\omega$$
$$\omega = \dfrac{I_0}{I} \omega_0 = \dfrac{1}{5} 5 = 1 \dfrac{rad}{s}$$
\end{example}

\subsubsection{Условия за равновесие}
За да бъде едно тяло в равновесие, трябва да бъдат изпълнени следните две условия: \\
Сумата от всички сили трябва да е нула:
$$\vec{F_1} + \vec{F_2} + \vec{F_3} + ... = 0$$
Сумата от всички моменти на сили трябва да е нула:
$$\vec{M_1} + \vec{M_2} + \vec{M_3} + ... = 0$$

\newpage
\section{Лекция 4: }

\newpage
\section{Формули}

\subsection{Лекция 1:}

\subsection{Лекция 2:}

\subsection{Лекция 3: }

\subsection{Лекция 4: }











\end{document}