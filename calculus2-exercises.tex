\documentclass[a4paper,fleqn,12pt]{article}

\usepackage[utf8]{inputenc}
\usepackage[bulgarian]{babel}
\usepackage{amsmath}
\usepackage{amssymb}
\usepackage{booktabs}
\usepackage{fancyhdr}
\usepackage{amsthm}
\usepackage{graphicx}

\theoremstyle{definition}
\newtheorem{theorem}{Tеорема}[subsection]
\newtheorem{lemma}[theorem]{Лема}

\newtheorem{definition}{Дефиниция}[subsection]

\newtheorem{example}{Пример}[subsection]

\newtheorem{task}{Задача}[subsection]



\title{Математически анализ 2 \\ Упражнения}
\author{Exonaut}

\pagestyle{fancy}
\fancyhf{}
\lhead{\rightmark}
\rhead{\thepage}
\cfoot{}
\renewcommand{\headrulewidth}{0pt}

\begin{document}

\pagenumbering{gobble}
\maketitle

\newpage
\pagenumbering{arabic}

\tableofcontents
\newpage

\section{Упражнение към лекция 1}
\subsection{Задачи}

\subsection*{Задача 1}
Да се покаже дали посочените редици $\{ X_n \} = \{ x_n, y_n \}$ са сходящи или разходящи. За сходящите да се намери границите им.\\
\begin{enumerate}
\item $x_n = 1 + \dfrac{1}{n}, \, y_n = 2 + \dfrac{\sin{n}}{n}$
\item $x_n = \left( 1 + \dfrac{1}{n} \right) ^n, \, y_n = 2 + n $
\item $x_n = (-1)^n, \, y_n = n$
\item $x_n = (-1)^n, \, y_n = \dfrac{1}{n}$
\item $x_n =\sin{\dfrac{n \pi }{2}}, \, y_n = (-1)^n$
\item $x_n = \sin{n}, \, y_n = \dfrac{(-1)^n}{n}$
\end{enumerate}

\newpage
\subsection{Решения}

\subsection*{Задача 1}
\begin{enumerate}

\item $\lim\limits_{n \to \infty} \dfrac{1}{n} = 0, \, \dfrac{\vert \sin{n} \vert}{n} \in \left[0, \dfrac{1}{n} \right] \implies 
\lim\limits_{n \to \infty} x_n = 1, \, \lim\limits_{n \to \infty} y_n = 2 \implies 
\text{редицата е сходяща; точката (1,2) е нейна граница}$

\item $ \lim\limits_{n \to \infty} x_n = e , \lim\limits_{n \to \infty} y_n = \infty \implies 
\text{разходяща редица} $

\item $ \lim\limits_{n \to \infty} x_n \text{не съществува, защото има две точки на сгъстяване.}, \,  
\lim\limits_{n \to \infty} y_n = \infty \implies \text{разходяща редица}$

\item $\lim\limits_{n \to \infty} x_n \text{не съществува, защото има две точки на сгъстяване.}, \,  
\lim\limits_{n \to \infty} y_n = 0 \implies \text{разходяща редица}$

\item $\lim\limits_{n \to \infty} x_n \text{не съществува}, \,  
\lim\limits_{n \to \infty} y_n = \infty \implies \text{разходяща редица}$

\item $\lim\limits_{n \to \infty} x_n \text{не съществува}, \,  
\lim\limits_{n \to \infty} y_n = 0 \implies \text{разходяща редица}$
\end{enumerate}

\newpage
\section{Упражнение към лекция 2}

\subsection{Задачи}

\subsection*{Задача 1}
Нека $D \subset \mathbb{R}^m$ и са разгледани няколко функции. Да се напишат дефиниционните им множества и да се даде пояснение.  
\begin{enumerate}
\item $z(x, y) = x^2 + y^2 $
\item $z(x, y) = \sqrt{ y^2 - 2x }$
\item $z(x, y) = \ln \sqrt{ y^2 - 2x } $
\item $z(x, y) = \dfrac{1}{\sqrt{ -y^2 + 2x + 1}}$
\item $w(x, y, z) = \arccos(x^2 + y^2 + z^2)$
\item $f(n) = 
\begin{cases}
1, & x\in \mathbb{Q}^m \\
0, & x \in \dfrac{\mathbb{R}^m}{\mathbb{Q}^m}
\end{cases}
$
\end{enumerate}

\subsection*{Задача 2}

Разгледаните по - долу функциите са дефинирани в $D = \mathbb{R}^2 \setminus \{ (0,0) \}$. Кои от границите същестуват и колко са
$$
A = \lim\limits_{(x,y) \to (0,0)} f(x,y) \quad
A_{1,2} = \lim\limits_{y \to 0} \left( \lim\limits_{x \to 0} f(x,y) \right) \quad
A_{2,1} = \lim\limits_{x \to 0} \left( \lim\limits_{y \to 0} f(x,y) \right) \quad
$$

\begin{enumerate}
\item $f(x,y) = \dfrac{x-y}{x+y}$
\item $f(x,y) = \dfrac{x^2 + y^2}{x^2y^2 + (x - y)^2}$
\item $f(x,y) = \dfrac{xy^2}{x^2+y^4}$
\item $f(x,y) = (x+y) \sin{\dfrac{1}{x}} \cos{\dfrac{1}{y}}$
\item $f(x,y) = \dfrac{x^4 + y^4}{x^2 + y^2}$
\end{enumerate}

\subsection*{Задача 3}
Нека $A,B,C,D$ са подмножества на $\mathbb{R}^2$ дефинирани както следва \\
$
A = \{ (x,y): x \geq 0, y \leq 1, y>x \} \\
B = \{ (x,y): x \leq 1, y \geq 0, y<x \} \\
C = \{ (x,y): x = y , 0 \leq x \leq 1 \} \\
D = A \cup B \cup C
$ \\
и функцията $f: D \to \mathbb{R} $ зададена по следния начин \\
$f(x,y) = 
\begin{cases}
\dfrac{1}{y^2}, & (x,y)\in A \\
0, & x = y \\
-\dfrac{1}{x^2}, & (x,y)\in B
\end{cases}$\\ 
Да се изследва непрекъснатостта на тази функция. \\

\newpage
\subsection{Решения}

\subsection*{Задача 1}
\begin{enumerate}
\item $z(x, y) = x^2 + y^2 \\ D= \mathbb{R}^2$

\item $z(x, y) = \sqrt{ y^2 - 2x } \\ D= \{ (x,y): y^2 - 2x \geq 0\} \subset \mathbb{R}^2, x \leq \dfrac{y^2}{2}$

\item $z(x, y) = \ln \sqrt{ y^2 - 2x } \\ D= \{ (x,y): y^2 - 2x > 0\} \subset \mathbb{R}^2, x < \dfrac{y^2}{2}$

\item $z(x, y) = \dfrac{1}{\sqrt{ -y^2 + 2x + 1}}\\ D= \{ (x,y):  -y^2 + 2x + 1 > 0\} \subset \mathbb{R}^2, x > \dfrac{y^2 - 1 }{2}$

\item $w(x, y, z) = \arccos(x^2 + y^2 + z^2) \\ D= \{ (x,y,z):  x^2 + y^2 + z^2 \leq \pi \} \subset \mathbb{R}^3, \\ 
\text{Графиката е кълбо с център (0,0,0) и радиус} \sqrt{\pi}$

\item $D \subset \mathbb{R}^m $
\end{enumerate}

\subsection*{Задача 2}

\begin{enumerate}
\item 
\begin{gather*}
f(x,y) = \dfrac{x-y}{x+y} \\
\lim\limits_{x \to 0} f(x,y) = \dfrac{-y}{y} = -1 \qquad 
\lim\limits_{y \to 0} f(x,y) = \dfrac{x}{x} = 1 \\
A_{1,2} = \lim\limits_{y \to 0} \left( \lim\limits_{x \to 0} f(x,y) \right) = \lim\limits_{y \to 0} \left( -1 \right) = -1\\
A_{2,1} = \lim\limits_{x \to 0} \left( \lim\limits_{y \to 0} f(x,y) \right) = \lim\limits_{x \to 0} \left( 1 \right) = 1 \\
A = \lim\limits_{(x,y) \to (0,0)} f(x,y) \text{ Не съществува, защото трябва }A_{1,2} = A_{2,1}
\end{gather*}

\item 
\begin{gather*}
f(x,y) = \dfrac{x^2 + y^2}{x^2y^2 + (x - y)^2}\\
\lim\limits_{x \to 0} f(x,y) = \dfrac{y^2}{(-y)^2} = 1 \qquad 
\lim\limits_{y \to 0} f(x,y) = \dfrac{x^2}{x^2} = 1 \\
\implies A_{1,2} = A_{2,1} = 1 \implies \exists A = \lim\limits_{(x,y) \to (0,0)} f(x,y)\\
\text{Редица: }(x_n, y_n) = \left (\dfrac{1}{n},\dfrac{1}{n}\right ) \to (0,0), f(x_n, y_n) = 1 \to 1\\
\text{Редица: }(x'_n, y'_n) = \left (\dfrac{1}{n},\dfrac{-1}{n}\right )  \to (0,0), f(x'_n, y'_n) = \dfrac{2n^2}{1 + 4n^2} \to \dfrac{1}{2} \neq 1 \\
\implies f(x,y) \text{ няма граница при } (x,y) \to (0,0)
\end{gather*}

\item 
\begin{gather*}
f(x,y) = \dfrac{xy^2}{x^2+y^4}\\
\lim\limits_{x \to 0} f(x,y) = \dfrac{0}{y^4} = 0 \qquad 
\lim\limits_{y \to 0} f(x,y) = \dfrac{0}{x^2} = 0 \\
A_{1,2} = A_{2,1} = 0 \implies \exists A = \lim\limits_{(x,y) \to (0,0)} f(x,y) \\
\text{Редица: }(x_n, y_n) = \left ( \dfrac{1}{n^2},\dfrac{1}{n}\right ) \to (0,0), f(x_n, y_n) = \dfrac{1}{2} \to \dfrac{1}{2} \neq 0\\
\implies f(x,y) \text{ няма граница при } (x,y) \to (0,0)
\end{gather*}

\item 
\begin{gather*}
f(x,y) = (x+y) \sin{\dfrac{1}{x}} \cos{\dfrac{1}{y}}\\
0 \leq \vert f(x,y) \vert \leq \vert x + y \vert \leq  \vert x \vert + \vert y \vert \text{ и } \vert x \vert + \vert y \vert \to 0 \\
A = 0 \\
\lim\limits_{x \to 0} \sin{\dfrac{1}{x}} \text{ - не съществува} \\
\lim\limits_{x \to 0} f(x,y) = y\cos{\dfrac{1}{y}} \lim\limits_{x \to 0} \sin{\dfrac{1}{x}}
\end{gather*}
Аналогично и другата вътрешна граница не съществува. Но тогава и повторните граници $A_{1,2},\, A_{2,1}$ не съществуват.

\item
\begin{gather*}
f(x,y) = \dfrac{x^4 + y^4}{x^2 + y^2}\\
\lim\limits_{x \to 0} f(x,y) = y^2 \qquad 
\lim\limits_{y \to 0} f(x,y) = x^2 \\
A_{1,2} = \lim\limits_{y \to 0} \left( \lim\limits_{x \to 0} f(x,y) \right) = \lim\limits_{y \to 0} \left( y^2 \right) = 0\\
A_{2,1} = \lim\limits_{x \to 0} \left( \lim\limits_{y \to 0} f(x,y) \right) = \lim\limits_{x \to 0} \left( x^2 \right) = 0 \\
\implies A = A_{1,2} = A_{2,1} = 0
\end{gather*}
\end{enumerate}

\subsection*{Задача 3}
Функцията f е непрекъсната в А, защото е частно на две функции със знаменател $y^2\neq 0$, в А.\\
Аналогично е непрекъсната в B защото знаменателя е $x^2 \neq 0$.\\
Остана да се изследва поведението върху C. \\
\begin{gather*}
(x_0,y_0) = (x_0, x_0) \in C \\
R = \{ (x_n , y_n) \},\ (x_n , y_n) \in A \\
\lim\limits_{n \to \infty} R = (x_0, y_0)\\
\lim\limits_{n \to \infty} f(x_n,y_n) = \dfrac{1}{y_0 ^2} = \dfrac{1}{x_0 ^2} \neq 0\\
\text{Ако } x_0 \neq 0, \ f(x_0,y_0) = 0 \\
\implies \text{ функцията е прекъсната в точката } (x_0, x_0) \neq (0,0) \\
\text{Ако } (x_n , y_n) \in B,\ \lim\limits_{n \to \infty} f(x_n,y_n) = - \dfrac{1}{x_0 ^2} \neq f(x_0, x_0) \neq 0. \\
\text{Ако } x_0 = 0,\ \lim\limits_{n \to \infty} f(x_n,y_n) = \infty(-\infty),\ f(0,0) = 0,\\
\implies \text{f е прекъсната в точката (0,0).} 
\end{gather*}
Функцията е непрексъната в D, с изключение на точките от C, където е прекъсната. 

\newpage
\section{Упражнение към лекция 3}
\subsection{Задачи}

\subsection*{Задача 1}
Да се намерят първите частни производни на следните функции
\begin{enumerate}

\item $f(x,y,z) = e^{4x+3y} + xy^2z^3 + 1111e^\pi$ за произволна точка $(x_0, y_0, z_0) \in \mathbb{R}^3$
\item $f(x,y) = \vert x + y \vert$ в точката (0,0)
\item $
f(x,y) = 
\begin{cases}
\dfrac{xy}{x^2 + y^2}, & (x,y) \neq (0,0)  \\
0, & (x,y) = (0,0) 
\end{cases}$ в равнината $\mathbb{R}^2$ 
\end{enumerate}

\subsection*{Задача 2}
$$f(x,y) = x + (y-1)\arcsin{\sqrt{\dfrac{x}{y}}}\qquad f'_x(x,1) = ?$$

\subsection*{Задача 3}
Да се докаже че функцията $f(x,y) = 
\begin{cases}
\dfrac{x^3y}{x^6 + y^2}, & (x,y) \neq (0,0)  \\
0, & x^2 + y^2 = (0,0) 
\end{cases}
$ \\
е прекъсната в точката (0,0) но има частни производни в тази точка. 

\subsection*{Задача 4}
Да се намерят първите частни производни на следните функции:
\begin{enumerate}
\item $f(x,y) = \sin{(2x+3)} + 3e^{-x}e^{4y} - 11x^3 + 19e^\pi$
\item $f(x,y) = \sqrt{x^2 + y^2} + \arctan {\frac{y}{x}}$
\item $f(x,y,z) = (xy)^z$
\item $\sqrt[3]{x^2+3y^2} e^{x^2 - 5y}$
\end{enumerate}

\newpage
\subsection{Решения}

\subsection*{Задача 1}

\begin{enumerate}
\item 
\begin{gather*}
f(x,y,z) = e^{4x+3y} + xy^2z^3 + 1111e^\pi \\
f(x,y_0,z_0) \implies f'_x (x_0,y_0,z_0) = 4e^{4x_0+3y_0} + y_0 ^2 z_0 ^3 \\
f(x_0,y,z_0) \implies f'_y (x_0,y_0,z_0) = 3e^{4x_0+3y_0} + 2x_0 y_0 z_0 ^3 \\
f(x_0,y_0,z) \implies f'_z (x_0,y_0,z_0) = 3x_0 y_0 ^2 z_0 ^2
\end{gather*}

\item 
\begin{gather*}
f(x,y) = \vert x + y \vert \\
\dfrac{g(h) - g(0)}{h} = \dfrac{f(0+h,0) - f(0,0)}{h} \\
\lim\limits_{h \to 0} \dfrac{f(0+h,0) - f(0,0)}{h} = \lim\limits_{h \to 0} \dfrac{\vert h \vert}{h} \text{ не съществува} \\
\implies \nexists f'_x (0,0) \text{(Аналогично се получава за $f'_y(0,0)$)} 
\end{gather*}

\item 
\begin{gather*}
f(x,y) = 
\begin{cases}
\dfrac{xy}{x^2 + y^2}, & (x,y) \neq (0,0)  \\
0, & (x,y) = (0,0) 
\end{cases}\\
(x,y) \neq (0,0)\\
f'_x (x,y) = \dfrac{y(y^2 - x^2)}{(x^2 + y^2)^2}\\
f'_y (x,y) = \dfrac{x(x^2 - y^2)}{(x^2 + y^2)^2}\\
\lim\limits_{h \to 0} \dfrac{f(0+h,0) - f(0,0)}{h} = \lim\limits_{h \to 0} \dfrac{0-0}{h} = \lim\limits_{h \to 0} = 0 \\
\lim\limits_{k \to 0} \dfrac{f(0,0+k) - f(0,0)}{k} =  \lim\limits_{k \to 0} \dfrac{0-0}{k} = \lim\limits_{k \to 0} = 0 \\
\implies  \text{Функцията има частни производни във всичко точки на равнината $\mathbb{R}^2$}
\end{gather*}
\end{enumerate}

\subsection*{Задача 2}
\begin{gather*}
f'_x (a,b) = \lim\limits_{h \to 0} \dfrac{f(a +h,b) - f(a,b)}{h} \text{(Ако съществува)} \implies \\
f'_x(x,1) =  \lim\limits_{h \to 0} \dfrac{f(x +h,1) - f(x,1)}{h} \text{(Ако съществува)}\\
f(x +h,1) = x + h + (1-1)\arcsin{\sqrt{\dfrac{x}{1}}} = x + h + 0 \arcsin{\sqrt{\dfrac{x}{1}}} =  x + h \\
f(x,1) = x + (1-1)\arcsin{\sqrt{\dfrac{x}{1}}} = x + 0 \arcsin{\sqrt{\dfrac{x}{1}}} =  x \implies \\
\lim\limits_{h \to 0} \dfrac{f(x +h,1) - f(x,1)}{h} =  \lim\limits_{h \to 0} \dfrac{x + h - x}{h} = \lim\limits_{h \to 0} \dfrac{h}{h} =  \lim\limits_{h \to 0} 1 \implies f'_x(x,1) = 1
\end{gather*}

\subsection*{Задача 3}

\begin{gather*}
\text{Редица } (x_n, y_n) = \left(\dfrac{1}{n}, \dfrac{1}{n^3} \right) \\
f(x_n, y_n) = \dfrac{ \left( \dfrac{1}{n} \right)^3 \cdot \dfrac{1}{n^3}}{\left( \dfrac{1}{n}\right)^6 + \left( \dfrac{1}{n^3} \right)^3 } = \dfrac{\dfrac{1}{n^6}}{\dfrac{2}{n^6}} = \dfrac{1}{2} \qquad \lim\limits_{n \to \infty} f(x_n, y_n) = \dfrac{1}{2}  \implies \\
\lim\limits_{x \to 0, y \to 0 } f(x, y) \neq f(0,0) = 0 \implies f(x, y) \text{ е прекъсната в т. }(0,0).\\
\\
f'_x (0,0) = \lim\limits_{x \to 0} \dfrac{f(x,0) - f(0,0)}{x - 0} = \dfrac{ \dfrac{x^3 \cdot 0}{x^6 + 0} - 0}{x - 0} = 0 \\
f'_y (0,0) = \lim\limits_{y \to 0} \dfrac{f(0,y) - f(0,0)}{y - 0} = \dfrac{\dfrac{0^3 \cdot y}{0^6 + y^2} - 0}{y - 0} = 0
\end{gather*}

\subsection*{Задача 4}
\begin{enumerate}
\item 
\begin{gather*}
f(x,y) = \sin{(2x+3)} + 3e^{-x}e^{4y} - 11x^3 + 19e^\pi \\
f'_x(x,y) = (\sin{(2x+3)})'_x + (3e^{-x}e^{4y})'_x - (11x^3)'_x + (19e^\pi )'_x \\
f'_x(x,y) = \cos{(2x+3)}\cdot 2 + (-3e^{-x}e^{4y}) - (3 \cdot 11x^2) + 0 \\
f'_x(x,y) = 2\cos{(2x+3)} -3e^{-x}e^{4y} - 33x^2 \\
f'_y(x,y) = (\sin{(2x+3)})'_y + (3e^{-x}e^{4y})'_y - (11x^3)'_y + (19e^\pi )'_y \\
f'_y(x,y) = 0 + (3 \cdot 4 e^{-x}e^{4y}) - 0 + 0 = 12e^{-x}e^{4y}
\end{gather*}
\item 
\begin{gather*}
f(x,y) = \sqrt{x^2 + y^2} + \arctan {\frac{y}{x}}\\
f'_x(x,y) = \dfrac{1}{2} (x^2)^{- \dfrac{1}{2}} \cdot 2x + \dfrac{1}{1 + \dfrac{y^2}{x^2}} \cdot y \cdot ( - \dfrac{1}{x^2}) \\
f'_x(x,y) = \dfrac{x}{\sqrt{x^2 + y^2}} - \dfrac{x^2y}{x^2 + y^2} \cdot \dfrac{1}{x^2} \\
f'_x(x,y) = \dfrac{x}{\sqrt{x^2 + y^2}} - \dfrac{xy}{x^2 + y^2} \\
f'_y(x,y) = \dfrac{1}{2} (x^2)^{- \dfrac{1}{2}} \cdot 2y + \dfrac{1}{1 + \dfrac{y^2}{x^2}} \cdot  \dfrac{1}{x} \\
f'_y(x,y) = \dfrac{y}{\sqrt{x^2 + y^2}} + \dfrac{x^2}{x^2 + y^2} \cdot \dfrac{1}{x}\\
f'_y(x,y) = \dfrac{y}{\sqrt{x^2 + y^2}} + \dfrac{x}{x^2 + y^2}
\end{gather*}
\item 
\begin{gather*}
f(x,y,z) = (xy)^z\\
f'_x(x,y,z) = z(xy)^{z-1} \cdot (xy)'x = yz(xy)^{z-1}\\
f'_y(x,y,z) = z(xy)^{z-1} \cdot (xy)'y = xz(xy)^{z-1}\\
f'_z(x,y,z) = (xy)^z \ln{(xy)}
\end{gather*}
\item 
\begin{gather*}
\sqrt[3]{x^2+3y^2} e^{x^2 - 5y}\\
f'_x(x,y) = \left[ \sqrt[3]{x^2+3y^2} \right]'_x \cdot e^{x^2 - 5y} + \sqrt[3]{x^2+3y^2} \cdot (e^{x^2 - 5y})'_x\\
f'_x(x,y) = \dfrac{1}{3} (x^2 + 3y^2)^{- \frac{2}{3}} \cdot 2x  \cdot e^{x^2 - 5y} +  \sqrt[3]{x^2+3y^2}\cdot 2x e^{x^2 - 5y} \\
f'_x(x,y) = \dfrac{2x}{3} \cdot \dfrac{e^{x^2 - 5y}}{\sqrt[3]{(x^2+3y^2)^2}} + 2x \sqrt[3]{x^2+3y^2} \cdot e^{x^2 - 5y} \\
f'_x(x,y) = \dfrac{2x}{3} \cdot \dfrac{e^{x^2 - 5y}}{\sqrt[3]{(x^2+3y^2)^2}} \left[ 1 + 3(x^2 + 3y^2) \right] \\
f'_x(x,y) = \dfrac{2x}{3} (1 + 3x^2 + 9y^2) \dfrac {e^{x^2 - 5y}}{\sqrt[3]{(x^2+3y^2)^2}}\\
\\
f'_y(x,y) =  \left[ \sqrt[3]{x^2+3y^2} \right]'_y \cdot e^{x^2 - 5y} + \sqrt[3]{x^2+3y^2} \cdot (e^{x^2 - 5y})'_y\\
f'_y(x,y) = \dfrac{1}{3} (x^2 + 3y^2)^{- \frac{2}{3}} \cdot 6y  \cdot e^{x^2 - 5y} +  \sqrt[3]{x^2+3y^2}\cdot (-5e^{x^2 - 5y}) \\
f'_y(x,y) = 2y \cdot \dfrac{1}{\sqrt[3]{(x^2+3y^2)^2}} \cdot e^{x^2 - 5y} - 5  \sqrt[3]{x^2+3y^2}\cdot e^{x^2 - 5y} \\
f'_y(x,y) = e^{x^2 - 5y} \cdot \sqrt[3]{(x^2+3y^2)^2} (2y - 5(x^2 + 3y^2)) \\
f'_y(x,y) = (2y - 5x^2 - 15y^2) \dfrac{e^{x^2 - 5y}}{\sqrt[3]{(x^2+3y^2)^2}}
\end{gather*}
\end{enumerate}

\newpage
\section{Упражнение към лекция 4}

\subsection{Задачи}

\subsection*{Задача 1}
$f(x,y) = \sqrt[3]{xy}$\\
Изследвайте $f(x,y)$ за диференцируемост в $(0,0)$.\\
$f'_x (0,0) = ?\\
f'_y (0,0) = ?$ 

\subsection*{Задача 2}
$f(x,y) = \sqrt[3]{x^3 + y^3}$\\
Изследвайте $f(x,y)$ за диференцируемост в $(0,0)$.

\subsection*{Задача 3}
Да се изследвай за диференцируемост в $(0,0)$ функцията 
$$f(x,y) = 
\begin{cases}
e^{- \dfrac{1}{x^2 + y^2}}, & x^2 + y^2 \neq 0 \\
0, & x^2 + y^2 = 0
\end{cases}
$$

\subsection*{Задача 4}
$f(x,y) = x^2 + 3xy - 8y^3 + 11, \quad df(0,1) =?$\\
$f(x,y,z) = x^2 + 3xy - 8y^3 - 2e^{3z}x, \quad df(0,0,4) = ? $

\subsection*{Задача 5}
$f(x,y) = x^6 - 7xy^2 + 14y,\\
f''_{xx} = ?, f''_{yy} = ?, f''_{xy} = ?, d^2f(x,y) = ?$\\
$f(x,y,z) = x^6 - 7xy +y^2 - xz + z^3 ,  \\
f''_{xx} = ?, f''_{xy} = ?, f''_{xz} = ?  f''_{yx} = ?, f''_{yy} = ?, f''_{yz} = ?  f''_{zx} = ?, f''_{zy} = ?, f''_{zz} = ?, d^2f(1,0,0)$

\newpage
\subsection{Решения} 

\subsection*{Задача 1}

\begin{gather*}
f(x,0) - f(0,0) = \sqrt[3]{x0} - \sqrt[3]{0}\implies \\
\lim\limits_{x \to 0} \dfrac{f(x,0) - f(0,0)}{x - 0} = \lim\limits_{x \to 0} \dfrac{0}{x} = 0
f'_x(0,0) = \lim\limits_{x \to 0} \dfrac{f(x,0) - f(0,0)}{x - 0} =  \lim\limits_{x \to 0} \dfrac{0}{x} = 0\\
f(0,y) - f(0,0) = \sqrt[3]{0y} - \sqrt[3]{0} \implies \\
f'_y(0,0) = \lim\limits_{y \to 0} \dfrac{f(0,y) - f(0,0)}{y - 0} =  \lim\limits_{y \to 0} \dfrac{0}{y} = 0\\
\text{Нека:}
\lim\limits_ {(x \to 0, y \to 0)} \varepsilon (x,y) \to 0, \rho (x,y) = \sqrt{x^2 + y^2}\\
\text{Проверка за диференцируемост в (0,0):}\\
 f(x,y) - f(0,0)  = f'_x(0,0)(x - 0) + f'_y(0,0)(y-0) + \varepsilon (x,y) \rho (x,y) \\
\sqrt[3]{xy} - 0 = 0x + 0y + \varepsilon (x,y)\sqrt{x^2 + y^2} \implies \\
\varepsilon (x,y) = \dfrac{\sqrt[3]{xy}}{\sqrt{x^2 + y^2}} \to 0? \\
\text{Разглеждаме редица с общ член } (x_n, y_n) = \left( \dfrac{1}{n^3}, \dfrac{1}{n^3} \right) \text{ за която } (x_n, y_n) \to (0,0), \\
\varepsilon (x_n, y_n) = \dfrac{\dfrac{1}{n^2}}{\dfrac{\sqrt{2}}{n^3}} = \dfrac{n}{\sqrt{2}} \implies
\lim\limits_ {(x,y) \to (0,0)}\varepsilon(x_n, y_n) \not\to 0 \implies \\
 f(x,y) \text{ не е диференцируема в т.} (0,0)
\end{gather*}

\subsection*{Задача 2}
\begin{gather*}
f(x,0) - f(0,0) = \sqrt[3]{x^3} - 0 = x\implies \\
\lim\limits_{x \to 0} \dfrac{f(x,0) - f(0,0)}{x - 0}= \lim\limits_{x \to 0} \dfrac{x}{x} = 1 \implies \exists f'_x(0,0) = 1\\
f(0,y) - f(0,0) = \sqrt[3]{y^3} - 0 = y \implies \\
\lim\limits_{y \to 0}\dfrac{f(0,y) - f(0,0)}{y - 0} = \lim\limits_{y \to 0} \dfrac{y}{y} = 1 \implies \exists f'_y(0,0) = 1\\
\text{Нека:}
\lim\limits_ {(x \to 0, y \to 0)} \varepsilon (x,y) \to 0, \rho (x,y) = \sqrt{x^2 + y^2}\\
\text{Проверка за диференцируемост в (0,0):}\\
f(x,y) - f(0,0)  = f'_x(0,0)(x - 0) + f'_y(0,0)(y-0) + \varepsilon (x,y) \rho (x,y) \\
\sqrt[3]{x^3 + y^3} = x + y + \varepsilon (x,y) \sqrt{x^2 + y^2} \\
\varepsilon (x,y) = \dfrac{\sqrt[3]{x^3 + y^3} - x - y}{ \sqrt{x^2 + y^2}} \\
\lim\limits_ {(x \to 0, y \to 0)} \varepsilon (x,y) \to 0?\\
\text{Разглеждаме редица с общ член } (x_n, y_n) = \left( \dfrac{1}{n}, \dfrac{1}{n} \right) \text{ за която } (x_n, y_n) \to (0,0), \\
\varepsilon (x_n, y_n) = \dfrac{\dfrac{\sqrt[3]{2}}{n} - \dfrac{2}{n}}{\dfrac{\sqrt{2}}{n}} = \dfrac{\sqrt[3]{2} - 2}{\sqrt{2}} \implies \lim\limits_ {(x \to 0, y \to 0)} \varepsilon (x,y) \not\to 0 \implies \\
 f(x,y) \text{ не е диференцируема в т.} (0,0)
\end{gather*}

\subsection*{Задача 3}
\begin{gather*}
f(x,0) - f(0,0) = e^{- \dfrac{1}{x^2}} - 0 = e^{- \dfrac{1}{x^2}}\\
\lim\limits_{x \to 0} \dfrac{f(x,0) - f(0,0)}{x - 0}= \lim\limits_{x \to 0} \dfrac{e^{- \dfrac{1}{x^2}}}{x} = \left[ \dfrac{0}{0}\right]\\
\lim\limits_{x \to 0} \dfrac{e^{- \dfrac{1}{x^2}}}{x} = \lim\limits_{x \to 0} \dfrac{\dfrac{1}{x}}{e^{\dfrac{1}{x^2}}} = \left[ \dfrac{\infty}{\infty}\right] \\
\left( \dfrac{1}{x} \right)' = -\dfrac{1}{x^2} \quad \left( e^{\dfrac{1}{x^2}} \right)' = -\dfrac{2}{x^3} e^{\dfrac{1}{x^2}}\\
\lim\limits_{x \to 0} \dfrac{-\dfrac{1}{x^2}}{-\dfrac{2}{x^3} e^{\dfrac{1}{x^2}}} = \lim\limits_{x \to 0} \dfrac{x}{2e^{\dfrac{1}{x^2}}} = \dfrac{0}{\infty} = 0 \implies f'_x(0,0) = 0\\
\text{Аналогично} f'_y(0,0) = 0\\
\text{Нека}: \lim\limits_ {(x \to 0, y \to 0)} \varepsilon (x,y) \to 0, \rho (x,y) = \sqrt{x^2 + y^2}\\
\text{Проверка за диференцируемост в (0,0):}\\
f(x,y) - f(0,0)  = f'_x(0,0)(x - 0) + f'_y(0,0)(y-0) + \varepsilon (x,y) \rho (x,y) \\
e^{- \dfrac{1}{x^2 + y^2}} - 0 = 0(x-0) + 0(y-0) + \varepsilon (x,y)\sqrt{x^2 + y^2}\\
e^{- \dfrac{1}{x^2 + y^2}} = \varepsilon (x,y)\sqrt{x^2 + y^2}\\
\varepsilon (x,y) = \dfrac{e^{- \dfrac{1}{x^2 + y^2}}}{\sqrt{x^2 + y^2}}\\
\lim\limits_ {(x \to 0, y \to 0)} \varepsilon (x,y) \to 0?\\
\end{gather*}
\begin{gather*}
\rho (x,y) = \sqrt{x^2 + y^2} \implies \lim\limits_ {(x \to 0, y \to 0)}\rho (x,y) \to 0 \\
\lim\limits_ {(x \to 0, y \to 0)} \varepsilon (x,y) = \lim\limits_ {\rho \to 0} \dfrac{e^{- \dfrac{1}{\rho^2}}}{\rho} = \left[ \dfrac{\infty}{\infty}\right]\\
\left( \dfrac{1}{\rho} \right)' = -\dfrac{1}{\rho^2} \quad \left( e^{\dfrac{1}{\rho^2}} \right)' = -\dfrac{2}{\rho^3} e^{\dfrac{1}{\rho^2}} \\
\lim\limits_ {\rho \to 0} \dfrac{\rho}{2e^{\dfrac{1}{\rho^2}}} = \dfrac{0}{\infty} = 0 \implies \\
\lim\limits_ {(x \to 0, y \to 0)} \varepsilon (x,y) = \lim\limits_ {\rho \to 0} \dfrac{\dfrac{1}{\rho}}{e^{\dfrac{1}{\rho^2}}} = \lim\limits_ {\rho \to 0} \dfrac{\left( \dfrac{1}{\rho} \right)'}{\left( e^{\dfrac{1}{\rho^2}}\right)'} = 0
\implies \\
\lim\limits_ {(x \to 0, y \to 0)} \varepsilon (x,y) = 0 \implies f(x,y) \text{ е диференцируема в } (0,0)
\end{gather*}

\subsection*{Задача 4}
\begin{gather*}
df(x,y) = f'_x(x,y)dx + f'_y(x,y)dy\\
f'_x(x,y) = 2x + 3y \qquad f'_x(0,1) = 3\\
f'_y(x,y) = 3x - 24y^2 \qquad f'_y(0,1) = -24\\
df(x,y) = (2x + 3y)dx + (3x - 24y^2)dy\\
df(0,1) = 3dx - 24dy\\
\end{gather*}
\begin{gather*}
df(x,y,z) = f'_x(x,y,z)dx + f'_y(x,y,z)dy +  f'_z(x,y,z)dz\\
f'_x(x,y,z) = 2x + 3y - 2e^{3z} \qquad f'_x(0,0,4) = - 2e^{12}\\
f'_y(x,y,z) = 3x - 24y^2 \qquad f'_y(0,0,4) = 0\\
f'_z(x,y,z) = 6xe^{3z} \qquad f'_z(0,0,4) = 0\\
df(x,y,z) =(2x + 3y - 2e^{3z})dx + (3x - 24y^2)dy +  (6xe^{3z})dz\\
df(x,y,z) = - 2e^{12}dx + 0dy + 0dz = - 2e^{12}dx
\end{gather*}

\subsection*{Задача 5}

\begin{gather*}
f'_x(x,y,z) = 6x^5 - 7y - z\\
f''_{xx}(x,y,z) = (6x^5 - 7y - z)'_x = 30x^4 \qquad f''_{xx}(1,0,0) = 30 \\
f''_{xy}(x,y,z) = (6x^5 - 7y - z)'_y = -7 \qquad f''_{xy}(1,0,0) = -7 \\
f''_{xz}(x,y,z) = (6x^5 - 7y - z)'_z = -1 \qquad f''_{xz}(1,0,0) = -1 
\end{gather*}
\begin{gather*}
f'_y(x,y,z) = -7x + 2y\\
f''_{yx}(x,y,z) = (-7x + 2y)'_x = -7 \qquad f''_{yx}(1,0,0) = -7 \\
f''_{yy}(x,y,z) = (-7x + 2y)'_y = 2 \qquad f''_{yy}(1,0,0) = 2 \\
f''_{yz}(x,y,z) = (-7x + 2y)'_z = 0\qquad f''_{yz}(1,0,0) = 0 
\end{gather*}
\begin{gather*}
f'_z(x,y,z) = -x + 3z^2\\
f''_{zx}(x,y,z) = (-x + 3z^2)'_x = -1 \qquad f''_{zx}(1,0,0) = -1 \\
f''_{zy}(x,y,z) = (-x + 3z^2)'_y = 0 \qquad f''_{zy}(1,0,0) = 0 \\
 f''_{zz}(x,y,z) = (-x + 3z^2)'_z = 6z  \qquad f''_{zz}(1,0,0) = 0 
\end{gather*}
\begin{gather*}
d^2f = f''_{xx}dx^2 + 2f''_{xy}dxdy + f''_{yy}dy^2 + 2f''_{xz}dxdz +  f''_{zz}dz^2 + f''_{yz}dydz\\
d^2f(x,y,z) = 30x^4dx^2 +2\cdot(-7) dxdy + 2dy^2 + 2\cdot(-1)dxdz + 6zdz^2 + 2\cdot 0 dydz \\
d^2f(1,0,0) = 30dx^2 -14dxdy + 2dy^2 - 2dxdz + 0dz^2 + 0 dydz \\
d^2f(1,0,0) = 30dx^2 + 2dy^2 -14dxdy - 2dxdz 
\end{gather*}

\newpage
\section{Упражнение към лекция 5}

\subsection{Задачи}

\subsection*{Задача 1}
Да се намерят посочените частни производни на следните функции.\\
\begin{enumerate}
\item $u(x,y) = x^4 + 11x^2y^3, \qquad u''_{xx} = ?,\, u''_{xy} = ?$ 
\item $u(x,y) = \arctan{\dfrac{x+y}{1-xy}}, \qquad u''_{xx} = ?,\, u''_{xy} = ?,\,u''_{yy} = ?$
\item $u(x,y) = \dfrac{1}{2} \ln{(x^2 + y^2)}, \qquad u''_{xx} = ?,\, u''_{xy} = ?,\,u''_{yx} = ?,\,u''_{yy} = ?$
\item $u(x,y) = \ln{(x + 2y)}, \qquad u'''_{xxy} = ?$
\item $u(x,y,z) = e^{xy^2z^3}, \qquad u'''_{xyz} = ?$
\end{enumerate}

\subsection*{Задача 2}
Дали са верни равенствата: \\
\begin{itemize}
\item Ако $z = y\ln{(x^2+y^2)}$ то $\quad \dfrac{1}{x} z'_x + \dfrac{1}{y} z'_y = \dfrac{z}{y^2}$
\item Ако $u = \ln{(x^3+y^3 + z^3 - 3xyz)} $ то $ \quad u'_x + u'_y + u'_z = \dfrac{3}{x+y+z}$
\end{itemize}

\subsection*{Задача 3}
Да се докаже, че функцията: $z(x,y) = \arctan{\left( \dfrac{x+y}{x-y} \right)}$ удовлетворява тъждеството: $z'_x + z'_y = \dfrac{x-y}{x^2 + y^2}$

\subsection*{Задача 4}
Да се провери тъждеството на Ойлер за следните функции:
$z(x,y) = \dfrac{1}{(x^2+y^2)^2}\qquad u(x,y,z) = \sqrt{x^2+y^2+z^2}\cdot \ln{\left( \dfrac{y}{x}\right)}$\\
Тъждество на Ойлер($f:D \to R, D\subset \mathbb{R}^m$)
$$x_1f'_{x_1} + x_2f'_{x_2} + ... +x_mf'_{x_m} = mf$$

\newpage
\subsection{Решения}

\subsection*{Задача 1}
\begin{gather*}
u(x,y) = x^4 + 11x^2y^3 \\
u'_x = 4x^3 + 22xy^3 \\
u''_{xx} = 12x^2 + 22y^3 \\
u''_{xy} = 4x^3 + 66xy^2 
\end{gather*}
\begin{gather*}
u(x,y) = \arctan{\dfrac{x+y}{1-xy}}\\
u'_x = \dfrac{1}{1 + \left( \dfrac{x+y}{1-xy} \right)^2} \cdot  \left( \dfrac{x+y}{1-xy} \right)'_x\\
u'_y = \dfrac{1}{1 + \left( \dfrac{x+y}{1-xy} \right)^2} \cdot  \left( \dfrac{x+y}{1-xy} \right)'_y\\
u''_{xx} = (u'_x)'_x\\
u''_{xy} = (u'_x)'_y\\
u''_{yy} = (u'_y)'_y
\end{gather*}
\begin{gather*}
A = \dfrac{1}{1 + \left( \dfrac{x+y}{1-xy} \right)^2}. \quad B = \left( \dfrac{x+y}{1-xy} \right)'_x \implies u'_x = AB\\
A = \dfrac{1}{1 + \left( \dfrac{x+y}{1-xy} \right)^2} =\dfrac{1}{1 + \dfrac{(x+y)^2}{(1-xy)^2}} = \dfrac{(1-xy)^2}{(1-xy)^2 + (x+y)^2}\\ 
A = \dfrac{(1-xy)^2}{1 - 2xy + x^2y^2 + x^2 + 2xy + y^2} = \dfrac{(1-xy)^2}{1 + x^2y^2 + x^2 + y^2}\\
A = \dfrac{(1-xy)^2}{(1 + y^2) + x^2 + x^2y^2} = \dfrac{(1-xy)^2}{(1 + y^2) + x^2(1 +y^2)} = \dfrac{(1-xy)^2}{(1 + y^2) (1 + x^2)}\\
B = \left( \dfrac{x+y}{1-xy} \right)'_x = \dfrac{1(1-xy) - (x+y)(-y)}{(1 - xy)^2} = \dfrac {1-xy + xy + y^2}{(1 - xy)^2} =\dfrac {1+y^2}{(1 - xy)^2}  \\
u'_x = AB =  \dfrac{(1-xy)^2}{(1 + y^2) (1 + x^2)} \cdot \dfrac {1-y^2}{(1 - xy)^2} = \dfrac{1}{1 + x^2}\\
C = \left( \dfrac{x+y}{1-xy} \right)'_y \implies u'_y = AC\\
C = \dfrac{1(1 -xy) - (x+y)(-x)}{(1 - xy)^2} = \dfrac{1 -xy + x^2 +xy}{(1 - xy)^2} = \dfrac{1 + x^2}{(1 - xy)^2}\\
u'_y = AC = \dfrac{(1-xy)^2}{(1 + y^2) (1 + x^2)} \cdot \dfrac{1 + x^2}{(1 - xy)^2} = \dfrac{1}{1 + y^2}
\end{gather*}
\begin{gather*}
u''_{xx} =\left( \dfrac{1}{1 + x^2} \right)'_x = ((1 + x^2)^{-1})'_x\\
u''_{yy}= -(1 + x^2)^{-2}(1+x^2)'_x = -2x(1 + x^2)^{-2}  = \dfrac{-2x}{(1 + x^2)^2}\\
u''_{xy} =\left( \dfrac{1}{1 + x^2} \right)'_y = 0 \\
u''_{yy} =\left( \dfrac{1}{1 + y^2} \right)'_y = ((1 + y^2)^{-1})'_y\\
u''_{yy}= -(1 + y^2)^{-2}(1+y^2)'_y = -2y(1 + y^2)^{-2}  = \dfrac{-2y}{(1 + y^2)^2}
\end{gather*}
\begin{gather*}
u(x,y) = \frac{1}{2} \ln{(x^2 + y^2)}\\
u'_x = \dfrac{1}{2(x^2 + y^2)} \cdot (x^2 + y^2)'_x = \dfrac{2x}{2(x^2 + y^2)} = \dfrac{x}{x^2 + y^2}\\
u'_y = \dfrac{1}{2(x^2 + y^2)} \cdot (x^2 + y^2)'_y = \dfrac{2y}{2(x^2 + y^2)} = \dfrac{y}{x^2 + y^2}\\
u''_{xx} = (u'_x)'_x = \left( \dfrac{x}{x^2 + y^2} \right)'_x =  \dfrac{1(x^2 + y^2) - (2x)x}{(x^2 + y^2)^2} = \dfrac{x^2 + y^2 - 2x^2}{(x^2 + y^2)^2} = \dfrac{ y^2 - x^2}{(x^2 + y^2)^2}\\
u''_{xy} = (u'_x)'_y = \left( \dfrac{x}{x^2 + y^2} \right)'_y =  \dfrac{- 2xy}{(x^2 + y^2)^2}\\
u''_{yy} = (u'_y)'_y = \left( \dfrac{y}{x^2 + y^2} \right)'_y = \dfrac{1(x^2 + y^2) - (2y)y}{(x^2 + y^2)^2} = \dfrac{x^2 + y^2 - 2y^2}{(x^2 + y^2)^2} = \dfrac{x^2 - y^2}{(x^2 + y^2)^2}\\
u''_{yx} = (u'_y)'_x = \left( \dfrac{y}{x^2 + y^2} \right)'_x = \dfrac{- 2xy}{(x^2 + y^2)^2}
\end{gather*}
\begin{gather*}
u(x,y) = \ln{(x + 2y)}\\
u'_x = \dfrac{1}{x + 2y}\\
u''_{xx} = \left( \dfrac{1}{x + 2y} \right)'_x = ((x+2y)^{-1})'_x = -(x+2y)^{-2}(x+2y)'_x = - \dfrac{1}{(x+2y)^2} \\
u'''_{xxy} = \left( -\dfrac{1}{(x + 2y)^2} \right)'_y = -((x+2y)^{-2})'_y = 2((x+2y)^{-3})(x+2y)'_y = \dfrac{4}{(x+2y)^3}
\end{gather*}
\begin{gather*}
u(x,y,z) = e^{xy^2z^3}\\
u'_x =  e^{xy^2z^3}(xy^2z^3)'_x = y^2z^3e^{xy^2z^3}\\
u''_{xy} = (y^2z^3 \cdot e^{xy^2z^3})'_y = (y^2z^3)'_y \cdot e^{xy^2z^3} + y^2z^3 (e^{xy^2z^3})'_y\\
u''_{xy} = 2yz^3e^{xy^2z^3} + 2xy^3z^6e^{xy^2z^3} =  2yz^3e^{xy^2z^3} (1 + xy^2z^3)
\end{gather*}
\begin{gather*}
u'''_{xyz} = \left[ 2yz^3e^{xy^2z^3} (1 + xy^2z^3) \right]'_z = (2yz^3e^{xy^2z^3})'_z (1 + xy^2z^3) + 2yz^3e^{xy^2z^3}(1 + xy^2z^3)'_z\\
=  \left[(2yz^3)'_z\cdot e^{xy^2z^3}+ 2yz^3 \cdot (e^{xy^2z^3})'_z\right] (1 + xy^2z^3) + 2yz^3e^{xy^2z^3}(1 + xy^2z^3)'_z \\
u'''_{xyz} = \left[6yz^2e^{xy^2z^3} +  2yz^3e^{xy^2z^3}3xy^2z^2 \right] (1 + xy^2z^3) + 2yz^3e^{xy^2z^3} (3xy^2z^2) \\
u'''_{xyz} =  \left[6yz^2e^{xy^2z^3} +  6xy^3z^5e^{xy^2z^3} \right] (1 + xy^2z^3) + 6xy^3z^5e^{xy^2z^3} \\
u'''_{xyz} = \left[6yz^2e^{xy^2z^3} + 6yz^2e^{xy^2z^3} xy^2z^3 + 6xy^3z^5e^{xy^2z^3} + 6xy^3z^5e^{xy^2z^3} xy^2z^3\right] + 6xy^3z^5e^{xy^2z^3}\\
u'''_{xyz} = \left[6yz^2e^{xy^2z^3} + 6xy^3z^5e^{xy^2z^3} + 6xy^3z^5e^{xy^2z^3} + 6x^2y^5z^8e^{xy^2z^3}\right] + 6xy^3z^5e^{xy^2z^3}\\
u'''_{xyz} = 6yz^2e^{xy^2z^3} + 6xy^3z^5e^{xy^2z^3} + 6xy^3z^5e^{xy^2z^3} + 6x^2y^5z^8e^{xy^2z^3}+ 6xy^3z^5e^{xy^2z^3}\\
u'''_{xyz} = 6yz^2e^{xy^2z^3} + 18xy^3z^5e^{xy^2z^3} + 6x^2y^5z^8e^{xy^2z^3}\\
u'''_{xyz} = 6yz^2e^{xy^2z^3} \left[1 + 3xy^2z^3 + x^2y^4z^6 \right]
\end{gather*}

\subsection*{Задача 2}
\begin{gather*}
z = y\ln{(x^2+y^2)} \\
z'_x = y\dfrac{1}{x^2 + y^2} 2x = \dfrac{2xy}{x^2 + y^2}\\
z'_y = \ln{(x^2+y^2)} + y\dfrac{1}{x^2 + y^2} -2y = \ln{(x^2+y^2)} - \dfrac{2y^2}{x^2 + y^2}\\ 
\dfrac{1}{x} z'_x + \dfrac{1}{y} z'_y = \dfrac{1}{x} \cdot \dfrac{2xy}{x^2 + y^2}  + \dfrac{1}{y} \cdot \left[\ln{(x^2+y^2)} - \dfrac{2y^2}{x^2 + y^2}\right] =\\
\dfrac{2y}{x^2 + y^2} + \dfrac{\ln{(x^2+y^2)}}{y} - \dfrac{2y}{x^2 + y^2} =   \dfrac{\ln{(x^2+y^2)}}{y} \\
\dfrac{z}{y^2} = \dfrac{y\ln{(x^2+y^2)}}{y^2} = \dfrac{\ln{(x^2+y^2)}}{y} \implies \text{ Равенството е вярно. }
\end{gather*}
\begin{gather*}
u = \ln{(x^3+y^3 + z^3 - 3xyz)} \\
u'_x = \dfrac{(x^3+y^3 + z^3 - 3xyz)'_x}{x^3+y^3 + z^3 - 3xyz} =  \dfrac{3x^2 - 3yz}{x^3+y^3 + z^3 - 3xyz}\\
u'_y = \dfrac{(x^3+y^3 + z^3 - 3xyz)'_y}{x^3+y^3 + z^3 - 3xyz} =  \dfrac{3y^2 - 3xz}{x^3+y^3 + z^3 - 3xyz}\\
u'_z = \dfrac{(x^3+y^3 + z^3 - 3xyz)'_z}{x^3+y^3 + z^3 - 3xyz} =  \dfrac{3z^2 - 3xy}{x^3+y^3 + z^3 - 3xyz}\\
u'_x + u'_y + u'_z  = \dfrac{3x^2 - 3yz}{x^3+y^3 + z^3 - 3xyz} + \dfrac{3y^2 - 3xz}{x^3+y^3 + z^3 - 3xyz} + \dfrac{3z^2 - 3xy}{x^3+y^3 + z^3 - 3xyz} =\\
\dfrac{3x^2 - 3yz + 3y^2 - 3xz + 3z^2 - 3xy}{x^3+y^3 + z^3 - 3xyz} = \dfrac{3(x^2 - yz + y^2 - xz + z^2 - xy)}{x^3+y^3 + z^3 - 3xyz} =\\
\dfrac{3(x^2 + y^2 + z^2 -xy -xz -yz)}{x^3+y^3 + z^3 - 3xyz} \cdot \dfrac{x+y+z}{x+y+z} = \dfrac{3(x^3+y^3 + z^3 - 3xyz)}{(x^3+y^3 + z^3 - 3xyz)(x+y+z)} =\\
\dfrac{3}{x+y+z} \implies \text{Равенството е вярно.}
\end{gather*}

\subsection*{Задача 3}
\begin{gather*}
z'_x = \dfrac{1}{1 + \left( \dfrac{x+y}{x-y} \right)^2} \cdot \left( \dfrac{x+y}{x-y} \right)'_x =  \dfrac{1}{\dfrac{(x-y)^2 + (x+y)^2}{(x-y)^2}} \cdot \dfrac{x-y-x-y}{(x-y)^2} \\
z'_x = \dfrac{(x-y)^2}{(x-y)^2 + (x+y)^2} \cdot \dfrac{-2y}{(x-y)^2}\dfrac{-2y}{x^2 -2xy+y^2 + x^2+2xy+y^2} \\
z'_x = \dfrac{-2y}{2(x^2+y^2)} =  -\dfrac{y}{x^2+y^2} \\
z'_y = \dfrac{1}{1 + \left( \dfrac{x+y}{x-y} \right)^2} \cdot \left( \dfrac{x+y}{x-y} \right)'_y  \dfrac{1}{\dfrac{(x-y)^2 + (x+y)^2}{(x-y)^2}} \cdot \dfrac{x-y+x+y}{(x-y)^2} = \\
z'_y = \dfrac{(x-y)^2}{(x-y)^2 + (x+y)^2} \cdot \dfrac{2x}{(x-y)^2} = \dfrac{2x}{x^2 -2xy+y^2 + x^2+2xy+y^2} = \dfrac{x}{x^2+y^2}\\
z'_x + z'_y = -\dfrac{y}{x^2+y^2} + \dfrac{x}{x^2+y^2} = \dfrac{x-y}{x^2+y^2} \implies \text{тъжеството е вярно}
\end{gather*}

\subsection*{Задача 4}

\begin{gather*}
z(x,y) = \dfrac{1}{(x^2+y^2)^2} \\
xz'_x +yz'_y = 2z \\
z'_x = \left( \dfrac{1}{(x^2+y^2)^2}\right)'_x = \left((x^2+y^2)^{-2}\right)'_x = -2(x^2+y^2)^{-3}(x^2+y^2)'_x =-\dfrac{4x}{(x^2+y^2)^3}\\
z'_y = \left( \dfrac{1}{(x^2+y^2)^2}\right)'_y = \left((x^2+y^2)^{-2}\right)'_y = -2(x^2+y^2)^{-3}(x^2+y^2)'_x =-\dfrac{4y}{(x^2+y^2)^3}\\
xz'_x +yz'_y = x \cdot \left( - \dfrac{4x}{(x^2+y^2)^3} \right)+ y \cdot \left( - \dfrac{4y}{(x^2+y^2)^3}\right) = -\dfrac{4x^2}{(x^2+y^2)^3} - \dfrac{4y^2}{(x^2+y^2)^3} = \\
\dfrac{-4(x^2+y^2)}{(x^2+y^2)^3} = -\dfrac{4}{(x^2+y^2)^2}\\
2z = \dfrac{2}{(x^2+y^2)^2}\\
-\dfrac{4}{(x^2+y^2)^2} \neq  \dfrac{2}{(x^2+y^2)^2} \implies \text{Тъждението не е изпълнено.}
\end{gather*}
\begin{gather*}
u(x,y,z) = \sqrt{x^2+y^2+z^2}\cdot \ln{\left( \dfrac{y}{x}\right)}\\
xu'_x + yu'_y + zu'_z = 3z \\
u'_x = \left(\sqrt{x^2+y^2+z^2}\right)'_x \ln{\left( \dfrac{y}{x}\right)} + \sqrt{x^2+y^2+z^2}\left( \ln{\left( \dfrac{y}{x}\right)}\right)'_x \\
u'_y = \left(\sqrt{x^2+y^2+z^2}\right)'_y \ln{\left( \dfrac{y}{x}\right)} + \sqrt{x^2+y^2+z^2}\left( \ln{\left( \dfrac{y}{x}\right)}\right)'_y \\
u'_z = \left(\sqrt{x^2+y^2+z^2}\right)'_z\ln{\left( \dfrac{y}{x}\right)} + \sqrt{x^2+y^2+z^2}\left( \ln{\left( \dfrac{y}{x}\right)}\right)'_z \\
\end{gather*}
\begin{gather*}
u'_x = \left(\sqrt{x^2+y^2+z^2}\right)'_x \ln{\left( \dfrac{y}{x}\right)} + \sqrt{x^2+y^2+z^2}\left( \ln{\left( \dfrac{y}{x}\right)}\right)'_x \\
u'_x = \dfrac{x\ln{\left( \dfrac{y}{x}\right)}}{\sqrt{x^2+y^2+z^2}} - \dfrac{\sqrt{x^2+y^2+z^2}}{x} = \dfrac{x\ln{\left( \dfrac{y}{x}\right)}x - \left(\sqrt{x^2+y^2+z^2}\right)^2}{x\sqrt{x^2+y^2+z^2}}\\
u'_x = \dfrac{x^2\ln{\left( \dfrac{y}{x}\right)} - x^2- y^2- z^2}{x\sqrt{x^2+y^2+z^2}}
\end{gather*} 
\begin{gather*}
u'_y = \left(\sqrt{x^2+y^2+z^2}\right)'_y \ln{\left( \dfrac{y}{x}\right)} + \sqrt{x^2+y^2+z^2}\left( \ln{\left( \dfrac{y}{x}\right)}\right)'_y \\
u'_y = \dfrac{y\ln{\left( \dfrac{y}{x}\right)}}{\sqrt{x^2+y^2+z^2}} + \dfrac{\sqrt{x^2+y^2+z^2}}{y} = \dfrac{y\ln{\left( \dfrac{y}{x}\right)}y + \left(\sqrt{x^2+y^2+z^2}\right)^2}{y\sqrt{x^2+y^2+z^2}}\\
u'_y = \dfrac{y^2\ln{\left( \dfrac{y}{x}\right)} + x^2 + y^2 + z^2}{y\sqrt{x^2+y^2+z^2}}
\end{gather*}
\begin{gather*}
u'_z = \left(\sqrt{x^2+y^2+z^2}\right)'_z\ln{\left( \dfrac{y}{x}\right)} + \sqrt{x^2+y^2+z^2}\left( \ln{\left( \dfrac{y}{x}\right)}\right)'_z \\
u'_z = \dfrac{z\ln{\left( \dfrac{y}{x}\right)}}{\sqrt{x^2+y^2+z^2}} + 0 \cdot \sqrt{x^2+y^2+z^2} = \dfrac{z\ln{\left( \dfrac{y}{x}\right)}}{\sqrt{x^2+y^2+z^2}}
\end{gather*}
\begin{gather*}
xu'_x + yu'_y + zu'_z = 3z , \quad A = xu'_x + yu'_y + zu'_z, \quad B = 3u\\
A = x \cdot \dfrac{x^2\ln{\left( \dfrac{y}{x}\right)} - x^2- y^2- z^2}{x\sqrt{x^2+y^2+z^2}} + y \cdot \dfrac{y^2\ln{\left( \dfrac{y}{x}\right)} + x^2 + y^2 + z^2}{y\sqrt{x^2+y^2+z^2}} + z \cdot \dfrac{z\ln{\left( \dfrac{y}{x}\right)}}{\sqrt{x^2+y^2+z^2}}\\
A = \dfrac{x^2\ln{\left( \dfrac{y}{x}\right)} - x^2- y^2- z^2}{\sqrt{x^2+y^2+z^2}} + \dfrac{y^2\ln{\left( \dfrac{y}{x}\right)} + x^2 + y^2 + z^2}{\sqrt{x^2+y^2+z^2}} + \dfrac{z^2\ln{\left( \dfrac{y}{x}\right)}}{\sqrt{x^2+y^2+z^2}}\\
A = \dfrac{x^2\ln{\left( \dfrac{y}{x}\right)} - x^2- y^2- z^2 + y^2\ln{\left( \dfrac{y}{x}\right)} + x^2 + y^2 + z^2 +z^2\ln{\left(\dfrac{y}{x}\right)}}{\sqrt{x^2+y^2+z^2}} \\
A = \dfrac{x^2\ln{\left( \dfrac{y}{x}\right)} \ln{\left( \dfrac{y}{x}\right)} + z^2\ln{\left(\dfrac{y}{x}\right)}}{\sqrt{x^2+y^2+z^2}} = \dfrac{\ln{\left( \dfrac{y}{x}\right)} (x^2 +y^2 + z^2)}{\sqrt{x^2+y^2+z^2}} = \sqrt{x^2+y^2+z^2}\cdot \ln{\left( \dfrac{y}{x}\right)}\\
B = 3u = 3\sqrt{x^2+y^2+z^2}\cdot \ln{\left( \dfrac{y}{x}\right)} \implies
A \neq B \implies \text{Тъждението не е изпълнено.}
\end{gather*}


\newpage
\section{Упражнение към лекция 6}

\subsection{Задачи}

\subsection*{Задача 1}
Дадени са функцията $z(x,y) = \varphi(x+y) + \psi(x-y)$, където $\varphi, \psi$ - непрекъснато диференцируеми
Да се намерят първите частни производни.

\subsection*{Задача 2}
Да се провери дали $w(x,y,z)$ удволетворява тъждествено равенството:
$$x w_x + y w_y+z w_z = w + \dfrac{xy}{z}$$
Ако $w = \dfrac{xy}{z} + \ln{x} + x \cdot \varphi \left(\dfrac{y}{x}, \dfrac{z}{x}\right), \varphi$ е непрекъснато диференцируема.

\subsection*{Задача 3}
Дадени са функциите и точката М(2,1). Да се пресметне gradf(M) и $\Vert grad f(M)\Vert$
\begin{enumerate}
\item $ f(x,y) = x^2 + 11y^2 - 3$
\item $ f(x,y) = x^2 - y^2$
\item $ f(x,y) = \ln{(x^2 + y^2)}$
\end{enumerate}

\subsection*{Задача 4}
Дадени са функциите и точката М(2,1). \\
Да се пресметне $\dfrac{\partial f(M)}{\partial \nu}, \nu = \left( \dfrac{\sqrt{3}}{2} , \dfrac{1}{2}\right)$ 
\begin{enumerate}
\item $ f(x,y) = x^2 + 11y^2 - 3$
\item $ f(x,y) = x^2 - y^2$
\item $ f(x,y) = \ln{(x^2 + y^2)}$
\end{enumerate}

\subsection*{Задача 5}
Да се определи ъгъла между градиентите на функцията 
$$u = x^2 + y^2 + z^2 - 111$$
в точките А$(\varepsilon,0,0)$ и B $(0,\varepsilon,0), \varepsilon >0$

\subsection*{Задача 6}
Да се намери $y', y''$ на неявната функция $y = f(x)$, дефинирана от уравнението 
$$x^2 - 2xy + 5y^2 + 4y = 2x + 9$$
Да се пресметнат $y'(0), y''(0)$, ако $y(0) = 1$ 

\newpage
\subsection{Решения}

\subsection*{Задача 1}
\begin{gather*}
z(x,y) = \varphi(x+y) + \psi(x-y) \\
z'_x = \varphi'(x+y)(x+y)'_x + \psi'(x-y)(x-y)'_x  = \varphi'(x+y)1 + \psi'(x-y)1\\
z'_x = \varphi'(x+y) + \psi'(x-y)\\
z'_y = \varphi'(x+y)(x+y)'_y + \psi'(x-y)(x-y)'_y = \varphi'(x+y)1 + \psi'(x-y)(-1) \\
z'_y = \varphi'(x+y) - \psi'(x-y)
\end{gather*}

\subsection*{Задача 2}
\begin{gather*}
u = \dfrac{y}{x} \qquad v = \dfrac{z}{x}\\
u'_x = - \dfrac{y}{x^2} \qquad u'_y = \dfrac{1}{x} \qquad u'_z = 0 \\
v'_x = - \dfrac{z}{x^2} \qquad v'_y = 0 \qquad v'_z = \dfrac{1}{x} \\
w'_x = \dfrac{y}{z} \ln{x} + \dfrac{xy}{z} \cdot \dfrac{1}{x} + \varphi \left(\dfrac{y}{x}, \dfrac{z}{x}\right) + x(\varphi'_u u'_x + \varphi'_v v_x) =\\
w'_x =\dfrac{y}{z} \ln{x} + \dfrac{y}{z}+ \varphi \left(\dfrac{y}{x}, \dfrac{z}{x}\right) -\dfrac{y}{x}\varphi'_u  - \dfrac{z}{x}\varphi'_v\\
w'_y = \dfrac{x}{z} \ln{x} + x(\varphi'_u u'_y + \varphi'_v v_y) = \dfrac{x}{z} \ln{x} + \varphi'_u \\
w'_z = -\dfrac{xy}{z^2} \ln{x} + x(\varphi'_u u'_z + \varphi'_v v_z) =  -\dfrac{xy}{z^2} \ln{x} + \varphi'_v\\
x w_x + y w_y+z w_z = \\
= \dfrac{xy}{z} \ln{x} + \dfrac{xy}{z}+ x \varphi \left(\dfrac{y}{x}, \dfrac{z}{x}\right) -y\varphi'_u  - z\varphi'_v + \dfrac{xy}{z} \ln{x} + y\varphi'_u + -\dfrac{xy}{z} \ln{x} + z\varphi'_v =\\
= \dfrac{xy}{z} + \ln{x} + x \cdot \varphi \left(\dfrac{y}{x}, \dfrac{z}{x}\right) +\dfrac{xy}{z} = w +\dfrac{xy}{z} 
\end{gather*}



\subsection*{Задача 3}
$$ grad f = (f'_x, f'_y)$$
\begin{gather*}
 f(x,y) = x^2 + 11y^2 - 3\\
f'_x = 2x \qquad f'_y = 22y\\ 
gradf(x,y) = (2x, 22y)\\ 
gradf(M) = (2\cdot 2, 22 \cdot 1) = (4,22)\\
\Vert grad f(M)\Vert = \sqrt{4^2 + 22^2} = \sqrt{500} =  10\sqrt{5}
\end{gather*}

\begin{gather*}
f(x,y) = x^2 - y^2\\
f'_x = 2x \qquad f'_y = -2y\\ 
gradf(x,y) = (2x, -2y)\\ 
gradf(M) = (2\cdot 2, -2 \cdot 1) = (4,-2)\\
\Vert grad f(M)\Vert = \sqrt{4^2 + (-2)^2} = \sqrt{20} =  2\sqrt{5}
\end{gather*}

\begin{gather*}
f(x,y) = \ln{(x^2 + y^2)}\\
f'_x = \dfrac{2x}{x^2+y^2} \qquad f'_y = \dfrac{2y}{x^2+y^2}\\ 
gradf(x,y) = \left(\dfrac{2x}{x^2+y^2} , \dfrac{2y}{x^2+y^2} \right)\\ 
gradf(M) = \left(\dfrac{2 \cdot 2}{2^2+1^2}, \dfrac{2 \cdot 1}{2^2+1^2} \right) = \left(\dfrac{4}{5} , \dfrac{2}{5} \right) \\
\Vert grad f(M)\Vert = \sqrt{\left( \dfrac{4}{5} \right) ^2 + \left( \dfrac{2}{5} \right) ^2} = \sqrt{\dfrac{20}{25}} = \dfrac{2}{\sqrt{5}}
\end{gather*}

\subsection*{Задача 4}
$$\dfrac{\partial f(M)}{\partial \nu} = (grad f, \nu)$$
\begin{gather*}
 f(x,y) = x^2 + 11y^2 - 3\\
gradf(M) = (2\cdot 2, 22 \cdot 1) = (4,22)\\
\dfrac{\partial f(M)}{\partial \nu} = 4 \cdot \dfrac{\sqrt{3}}{2} + 22 \cdot \dfrac{1}{2} = 2\sqrt{3} + 11
\end{gather*}

\begin{gather*}
f(x,y) = x^2 - y^2\\
gradf(M) = (2\cdot 2, -2 \cdot 1) = (4,-2)\\
\dfrac{\partial f(M)}{\partial \nu} = 4 \cdot \dfrac{\sqrt{3}}{2} -2 \cdot \dfrac{1}{2} = 2\sqrt{3} - 1
\end{gather*}

\begin{gather*}
f(x,y) = \ln{(x^2 + y^2)}\\
gradf(M) = \left(\dfrac{2 \cdot 2}{2^2+1^2}, \dfrac{2 \cdot 1}{2^2+1^2} \right) = \left(\dfrac{4}{5} , \dfrac{2}{5} \right) \\
\dfrac{\partial f(M)}{\partial \nu} = \dfrac{4}{5} \cdot \dfrac{\sqrt{3}}{2} + \dfrac{2}{5} \cdot \dfrac{1}{2} = \dfrac{4\sqrt{3}}{10} + \dfrac{1}{5} = \dfrac{4\sqrt{3} + 2}{10}
\end{gather*}
\subsection*{Задача 5}
\begin{gather*}
u'_x = 2x \qquad u'_y = 2y \qquad u'_z = 2z \\
grad u(A) = (2\varepsilon,0,0) \qquad grad u(B) = (0,2\varepsilon,0) \\
(grad u(A), grad u(B))= 2\varepsilon \cdot 0 + 0 \cdot 2\varepsilon + 0 \cdot 0 = 0 \\
(grad u(A), grad u(B)) = \Vert u(A) \Vert \cdot \Vert u(B) \Vert \cdot \cos{\alpha} \\
\cos{\alpha} = 0 \Leftrightarrow \alpha = \dfrac{\pi}{2}
\end{gather*}

\subsection*{Задача 6}
\begin{gather*}
F(x,y) = x^2 - 2xy + 5y^2 + 4y = 2x + 9\\
F'_y = -2x + 10y + 4 \neq 0\\
F'_x(x,y) = 2x - 2y - 2\\
F'_y(0,1) = -2 \cdot 0 + 10 \cdot 1 + 4 \neq 0\\
y'(x) = - \dfrac{F'_x(x,y)}{F'_y(x,y)} = - \dfrac{2x - 2y - 2}{-2x + 10y + 4} = - \dfrac{x - y - 1}{- x + 5y + 2}\\
y'(0) = - \dfrac{0 - 1 - 1}{- 0 + 5 \cdot 1 + 2} = - \dfrac{-2}{7} = \dfrac{2}{7}\\
y''(x) = - \dfrac{F''_{xx}(x,y) + 2F''_{xy}y' + F''_{yy}(x,y)y'^2}{F'_y(x,y)}\\
F''_{xx} = 2, \quad F''_{yy} = 10, \quad F''_{xy} = -2\\
F''_{xx}(0,1) = 2, \quad F''_{yy}(0,1) = 10, \quad F''_{xy}(0,1) = -2\\
y''(x) = - \dfrac{2 + 2 \cdot (-2)y' + 10y'^2}{-2x + 10y + 4}\\
y''(x) = - \dfrac{2 + - 4y' + 10y'^2}{-2x + 10y + 4}\\
y''(0) = - \dfrac{2 + - 4 \cdot \dfrac{2}{7} + 10 \cdot \left( \dfrac{2}{7} \right) ^2}{-2 \cdot 0 + 10 \cdot 1 + 4}\\
y''(0) = - \dfrac{2 + - \dfrac{8}{7} + \dfrac{40}{49}}{14}\\
y''(0) = - \dfrac{\dfrac{98 - 56 + 40}{49}}{14} = - \dfrac{\dfrac{82}{49}}{14} = \dfrac{82}{49} \cdot \dfrac{1}{14} = \dfrac{41}{343}\\
\end{gather*}

\newpage
\section{Упражнение към лекция 7}

\subsection{Задачи}

\subsection*{Задача 1}
Да се намерят локалните екстремуми на функциите
\begin{itemize}
\item $z = \sin{x} + \sin{y} + \sin{(x+y)} \quad (0 < x < \frac{\pi}{2}, 0 < y < \frac{\pi}{2})$
\item $z = x^4 + y^4 - 4xy$
\end{itemize}

\subsection*{Задача 2}
Да се намерят локалните екстремуми на функциите
\begin{itemize}
\item $u = x^2 + y^2 + z^2 + 2x + 4y - 6z$
\item $u = x^3 + y^2 + z^2 - 3x + 6y - 2z$
\item $u = x^3 + y^2 + z^2 - 3x -2y$
\end{itemize}

\subsection*{Задача 3}
Да се намерят $y'(0), y''(0)$ ако $y(0) = 2$ на неявната функция $y = f(x)$ дефинирана от уравнението 
$$\frac{x^2}{9} + \frac{y^2}{4} = 1$$

\subsection*{Задача 4}
Да се покаже, че функцията $z = f(x,y)$ дефинирана неявно от уравнението 
$$z = x \varphi (\frac{z}{y})$$
$\varphi $ - непрекъснато диференцируема, удовелетворява тъждествено уравнението
$$xz'_x + yz'_y = z$$

\newpage
\subsection{Решения}
\subsection*{Задача 1}


\begin{gather*}
z = \sin{x} + \sin{y} + \sin{(x+y)} \quad (0 < x < \frac{\pi}{2}, 0 < y < \frac{\pi}{2})\\
z'_x = \cos{x} + \cos{(x+y)} \quad z'_y = \cos{y} + \cos{(x+y)}\\
\begin{array}{|l@{}}
\cos{x} + \cos{(x+y)} = 0\\ 
\cos{y} + \cos{(x+y)} = 0
\end{array} \Leftrightarrow
\begin{array}{|l@{}}
2\cos{\frac{2x+y}{2}}\cos{\frac{y}{2}} = 0\\ 
2\cos{\frac{x+2y}{2}}\cos{\frac{x}{2}} = 0
\end{array}\Leftrightarrow
\begin{array}{|l@{}}
\frac{2x+y}{2} =  \frac{\pi}{2}\\ 
\frac{x+2y}{2} =  \frac{\pi}{2}
\end{array} \\
\frac{y}{2} = \frac{\pi}{2}, \quad \frac{x}{2} = \frac{\pi}{2} \implies x = y = \pi \not\in (0 < x ,y < \frac{\pi}{2})\\
x_0 = y_0 = \frac{\pi}{3} \implies M_0 \left( \frac{\pi}{3}, \frac{\pi}{3} \right)\\
z''_{xx} = -\sin{x} - \sin{(x+y)} \quad z''_{yy} = -\sin{y} - \sin{(x+y)} \quad  z''_{xy} = - \sin{(x+y)} \\
z''_{xx}(M_0) = - \frac{2\sqrt{3}}{2} = - \sqrt{3} = \Delta_1 \quad z''_{yy}(M_0) = - \frac{2\sqrt{3}}{2} = - \sqrt{3} \quad  z''_{xy}(M_0) =  - \frac{\sqrt{3}}{2} \\
\begin{pmatrix}
z''_{xx}(M_0) & z''_{xy}(M_0)\\
z''_{yx}(M_0) & z''_{yy}(M_0)
\end{pmatrix} = 
\begin{pmatrix}
 - \sqrt{3} &  - \frac{2\sqrt{3}}{2}\\
 - \frac{2\sqrt{3}}{2} & - \sqrt{3}
\end{pmatrix}\\
\Delta_1 = -\sqrt{3} < 0 \qquad \Delta_2 = 3 - \frac{3}{4} > 0 \\
\implies \exists \text{ локален максимум}, z_{max} = z(M_0) - \frac{3\sqrt{3}}{2}
\end{gather*}

\begin{gather*}
z = x^4 + y^4 - 4xy\\
z'_x = 4x^3 - 4y \quad z'_y =4y^3 - 4x\\
\begin{array}{|l@{}}
4x^3 - 4y = 0\\
4y^3 - 4x = 0
\end{array} \Leftrightarrow
\begin{array}{|l@{}}
y = x^3\\
x^9 - x = 0
\end{array} \Leftrightarrow
\begin{array}{|l@{}}
x(x^2-1)(x^2+1)(x^4+1) = 0\\
y = x^3
\end{array} \implies\\
M_0(0,0) \quad M_1(1,1) \quad M_2(-1,1) \\
z''_{xx} = 12x^2 \quad z''_{yy} = 12y^2 \quad z''_{xy} = -4 \\
d^2z = 
\begin{pmatrix}
z''_{xx}(M_0) & z''_{xy}(M_0)\\
z''_{yx}(M_0) & z''_{yy}(M_0)
\end{pmatrix} = 
\begin{pmatrix}
12x^2 & -4\\
-4 & 12y^2
\end{pmatrix} 
\end{gather*}

\begin{gather*}
d^2z(M_0) = 
\begin{pmatrix}
0 & -4\\
-4 & 0
\end{pmatrix} \implies \Delta = 
\begin{vmatrix}
0 & -4\\
-4 & 0
\end{vmatrix} = -16 < 0 \implies \text{няма лок. екстремум в } M_0 \\
d^2z(M_1) = 
\begin{pmatrix}
12 & -4\\
-4 & 12
\end{pmatrix} \implies \Delta_1 = 12 > 0 \Delta_2 = 
\begin{vmatrix}
0 & -4\\
-4 & 0
\end{vmatrix} 144 - 16 > 0 \implies \\
\text{z има локален минимум}\\
z_{min} = z(M_1) = 1^4 + 1^4 - 4 \cdot 1 \cdot 1 = -2\\
\text{Аналогично и за $M_2$ има лок. мин $z_{min} = -2$}
\end{gather*} 

\subsection*{Задача 2}

\begin{gather*}
u = x^2 + y^2 + z^2 + 2x + 4y - 6z\\
u'_x = 2x + 2 \quad u'_y = 2y + 4 \quad  u'_z = 2z - 6\\
\begin{array}{|l@{}}
x+1 = 0\\
y+2 = 0 \\
z-3 = 0
\end{array} \implies M_0(-1,-2,3)\\
u''_{xx} = 2 \quad u''_{yy} = 2 \quad  u''_{zz} = 2 \\
u''_{xy} = u''_{xz} =u''_{yx} =u''_{yz} = u''_{zx} = u''_{zy} = 0 \\
d^2u(M_0) = 
\begin{pmatrix}
2 & 0 & 0\\
0 & 2 & 0 \\
0 & 0 & 2
\end{pmatrix}\\
\Delta_1 = u''_{xx} = 2 >0 \quad \Delta_2 = 
\begin{vmatrix}
2 & 0 \\
0 & 2 \\
\end{vmatrix} = 4 > 0 \quad \Delta_3 = 
\begin{vmatrix}
2 & 0 & 0\\
0 & 2 & 0 \\
0 & 0 & 2
\end{vmatrix} = 8 > 0 \implies \\
d^2u \text{ e положително дефинитна квадратична форма} \\
\text{u има лок. минимум} \quad u_{min} = u(M_0) = 1 + 4 + 9 -2 -4 \cdot 2 - 18 = -14
\end{gather*}

\begin{gather*}
u = x^3 + y^2 + z^2 - 3x + 6y - 2z\\
u'_x = 3x^2 + 2 \quad u'_y = 2y + 6 \quad  u'_z = 2z - 2\\
\begin{array}{|l@{}}
3x^2 + 2 = 0\\
2y + 6 = 0 \\
2z - 2 = 0
\end{array} \implies M_0(1,-3,1) \, M_1(-1,-3,1)\\
u''_{xx} = 6x \quad u''_{yy} = 2 \quad  u''_{zz} = 2 \\
u''_{xy} = u''_{xz} =u''_{yx} =u''_{yz} = u''_{zx} = u''_{zy} = 0 \\
d^2u(M_0) = 
\begin{pmatrix}
6 & 0 & 0\\
0 & 2 & 0 \\
0 & 0 & 2
\end{pmatrix} \implies\\ 
\Delta_1 = 6 > 0 \quad \Delta_2 = 
\begin{vmatrix}
6 & 0 \\
0 & 2  
\end{vmatrix} = 12 > 0 \quad
\Delta_3 = 
\begin{vmatrix}
6 & 0 & 0\\
0 & 2 & 0 \\
0 & 0 & 2
\end{vmatrix} = 24 > 0 \implies \\
d^2u \text{ e положително дефинитна квадратична форма} \\
\text{u има лок. минимум} \quad u_{min} = u(M_0) = 1 + 9 +1 -3 -18 -2 = -12 \\
d^2u(M_1) = 
\begin{pmatrix}
-6 & 0 & 0\\
0 & 2 & 0 \\
0 & 0 & 2
\end{pmatrix} \implies\\ 
\Delta_1 = -6 < 0 \quad \Delta_2 = 
\begin{vmatrix}
-6 & 0 \\
0 & 2  
\end{vmatrix} = -12 < 0 \quad
\Delta_3 = 
\begin{vmatrix}
-6 & 0 & 0\\
0 & 2 & 0 \\
0 & 0 & 2
\end{vmatrix} = -24 < 0 \implies \\
d^2u \text{ e не е дефинитна квадратична форма} \implies \text{няма лок. екстремуми}
\end{gather*}

\begin{gather*}
u = x^3 + y^2 + z^2 - 3x -2y \\
u'_x = 3x^2 -3  \quad u'_y = 2y - 2 \quad  u'_z = 2z \\
\begin{array}{|l@{}}
3x^2  -3 = 0\\
2y - 2 = 0 \\
2z  = 0
\end{array} \implies M_0(1,1,0) \, M_1(-1,0,0)\\
u''_{xx} = 6x \quad u''_{yy} = 2 \quad  u''_{zz} = 2 \\
u''_{xy} = u''_{xz} =u''_{yx} =u''_{yz} = u''_{zx} = u''_{zy} = 0 \\
d^2u(M_0) = 
\begin{pmatrix}
6 & 0 & 0\\
0 & 2 & 0 \\
0 & 0 & 2
\end{pmatrix} \implies\\ 
\Delta_1 = 6 > 0 \quad \Delta_2 = 
\begin{vmatrix}
6 & 0 \\
0 & 2  
\end{vmatrix} = 12 > 0 \quad
\Delta_3 = 
\begin{vmatrix}
6 & 0 & 0\\
0 & 2 & 0 \\
0 & 0 & 2
\end{vmatrix} = 24 > 0 \implies \\
d^2u \text{ e положително дефинитна квадратична форма} \\
\text{u има лок. минимум} \quad u_{min} = u(M_0) = 1 +1 -3 -2 = -3 \\
d^2u(M_1) = 
\begin{pmatrix}
-6 & 0 & 0\\
0 & 2 & 0 \\
0 & 0 & 2
\end{pmatrix} \implies\\ 
\Delta_1 = -6 < 0 \quad \Delta_2 = 
\begin{vmatrix}
-6 & 0 \\
0 & 2  
\end{vmatrix} = -12 < 0 \quad
\Delta_3 = 
\begin{vmatrix}
-6 & 0 & 0\\
0 & 2 & 0 \\
0 & 0 & 2
\end{vmatrix} = -24 < 0 \implies \\
d^2u \text{ e не е дефинитна квадратична форма} \implies \text{няма лок. екстремуми}
\end{gather*}
\subsection*{Задача 3}
\begin{gather*}
F(x,y) = \frac{x^2}{9} + \frac{y^2}{4} - 1 \quad M_0(0,2)\\
F'_y = \frac{2y}{4} = \frac{y}{2} \neq 0 
y'(x) = - \frac{F'_x}{F'_y} \qquad y''(x) = - \frac{F"_{xx} + 2F''_{xy}y' + F''_{yy}(y')^2}{F'_y}\\
F'_x = \frac{2}{9} x \quad F'_y = \frac{2y}{4} \quad F''_{xx} = \frac{2}{9} \quad F''_{xy} = F''_{yx} = 0 \quad F''_{yy} = \frac{1}{2} \\
F'_x(0,2) = 0 \quad F'_y(0,2) = 1 \quad F''_{xx}(0,2) = \frac{2}{9} \quad F''_{xy}(0,2) = F''_{yx}(0,2) = 0 \quad F''_{yy}(0,2) = \frac{1}{2} \\
y'(0) = - \frac{0}{1} = 0 \qquad y''(0) = - \frac{\frac{2}{9} + 2 \cdot 0 \cdot 0 + \frac{1}{2} \cdot 0^2}{1} = - \frac{2}{9}
\end{gather*}

\subsection*{Задача 4}
\begin{figure}[htp!]
  \includegraphics{Pics/calc/ex7-task4.png}
\end{figure}


\newpage
\section{Упражнение към лекция 8}
\subsection{Задачи}

\subsection*{Задача 1}
Да се изследва за локален екстремум следната функция.
$$z = 1-\sqrt{x^2-y^2}$$

\subsection*{Задача 2}
Намерете точките на условен екстремум и екстремумите на следните функции.
\begin{itemize}
\item $z = x^2 + y^2$, aко $x+y=1$
\item $u = x^2 + y^2 - 12x + 16y$, aко $x^2+y^2=25$
\item $u = x + y + z$, ако $z = 1$ и $x^2 + y^2 = 1$
\end{itemize}

\subsection*{Задача 3}
Да се изследва функцията $u = xy + yz$ за условен екстремум, при ограничения.
$$x^2+y^2 = 2$$
$$y+z = 2$$

\subsection*{Задача 4}
Да се изследва функцията $z = x+y$ за условен екстремум, при ограничения.
$$xy = 1$$

\subsection*{Задача 5}
Да се намери дефиниционното множество на функциите.
\begin{itemize}
\item $z = \sqrt{1 - x^2 - y^2 + 2x}$
\item $z = \frac{x^2y}{2x+y}$
\item $z = \arcsin{(x+y)}$
\item $w = \frac{1}{\sqrt{xy}}$
\end{itemize}

\subsection*{Задача 6}
Да се намерят границите ако съществуват.
\begin{itemize}
\item $\lim\limits_{(x,y) \to (0,0)} \frac{\tan(xy)}{xy}$
\item $\lim\limits_{(x,y) \to (0,0)} \frac{y}{\sin{(xy)}}$
\item $\lim\limits_{(x,y) \to (0,0)} \frac{1 - \sqrt{1 - xy}}{xy}$
\end{itemize}

\subsection*{Задача 7}
Да се провери дали уравнението удовлетворява посочената функция.
\begin{itemize}
\item $\frac{\partial^2 z}{\partial x \partial y} = \frac{\partial^2 z}{\partial y \partial x}, z(x,y) = \ln(x^2+y^2+1)$
\item $\frac{x}{y}\frac{\partial z}{\partial x} + \frac{1}{\ln x} \frac{\partial z}{\partial y } = 2z, z(x,y) =x^y$
\item $2 \frac{\partial^2 z}{\partial x^2 } + \frac{\partial^2 z}{\partial x \partial y} = 0, z(x,y) = 2\cos^2(y - \frac{x}{2})$
\item $\frac{\partial u}{\partial x} + \frac{\partial u}{\partial y} + \frac{\partial u}{\partial z} = 1, u(x,y,z) = x + \frac{x-y}{y-z}$
\end{itemize}

\subsection*{Задача 8}
Да се изследват за локален екстремум следните функции.
\begin{itemize}
\item $z = x^4 + y^4- x^2- 2xy - y^2$
\item $z = xy(1 - x - y)$
\item $z = x^3 - y^3 - 3x + 3y + 2$
\item $u = x^3 + y^2 + z^2 + 12xy + 2z$
\end{itemize}

\subsection*{Задача 9}
Да се изследват за локален екстремум следните неявно зададени функции.
\begin{itemize}
\item $x^3 + y^3 = 3xy, y = y(x)$
\item $y^2 -3y - \sin(x)= 0, y = y(x)$
\item $x^2 + y^2 + z^2 - xz -yz + 2x + 2y + 2z - 2 = 0, z = z(x,y)$
\item $2x^2 + 2y^2 + z^2 + 8xz -8yz + 8 = 0, z = z(x,y)$
\end{itemize}

\subsection*{Задача 10}
Да се изследва за условен екстремум
\begin{itemize}
\item $z = xy$, aко $2x+y=1$
\item $z = x^2 + y^2$, aко $x - y =1$
\item $u = x^2 + y^2 + z^2$, ако $\frac{x^2}{16} + \frac{y^2}{9} + \frac{z^2}{4} = 1$
\item $u = xyz$, ако $x + y + z = 5, xy+yz+zx = 8$ 
\end{itemize}

\subsection*{Задача 11}
Намерете точките на условен екстремум и екстремумите на следните функции.
\begin{itemize}
\item $u = x^2 + y^2+ z^2 +2x + 4y -6z $, aко $ x^2 + y^2+ z^2 = 14$
\item $u = x^2 + y^2 + z^2 + 2x + 4y $, ако $x^2 + y^2 = 20$
\item $u = x^2 + y^2+ z^2 +6x - 2y + 4z $, ако $x^2 + y^2+ z^2 = 56$ 
\end{itemize}

\subsection*{Задача 12}
Намерете абсолютните екстремуми на следните функции и определете вида им (условен, локален, минимум, максимум)
\begin{itemize}
\item $u = x^2 + y^2 - 12x + 16y$, aко $ x^2 + y^2 \leq 25, x^2 + y^2 \leq 400, x^2 + y^2 \leq 100$
\item $u = x^2 + y^2+ z^2 +2x + 4y -6 $, ако $x^2 + y^2 + z^2 \leq 9$
\item $u = x^2 + 2y^2+ 3z^2$, ако $x^2 + y^2+ z^2 \leq 100$ 
\end{itemize}

\newpage
\subsection{Решения}
\subsection*{Задача 1}
$$z(\Delta x, \Delta y) - z(0,0) = 1-\sqrt{\Delta x^2 - \Delta y^2} - 1 = -\sqrt{\Delta x^2 - \Delta y^2}< 0$$
Имаме строг локален максимум в $z(0,0) = 1$

\subsection*{Задача 2}
\begin{gather*}
z = x^2 + y^2, \quad x+y = 1\\
F(x,y\lambda) = x^2 + y^2 + \lambda(x+y-1) , \quad \lambda \neq 0\\
F'_x = 2x + \lambda \qquad F'_y = 2y + \lambda \\
\begin{array}{|l@{}}
2x + \lambda = 0 \\
2y + \lambda = 0\\
x+y = 1
\end{array}
\Leftrightarrow
\begin{array}{|l@{}}
x = -\frac{\lambda}{2}\\
y = -\frac{\lambda}{2} \\
 -\frac{\lambda}{2}   -\frac{\lambda}{2} = 1
\end{array}
\Leftrightarrow
\begin{array}{|l@{}}
x = \frac{1}{2} \\
y = \frac{1}{2} \\
\lambda = -1
\end{array} \implies M_0 \left(\frac{1}{2}, \frac{1}{2}, -1 \right) \\
F"_{xx} = 2 \quad F"_{yy} = 2 \qquad F"_{xy} = F"_{yx} = 0 \\
\Delta = F''_{xx}F_{yy} - F''_{xy}F''_{yx} = 4 \Big|_{M_0} = 4 > 0 \\
F''_{xx} \Big|_{M_0} = 2 > 0 \implies z_{min} = z\left(\frac{1}{2}, \frac{1}{2} \right) = \frac{1}{4} + \frac{1}{4} = \frac{1}{2}
\end{gather*}

\begin{gather*}
u = x^2 + y^2 - 12x + 16y, \quad x^2 + y^2 = 25\\
F(x,y,\lambda) = x^2 + y^2 - 12x - 16x + \lambda(x^2 + y^2 - 25) \quad \lambda \neq 0\\
F'_x = 2x + 16 + 2\lambda x \qquad F'_y = 2y - 12 + 2\lambda y \\
\begin{array}{|l@{}}
2x + 16 + 2\lambda x = 0\\
2y - 12 + 2\lambda y = 0 \\
x^2 + y^2 = 25
\end{array}\Leftrightarrow 
\begin{array}{|l@{}}
x = \frac{6}{1 + \lambda} \\
y = - \frac{8}{1 + \lambda}\\
\frac{36}{(1+\lambda)^2} + \frac{64}{(1+ \lambda)^2}=25
\end{array}\Leftrightarrow
\begin{array}{|l@{}}
x = -3 \land x = 3\\
y = 4 \land y = -4\\
\lambda = -3 \land \lambda = 1
\end{array}\\
\implies M_1(-3,4,-3) \land M_2(3,-4,1)\\
F''_{xx} = 2+ 2\lambda \quad F''_{yy} = 2 + 2\lambda \quad F''_{xy} = F"_{yx} = 0\\
\Delta = F''_{xx}F_{yy} - F''_{xy}F''_{yx} = 4(1+\lambda)^2\\
\end{gather*}
\begin{gather*}
\Delta \Big|_{M_1} = 4(1+(-3))^2 = 16 > 0 \\
F''_{xx} \Big|_{M_1} = 2 + 2\cdot -3 = -4 < 0 \implies \\
u_{max} = u(-3,4) = (-3)^2 + 4^2 - 12(-3) + 16(4) = 125 \\
\Delta \Big|_{M_2} = 4(1+1)^2 = 16\\
F''_{xx} \Big|_{M_2} = 2 + 2 \cdot 1 = 4 >0 \implies \\
u_{min} = u(3,-4) = 3^2 + (-4)^2 - 12\cdot 3 + 16(-4) = -75
\end{gather*}

\begin{gather*}
u= x+y+z, \qquad z = 1, \, x^2 + y^2 = 1\\
F(x,y,z,\lambda, \nu) = x+y+z + \lambda(z-1) + \nu(x^2+y^2-1), \quad \lambda, \nu \neq 0\\
F'_x = 1 + 2\nu x \qquad F'_y = 1 + 2\nu y \qquad F'_z = 1 + \lambda \\
\begin{array}{|l@{}}
1 + 2\nu x = 0 \\
1 + 2\nu y = 0\\
1 + \lambda = 0\\
z = 1 \\
x^2+y^2 = 1
\end{array}\Leftrightarrow 
\begin{array}{|l@{}}
x = - \frac{1}{2\nu} \\
y = - \frac{1}{2\nu} \\
z = 1\\
\lambda = -1 \\
\frac{1}{4\nu^2} + \frac{1}{4\nu^2} = 1
\end{array}\Leftrightarrow 
\begin{array}{|l@{}}
x = -\frac{1}{\sqrt{2}} \land \frac{1}{\sqrt{2}}\\
y = -\frac{1}{\sqrt{2}} \land \frac{1}{\sqrt{2}}\\
z = 1\\
\lambda = -1\\
\nu = \frac{1}{\sqrt{2}} \land -\frac{1}{\sqrt{2}}\\
\end{array} \implies \\
M_1 \left( -\frac{1}{\sqrt{2}}, -\frac{1}{\sqrt{2}}, 1, -1, \frac{1}{\sqrt{2}}\right)\land
M_2  \left( \frac{1}{\sqrt{2}}, \frac{1}{\sqrt{2}}, 1, -1, -\frac{1}{\sqrt{2}}\right)\\
F''_{xx} = 2\nu \quad F"_{yy} = 2\nu \quad F''_{zz} = F''_{xy} = F''_{xz} = F''_{yx} = F''_{yz} = F''_{zx} = F''_{zy} = 0\\
d^2F = \begin{pmatrix}
2\nu & 0 & 0 \\
0 & 2\nu & 0 \\
0 & 0 & 0 \\
\end{pmatrix} \\
\Delta_1 = F''_{xx} = 2\mu \Big|_{M_1} =\frac{2}{\sqrt{2}} > 0 \quad \Delta_2 = 4\nu^2 \Big|_{M_1} = 2 > 0 \quad \Delta_3 = 0 \Big|_{M_1} = 0 \\ 
\Delta_1 = F''_{xx} = 2\mu \Big|_{M_2} = -\frac{2}{\sqrt{2}} < 0 \quad \Delta_2 = 4\nu^2 \Big|_{M_2} = 2 > 0 \quad \Delta_3 = 0 \Big|_{M_2} = 0 \\ 
d^2 F = 2\nu \, dx^2 + 2\nu \, dy^2\\
x^2 + y^2 = 1 \implies x\, dx + y\, dy = 0 (x=y \in M_1, M_2) \implies dx + dy = 0  \implies dy = -dx \\
d^2F= 2\nu \, dx^2 + 2\nu\, (-dx)^2 = 4\nu \, dx^2\\
d^2F \Big|_{M_1} >0 \implies u_{min} = u \left( -\frac{1}{\sqrt{2}}, -\frac{1}{\sqrt{2}}\right) = -\sqrt{2} + 1 \\
d^2F \Big|_{M_2} < 0 \implies u_{max} = u \left( \frac{1}{\sqrt{2}}, \frac{1}{\sqrt{2}}\right) = \sqrt{2} + 1 \\
\end{gather*}

\subsection*{Задача 3}
\begin{gather*}
u = xy+yz \quad x^2 + y^2 = 2, \, y+z = 2 \\
F(x,y,z,\lambda, \nu) = xy + yz + \lambda(x^2 + y^2 - 2) + \nu(y + z - 2), \qquad \lambda, \nu \neq 0 \\
F'_x = 2\lambda x + y \qquad F'_y = 2 \lambda y + \nu + x + z \qquad F'_z = y + \nu \\
\begin{array}{|l@{}}
2\lambda x + y = 0 \\
2 \lambda y + \nu + x + z  = 0 \\
y + \nu = 0 \\
x^2 + y^2 = 2 \\
y+z = 2 
\end{array} \Leftrightarrow 
\begin{array}{|l@{}}
x = 1 \land x = -1\\
y = 1 \\
z = 1 \\
\lambda = - \frac{1}{2} \land \lambda = \frac{1}{2} \\
\nu = -1
\end{array} \implies M \left(1,1,1,-\frac{1}{2}, -1 \right) \land N \left(-1,1,1,\frac{1}{2}, -1 \right) \\
F''_{xx} = 2\lambda \qquad F''_{yy} = 2\lambda \qquad F''_{zz} = 0 \\
F''_{xy} = F''_{yx} = F''_{yz} = F''_{zy} = 1 \\
F''_{xz} = F''_{zx} = 0 \\
d^2 F = 
\begin{pmatrix}
2\lambda & 1 & 0 \\
1 & 2\lambda & 1 \\
0 & 1 & 0
\end{pmatrix} \implies d^2 F = 2\lambda \,dx^2 + 2\lambda \,dy^2 + 2 \, dx \,dy + 2 \,dy \,dz \\
2x \,dx + 2y\,dy = 0 \qquad dy + dz = 0 \\
d^2 L(M) = -dx^2 - dy^2 + 2 \, dx \, dy + 2 \, dy \, dz \quad (dx + dy = 0, dy + dz = 0, dx + dz = -2 \, dy ) \\
d^2 L(M) = -dx^2 - dy^2 + 2 (dx + dz) dy = -dx^2 -dy^2 - 2dy^2 < 0 \implies \\
u_{max} =u(1,1,1) = 1 \cdot 1 + 1 \cdot 1 = 2\\
d^2(N) = dx^2 + dy^2 + 2(dx + dz)dy \quad (-dx + dy = 0, dy+dz = 0, dx + dz = 0)\\
d^2 (N)= dx^2 dy^2 - 2(dx + dz)dy = dx^2+dy^2 >0 \implies \\
u_{min} = u(-1,1,1) = -1 \cdot 1 + 1 \cdot 1 = 0 
\end{gather*}

\subsection*{Задача 4}
\begin{gather*}
z = x + y \qquad xy = 1\\
F = x + y + \lambda(xy - 1) \implies F'_x = 1 + \lambda y \quad F'_y = 1 + \lambda x \\
\begin{array}{|l@{}}
1 + \lambda y = 0\\
1 + \lambda x = 0 \\
xy = 1
\end{array}\Leftrightarrow 
\begin{array}{|l@{}}
x = 1 \land x = -1\\
y = 1 \land y = -1\\
\lambda = -1 \land \lambda = 1
\end{array} \implies M(1,1,-1) \land N(-1,-1,1) \\
F''_{xx} = F''_{yy} = 0 ,\qquad F''_{xy} = F''_{yx} = \lambda \implies d^2 F = 2\lambda \, dx \, dy \\
x \, dx + y \, dy = 0 \implies dx+dy =0 (x,y \in M, N)  \implies d^2 F = -2 \lambda \, dx^2\\
d^2 F(M) = 2 dx^2 > 0 \implies u_{min} = u(1,1) = 1+1 = 2 \\ 
d^2 F(N) = -2 dx^2 < 0 \implies u_{max} = u(-1,-1) = -1 + (-1) = -2 
\end{gather*}

\subsection*{Задача 5}
\begin{itemize}
\item $z = \sqrt{1 - x^2 - y^2 + 2x}$
$$D: (x, y \in \mathbb{R}: x^2 + y^2 \leq 2x + 1)$$
\item $z = \frac{x^2y}{2x+y}$
$$D: (x,y \in \mathbb{R}: 2x + y \neq 0)$$
\item $z = \arcsin{(x+y)}$
$$D: (x,y \in \mathbb{R}: -1 \leq x+y \leq 1)$$
\item $w = \frac{1}{\sqrt{xy}}$
$$D: (x,y \in \mathbb{R} \setminus 0: xy > 0)$$
\end{itemize}

\subsection*{Задача 6}
\begin{itemize}
\item $\lim\limits_{(x,y) \to (0,0)} \frac{\tan(xy)}{xy}$\\
При заместване се получават недефинирани форми от вида $\left[ \frac{0}{0}\right]$
\begin{gather*}
\lim\limits_{(x,y) \to (0,0)} \frac{\tan(xy)}{xy} = \lim\limits_{(x,y) \to (0,0)} \frac{\sin(xy)}{xy\cos(xy)} = \lim\limits_{(x,y) \to (0,0)} \\
\frac{\sin(xy)}{xy} =  \lim\limits_{(x,y) \to (0,0)} \frac{y\cos(xy)}{y} =  \lim\limits_{(x,y) \to (0,0)} \cos(xy) = 1
\end{gather*}
\item $\lim\limits_{(x,y) \to (0,0)} \frac{y}{\sin(xy)}$\\
\begin{gather*}
\lim\limits_{y \to 0} \frac{y}{\sin(xy)} = \left[ \frac{0}{0} \right] \implies \lim\limits_{y \to 0} \frac{1}{x\cos(xy)} = \frac{1}{x} \\
\lim\limits_{x \to 0} \frac{1}{x} = \frac{1}{0} = \infty \\
\lim\limits_{x \to 0} \frac{y}{\sin{xy}} = \left[ \frac{0}{0} \right] \implies \lim\limits_{x \to 0} \frac{0}{y} = 0 \\
\lim\limits_{y \to 0} 0 = 0 \\
0 \neq \infty \implies \text{Няма граница.}
\end{gather*} 
\item $\lim\limits_{(x,y) \to (0,0)} \frac{1 - \sqrt{1 - xy}}{xy}$
\begin{gather*}
\lim\limits_{y \to 0} \frac{1 - \sqrt{1 - xy}}{xy} = \left[ \frac{0}{0} \right] \implies 
\lim\limits_{y \to 0} \frac{x}{2\sqrt{1-xy}} \cdot \frac{1}{x} = \lim\limits_{y \to 0} \frac{1}{2\sqrt{1-xy}} = \frac{1}{2} \\
\lim\limits_{x \to 0} \frac{1}{2} = \frac{1}{2}\\
\lim\limits_{x \to 0} \frac{1 - \sqrt{1 - xy}}{xy} = \left[ \frac{0}{0} \right] \implies 
\lim\limits_{x \to 0} \frac{y}{2\sqrt{1-xy}} \cdot \frac{1}{y} = \lim\limits_{x \to 0} \frac{1}{2\sqrt{1-xy}} = \frac{1}{2} \\
\lim\limits_{y \to 0} \frac{1}{2} = \frac{1}{2}
\end{gather*}
\end{itemize}

\subsection*{Задача 7}

\begin{itemize}
\item $\frac{\partial^2 z}{\partial x \partial y} = \frac{\partial^2 z}{\partial y \partial x}, z(x,y) = \ln(x^2+y^2+1)$
\begin{gather*}
z'_x = \frac{2x}{x^2 + y^2 + 1} \qquad z'_y = \frac{2y}{x^2 + y^2 + 1} \\
z''_{xy} = -\frac{4xy}{(x^2+y^2 + 1)^2} \qquad z''_{yx} = -\frac{4xy}{(x^2+y^2 + 1)^2} \implies \\
\frac{\partial^2 z}{\partial x \partial y} = \frac{\partial^2 z}{\partial y \partial x} \text{ e вярно}
\end{gather*}

\item $\frac{x}{y}\frac{\partial z}{\partial x} + \frac{1}{\ln x} \frac{\partial z}{\partial y } = 2z, z(x,y) =x^y$
\begin{gather*}
z'_x = \frac{x^y y}{x} \qquad z'_y = x^y \ln(x) \\
A = \frac{x}{y} \cdot z'_x = \frac{x}{y} \cdot  \frac{x^y y}{x} = x^y \\
B = \frac{1}{\ln(x)} \cdot z'_y = \frac{1}{\ln(x)} \cdot x^y \ln(x) = x^y \\
A + B = 2x^y \qquad 2z = 2x^y \implies \\
\frac{x}{y}\frac{\partial z}{\partial x} + \frac{1}{\ln x} \frac{\partial z}{\partial y } = 2z \text{ e вярно}
\end{gather*}

\item $2 \frac{\partial^2 z}{\partial x^2 } + \frac{\partial^2 z}{\partial x \partial y} = 0, z(x,y) = 2\cos^2(y - \frac{x}{2})$
\begin{gather*}
z'_x = -2 \cos \left (-y + \frac{x}{2} \right) \sin \left (-y + \frac{x}{2} \right) \\
z''_{xx} = -2 \cos^2 \left (-y + \frac{x}{2} \right) + 1 \qquad z''_{xy} = 4 \cos^2 \left (-y + \frac{x}{2} \right) - 2 \\
2z''_{xx} = -4 \cos^2 \left (-y + \frac{x}{2} \right) + 2 \\
2z''_{xx} + z''_{xy} = -4 \cos^2 \left (-y + \frac{x}{2} \right) + 2 + 4 \cos^2 \left (-y + \frac{x}{2} \right) - 2 = 0 \implies \\
2 \frac{\partial^2 z}{\partial x^2 } + \frac{\partial^2 z}{\partial x \partial y} = 0 \text{ e вярно}
\end{gather*}
\item $\frac{\partial u}{\partial x} + \frac{\partial u}{\partial y} + \frac{\partial u}{\partial z} = 1, u(x,y,z) = x + \frac{x-y}{y-z}$ 
\begin{gather*}
u'_x = \frac{y-z+1}{y-z} \qquad u'_y = \frac{z-x}{(y-x)^2} \qquad u'_z = \frac{x-y}{(y-z)^2} \\
u'_x + u'_y + u'_z = \frac{y-z+1}{y-z} + \frac{z-x}{(y-x)^2} + \frac{x-y}{(y-z)^2} = \\
\frac{(y-z+1)(y-z) +z-x + x-y}{(y-z)^2} = \frac{y^2 - yz -yz + z^2 +y- z +z -y}{(y-z)^2} = \\ = \frac{y^2 -2yz + z^2}{(y-z)^2} = 1 \implies \\
\frac{\partial u}{\partial x} + \frac{\partial u}{\partial y} + \frac{\partial u}{\partial z} = 1 \text{ e вярно}
\end{gather*}
\end{itemize}

\subsection*{Задача 8}

\begin{itemize}
\item $z = x^4 + y^4- x^2- 2xy - y^2$
\begin{gather*}
z'_x = 4x^3 + 2x -2y \qquad z'_y = 4y^3 - 2y - 2x \\
\begin{array}{|l@{}}
4x^3 + 2x - 2y = 0 \\
4y^3 - 2y - 2x = 0 
\end{array} \Leftrightarrow 
\begin{array}{|l@{}}
x = 0 \land x = 1 \land x = -1\\ 
y = 0 \land y = 1 \land y = -1
\end{array} \implies L(0,0),\, M(1,1), \, N(-1,-1)\\
z''_{xx}  = 12x^2 + 2 \qquad z''_{yy} = 12y^2 - 2 \qquad z'_{xy} = z'_{yx} = -2 \\
\Delta = z''_{xx}z''_{yy} - z''_{xy}z''_{yx} = (12x^2 - 2)(12y^2 - 2) - 4 \\
z''_{xx}(L) = -2< 0 \qquad \Delta(L) =0 \implies \text{Няма екстремум.}\\
z''_{xx}(M) = 10 > 0 \qquad \Delta(M) = 96>0 \implies \\
z'_{min1} = z(1,1) = -2 \\
z''_{xx}(N) = 10 > 0 \qquad \Delta(N) = 96>0 \implies \\
z'_{min2} = z(-1,-1) = -2 
\end{gather*}

\item $z = xy(1 - x - y)$
\begin{gather*}
z'_x = y(1- x - y)- xy \qquad z'_y =  x(1- x - y)- xy \\
\begin{array}{|l@{}}
y(1- x - y)- xy= 0 \\
x(1- x - y)- xy = 0
\end{array} \Leftrightarrow 
\begin{array}{|l@{}}
x= 0 \land x = 0 \land x = 1 \land x=\frac{1}{3}\\
y= 0 \land y = 1 \land y = 0 \land y=\frac{1}{3}
\end{array} \implies \\
A(0,0), \, B(0,1), \, C(1,0), \, D \left(\frac{1}{3},\frac{1}{3} \right) \\
z''_{xx} -2y = \qquad z''_{yy} = -2y \qquad z''_{xy} = z''_{yx} = 1 - 2x - 2y\\
\Delta = z''_{xx}z''_{yy} - z''_{xy}z''_{yx} = 4xy - (1 - 2x - 2y)^2 \\
z''_{xx}(A) = 0 \qquad \Delta(A) = -1 \implies \text{Няма екстремум.} \\
z''_{xx}(B) = -2 \qquad \Delta(A) = -1 \implies \text{Няма екстремум.} \\
z''_{xx}(C) = 0 \qquad \Delta(C) = -1 \implies \text{Няма екстремум.} \\
z''_{xx}(D) = -\frac{2}{3} \qquad \Delta(D) = \frac{1}{3} \implies z_{max} = z\left(\frac{1}{3},\frac{1}{3} \right) = \frac{1}{27}
\end{gather*}

\item $z = x^3 - y^3 - 3x + 3y + 2$
\begin{gather*}
z'_x = 3x^2 - 3 \qquad z'_y =  -3y^2 + 3\\
\begin{array}{|l@{}}
3x^2 - 3 = 0 \\
-3y^2 + 3 = 0
\end{array} \Leftrightarrow 
\begin{array}{|l@{}}
x = 1 \land x =-1 \\
x = 1 \land x =-1 
\end{array} \implies A(1,1), \, B(1,-1), \, C(-1, 1), \, D(-1,-1)\\
z''_{xx}  6x= \qquad z''_{yy} = -6y \qquad z''_{xy} = z''_{yx} = 0\\
\Delta = z''_{xx}z''_{yy} - z''_{xy}z''_{yx} = -36xy\\
z''_{xx}(A) = 6>0 \qquad \Delta(A) = -36<0 \implies \text{Няма екстремум.}\\
z''_{xx}(B) = 6>0 \qquad \Delta(A) = 36>0 \implies z_{min} = z(1,-1) = -2\\
z''_{xx}(C) = -6<0 \qquad \Delta(A) = 36>0 \implies z_{max} = z(-1,1) = 6\\
z''_{xx}(D) = -6<0 \qquad \Delta(A) = -36<0 \implies \text{Няма екстремум.} 
\end{gather*}

\item $u = x^3 + y^2 + z^2 + 12xy + 2z$
\begin{gather*}
u'_x = 3x^2 - 12y \qquad u'_y = 3y^2 - 12x \qquad u'_z = 2z^2 + 2\\
\begin{array}{|l@{}}
3x^2 - 12y = 0\\
3y^2 - 12x = 0\\
2z^2 + 2 = 0
\end{array} \Leftrightarrow 
\begin{array}{|l@{}}
x = 0 \land x = 24 \\
y = 0 \land y = -144 \\
z = -1 
\end{array} \implies A(0,0,-1), \, B(24,-144,-1) \\
u''_{xx} = 6x \qquad u''_{xy} = 12 \qquad u''_{xz} = 0\\
u''_{yx} = 12 \qquad u''_{yy} = 2 \qquad u''_{yz} = 0\\
u''_{zx} = 0 \qquad u''_{zy} = 0 \qquad u''_{zz} = 2 \\
\Delta_1 = u''_{xx} = 6x \\
\Delta_2  = \begin{vmatrix} u''_{xx} & u''_{xy} \\ u''_{yx} & u''_{yy} \end{vmatrix} = \begin{vmatrix} 6x & 12 \\ 12 & 2 \end{vmatrix} = 12x - 144\\
\Delta_3 = \begin{vmatrix}
u''_{xx} & u''_{xy}  & u''_{xz}\\ u''_{yx} & u''_{yy} & u''_{yz} \\ u''_{zx} & u''_{zy} & u''_{zz} \end{vmatrix} =
\begin{vmatrix} 6x & 12 & 0 \\ 12 & 2 & 0 \\ 0 & 0 & 2 \end{vmatrix} = 24x - 288 \\
\Delta_1 (A) = 0 \qquad \Delta_2 (A) = -144 \qquad \Delta_3 (A) = -288 \implies \text{Няма екстремум.}\\
\Delta_1 (B) = 144 > 0 \qquad \Delta_2 (B) = 144 > 0 \qquad \Delta_3 (B) = 288 > 0\implies\\
u_{min} = u(24,-144,-1) = -6913
\end{gather*}

\end{itemize}

\subsection*{Задача 9}
\begin{itemize}

\item $x^3 + y^3 = 3xy, y = y(x)$
\begin{gather*}
F = x^3 + y^3 - 3xy \\
F'_x = 3x^2 - 3y \qquad F'_y = 3y^2 - 3x \\
y' = -\frac{f'_x}{f'_y} = - \frac{3x^2 - 3y}{3y^2 - 3x} \\
\begin{array}{|l@{}}
- \frac{3x^2 - 3y}{3y^2 - 3x} = 0 \\
x^3 + y^3 = 3xy \\
3y^2 - 3x \neq 0 
\end{array} \Leftrightarrow 
\begin{array}{|l@{}}
x = \sqrt[3]{2} \\
y = \sqrt[3]{2^2}
\end{array} \implies A = (\sqrt[3]{2}, \sqrt[3]{2^2}) \\
F''_{xx} = 6x \\
y'' = -\frac{f''_{xx}}{f'_y} = -\frac{6x}{3y^2-3x}\\
y''(A) = -2 <0 \implies y_{max} = y(\sqrt[3]{2}) = \sqrt[3]{2^2}
\end{gather*}

\item $y^2 -3y - \sin(x)= 0, y = y(x)$
\begin{gather*}
F = y^2 -3y - \sin(x) \\
F'_x = -\cos(x) \qquad F'_y = 2y-3 \\
y' = -\frac{f'_x}{f'_y} = \frac{\cos(x)}{2y-3} \\
\begin{array}{|l@{}}
\frac{\cos(x)}{2y-3} = 0 \\
y^2 -3y - \sin(x)= 0 \\
2y-3 \neq 0 
\end{array} \Leftrightarrow 
\begin{array}{|l@{}}
x = \frac{\pi}{2} \land x = -\frac{\pi}{2} \\
y = \frac{3 +\sqrt{13}}{2} \land y = \frac{3 - \sqrt{5}}{2}
\end{array} \implies \\
A = \left(\frac{\pi}{2} + 2k\pi, \frac{3 +\sqrt{13}}{2} \right), \, B \left(-\frac{\pi}{2}+2k\pi, \frac{3 - \sqrt{5}}{2} \right) \\
F''_{xx} = \sin(x) \\
y'' = -\frac{f''_{xx}}{f'_y} = -\frac{\sin(x)}{2y - 3}\\
y''(A) = -\frac{\sqrt{13}}{13} < 0 \implies y_{max,k} = y\left( \frac{\pi}{2} + 2k\pi \right) = \frac{3 +\sqrt{13}}{2}\\
y''(B) = -\frac{\sqrt{5}}{5} < 0 \implies y_{min,k} =  y\left( -\frac{\pi}{2} + 2k\pi \right) = \frac{3 - \sqrt{5}}{2} \\
\text{(Панева, как тва е минимум беее)}
\end{gather*}

\item $x^2 + y^2 + z^2 - xz -yz + 2x + 2y + 2z - 2 = 0, z = z(x,y)$
\begin{gather*}
F = x^2 + y^2 + z^2 - xz -yz + 2x + 2y + 2z - 2\\
F'_x = 2x - z + 2 \qquad F'_y = 2y - z + 2 \qquad F'_z = -x -y + 2z + 2 \\
z'_x = - \frac{F'_x}{F'_z} = - \frac{2x - z + 2}{-x -y + 2z + 2}\qquad z'_y = -\frac{F'_y}{F'_z} = -\frac{2y - z + 2}{-x -y + 2z + 2}\\
\begin{array}{|l@{}}
- \frac{2x - z + 2}{-x -y + 2z + 2} = 0 \\
-\frac{2y - z + 2}{-x -y + 2z + 2} = 0 \\
x^2 + y^2 + z^2 - xz -yz + 2x + 2y + 2z - 2 = 0\\
 -x -y + 2z + 2 \neq 0
\end{array}\Leftrightarrow 
\begin{array}{|l@{}}
x = -3 + \sqrt{6} \\
y = -3 + \sqrt{6} \\
z = -4 + 2\sqrt{6}
\end{array} \implies \\
A( -3 + \sqrt{6},-3 + \sqrt{6}, -4 + 2\sqrt{6})
\end{gather*}
\begin{gather*}
F''_{xx} =  2 \qquad F''_{xy} =  0 \qquad F''_{yx} = 0 \qquad F''_{yy} = 2 \\
\Delta = \begin{vmatrix} -\frac{F''_{xx}}{F'_z} &  -\frac{F''_{xy}}{F'_z} \\  -\frac{F''_{yx}}{F'_z} &  -\frac{F''_{yy}}{F'_z} \end{vmatrix} = \begin{vmatrix} -\frac{2}{ -x -y + 2z + 2 } & 0 \\  0 &  -\frac{2}{ -x -y + 2z + 2 } \end{vmatrix} = \frac{4}{( -x -y + 2z + 2 )^2} \\
-\frac{F''_{xx}}{F'_z} (A) = -\frac{\sqrt{6}}{6} \qquad \Delta(A) = \frac{1}{6} \implies z_{max} = z( -3 + \sqrt{6},-3 + \sqrt{6}) =  -4 + 2\sqrt{6}\\
\text{от къде дойде втората точка в решението на Панева}
\end{gather*}

\item $2x^2 + 2y^2 + z^2 + 8xz -8yz + 8 = 0, z = z(x,y)$
\begin{gather*}
F = 2x^2 + 2y^2 + z^2 + 8xz -8yz + 8 \\
F'_x = 4x+ 8z \qquad F'_y = 4y - 8z \qquad F'_z = 8x - 8y + 2z \\
z'_x = - \frac{F'_x}{F'_z} = - \frac{4x+ 8z}{8x - 8y + 2z}\qquad z'_y = -\frac{F'_y}{F'_z} = - \frac{4y - 8z}{8x - 8y + 2z}\\
\begin{array}{|l@{}}
- \frac{4x+ 8z}{8x - 8y + 2z} = 0 \\
- \frac{4y - 8z}{8x - 8y + 2z} = 0\\
2x^2 + 2y^2 + z^2 + 8xz -8yz + 8 = 0\\
8x - 8y + 2z \neq 0\\
\end{array}\Leftrightarrow 
\begin{array}{|l@{}}
x = - \frac{4 \sqrt{30}}{15}\\
y = \frac{4 \sqrt{30}}{15}\\
z = \frac{2 \sqrt{30}}{15}
\end{array} \implies 
A \left( - \frac{4 \sqrt{30}}{15}, \frac{4 \sqrt{30}}{15}, \frac{2 \sqrt{30}}{15}\right) \\
F''_{xx} = 4 \qquad F''_{xy} = 0 \qquad  F''_{yx} = 0 \qquad  F''_{yy} =4 \\
\Delta = \begin{vmatrix} -\frac{F''_{xx}}{F'_z} &  -\frac{F''_{xy}}{F'_z} \\  -\frac{F''_{yx}}{F'_z} &  -\frac{F''_{yy}}{F'_z} \end{vmatrix} = \begin{vmatrix} -\frac{4}{8x - 8y + 2z} & 0 \\  0 &  -\frac{4}{ 8x - 8y + 2z} \end{vmatrix} = \frac{16}{(8x - 8y + 2z)^2} \\
-\frac{F''_{xx}}{F'_z} (A) = \frac{\sqrt{30}}{30}>0 \qquad \Delta(A) = \frac{1}{30}>0 \implies z_{min} = z\left( - \frac{4 \sqrt{30}}{15}, \frac{4 \sqrt{30}}{15} \right) = \frac{2 \sqrt{30}}{15}
\end{gather*}
\end{itemize}

\subsection*{Задача 10}

\begin{itemize}
\item $z = xy$, aко $2x+y=1$
\begin{gather*}
F(x,y,\lambda)   = xy + \lambda(2x+y-1) \quad \lambda \neq 0\\
F'_x = 2\lambda +y \qquad F'_y = \lambda + x \\
\begin{array}{|l@{}}
2\lambda +y= 0 \\
\lambda + x = 0\\
2x+y=1 
\end{array}\Leftrightarrow 
\begin{array}{|l@{}}
x = \frac{1}{4} \\
y = \frac{1}{2} \\
\lambda = -\frac{1}{4}
\end{array} \implies 
A \left( \frac{1}{4}, \frac{1}{2}, -\frac{1}{4}\right) \\
F''_{xx} = 0 \qquad F''_{xy} = 1 \qquad F''_{yx} = 1 \qquad F''_{yy} = 0\\
\Delta =  F''_{xx}F''_{yy} - F''_{xy}F''_{yx} = -1
\end{gather*}

\item $z = x^2 + y^2$, aко $x - y =1$
\begin{gather*}
F(x,y,\lambda)  = x^2 + y^2 + \lambda(x-y-1)  \quad \lambda \neq 0 \\
F'_x = \lambda + 2x \qquad F'_y = 2y - \lambda \\
\begin{array}{|l@{}}
\lambda + 2x = 0\\
2y - \lambda = 0\\
x-y = 1
\end{array}\Leftrightarrow 
\begin{array}{|l@{}}
x = \frac{1}{2} \\
y = -\frac{1}{2} \\
\lambda = -1
\end{array}
\implies 
A \left( \frac{1}{2}, -\frac{1}{2}, -1\right) \\
F''_{xx} = 2 \qquad F''_{xy} = 0 \qquad F''_{yx} = 0 \qquad F''_{yy} = 2\\
\Delta =  F''_{xx}F''_{yy} - F''_{xy}F''_{yx} = 4 \\
F''_{xx}(A) = 2>0 \qquad \Delta(A) = 4 >0 \implies z_{min} = z \left( \frac{1}{2}, -\frac{1}{2}\right) = \frac{1}{2}
\end{gather*}

\item $u = x^2 + y^2 + z^2$, ако $\frac{x^2}{16} + \frac{y^2}{9} + \frac{z^2}{4} = 1$
\begin{gather*}
F(x,y,z,\lambda) = x^2 + y^2 + z^2 + \lambda\left( \frac{x^2}{16} + \frac{y^2}{9} + \frac{z^2}{4} - 1 \right)   \quad \lambda \neq 0 \\
F'_x = \frac{1}{8}\lambda x + 2x \qquad F'_y = \frac{2}{9}\lambda y+ 2y \qquad F'_z = \frac{1}{2}\lambda z + 2z\\
\begin{array}{|l@{}}
\frac{1}{8}\lambda x + 2x= 0\\
\frac{2}{9}\lambda y+ 2y = 0 \\
\frac{1}{2}\lambda z + 2z = 0\\
\frac{x^2}{16} + \frac{y^2}{9} + \frac{z^2}{4} = 1
\end{array}\Leftrightarrow 
\begin{array}{|l@{}}
x =  4 \land x = -4 \land x = 0\\
y =  0 \land y = 3 \land y = -3\\
z = 0 \land z = 2 \land z = -2 \\
\lambda = -16 \land \lambda = -9 \land \lambda = -4
\end{array} \implies \\ 
A (4,0,0,-16), \, 
B(-4,0,0,-16), \, 
C(0,3,0,-9), \\
D(0,-3,0,-9) \, 
E(0,0,2,-4), \, 
F(0,0,-2,-4) 
\end{gather*}

\begin{gather*}
F''_{xx} = \frac{1}{8} \lambda + 2 \qquad  F''_{yy} = \frac{2}{9} \lambda + 2 \qquad  F''_{zz} = \frac{1}{2} \lambda + 2 \\
F''_{xy} = F''_{xz} = F''_{yx} = F''_{yz} = F''_{zx} = F''_{zy} = 0 \\ 
\Delta_1 = F''_{xx} =  \frac{1}{8} \lambda \\
\Delta_2 = \begin{vmatrix} F''_{xx} & F''_{xy} \\ F''_{yx} & F''_{yy} \end{vmatrix} = \begin{vmatrix} \frac{1}{8} \lambda + 2 & 0 \\  0 & \frac{2}{9} \lambda + 2  \end{vmatrix} = \left( \frac{1}{8} \lambda + 2 \right)\left( \frac{2}{9} \lambda + 2 \right) \\
\Delta_3 = \begin{vmatrix} F''_{xx} & F''_{xy} & F''_{xz}\\ F''_{yx} & F''_{yy} & F''_{yz} \\ F''_{zx} & F''_{zy} & F''_{zz}\end{vmatrix} = \begin{vmatrix} \frac{1}{8} \lambda + 2 & 0 & 0 \\0 & \frac{2}{9} \lambda + 2 & 0 \\ 0 & 0 & \frac{1}{2} \lambda + 2 \end{vmatrix} = \left( \frac{1}{8} \lambda + 2 \right)\left( \frac{2}{9} \lambda + 2 \right) \left( \frac{1}{2} \lambda + 2 \right)
\end{gather*}

\begin{gather*}
\Delta_1(A) = 0 \qquad \Delta_2(A) = 0 \qquad \Delta_3(A) = 0 \implies \text{Няма екстремум.}\\
\Delta_1(B) = 0 \qquad \Delta_2(B) = 0 \qquad \Delta_3(B) = 0 \implies \text{Няма екстремум.}\\
\Delta_1(C) = 0 \qquad \Delta_2(C) = 0 \qquad \Delta_3(C) = 0 \implies \text{Няма екстремум.}\\
\Delta_1(D) = 0 \qquad \Delta_2(D) = 0 \qquad \Delta_3(D) = 0 \implies \text{Няма екстремум.}\\
\Delta_1(E) = 0 \qquad \Delta_2(E) = 0 \qquad \Delta_3(E) = 0 \implies \text{Няма екстремум.}\\
\Delta_1(F) = 0 \qquad \Delta_2(F) = 0 \qquad \Delta_3(F) = 0 \implies \text{Няма екстремум.}\\
\text{Панева кво стааа?!}
\end{gather*}

\item $u = xyz$, ако $x + y + z = 5, xy+yz+zx = 8$ 
\begin{gather*}
F(x,y,z,\lambda, \mu)= xyz +\lambda(x+y+z - 5) + \mu(xy+yz+zx - 8)  \quad \lambda,\mu \neq 0\\
F'_x = \lambda + \mu(y+z) + yz \qquad F'_y = \lambda + \mu(x+z) + xz \qquad F'_z = \lambda + \mu(y+x) + yx\\
\begin{array}{|l@{}}
\lambda + \mu(y+z) + yz = 0\\
\lambda + \mu(x+z) + xz = 0\\
\lambda + \mu(y+x) + yx = 0\\
x + y + z = 5\\
xy+yz+zx = 8
\end{array}\Leftrightarrow 
\begin{array}{|l@{}}
x = 1 \land x = 2 \land x = \frac{4}{3} \land x = \frac{7}{3} \\
y = 2 \land y = 1 \land y = \frac{4}{3} \land y = \frac{7}{3} \\
z = 2 \land z = 1 \land z = \frac{4}{3}\\
\lambda = 4  \land \lambda = \frac{16}{9} \\
\mu = -2 \land \mu = -\frac{4}{3}
\end{array} \implies \\
A \left(1,2,2,4,-2 \right), \, 
B \left(2,2,1,4,-2 \right), \,
C \left(2,1,2,4,-2 \right) \\
D \left(\frac{4}{3},\frac{7}{3},\frac{4}{3},\frac{16}{9},-\frac{4}{3} \right), \,
E \left(\frac{7}{3},\frac{4}{3},\frac{4}{3},\frac{16}{9},-\frac{4}{3} \right), \,
F \left(\frac{4}{3},\frac{4}{3},\frac{7}{3},\frac{16}{9},-\frac{4}{3} \right) \\
\end{gather*}

\begin{gather*}
F''_{xx} = F''_{yy} =  F''_{zz} = 0 \\
F''_{xy} = F''_{yx} = \mu + z \\
F''_{xz} = F''_{zx} = \mu + y \\
F''_{yz} = F''_{zy} = \mu + x \\ 
\Delta_1 = F''_{xx} =  0 \\
\Delta_2 = \begin{vmatrix} F''_{xx} & F''_{xy} \\ F''_{yx} & F''_{yy} \end{vmatrix} = \begin{vmatrix} 0 & \mu + z  \\   \mu + z  & 0 \end{vmatrix} = -(\mu + z)^2 \\
\Delta_3 = \begin{vmatrix} F''_{xx} & F''_{xy} & F''_{xz}\\ F''_{yx} & F''_{yy} & F''_{yz} \\ F''_{zx} & F''_{zy} & F''_{zz}\end{vmatrix} = \begin{vmatrix} 0 &  \mu + z  &  \mu + y  \\ \mu + z  &  0  &  \mu + x  \\  \mu + y  &  \mu + x  & 0 \end{vmatrix} = 2( \mu + z)( \mu + y )( \mu + x )
\end{gather*}

\begin{gather*}
\Delta_1(A) = 0 \qquad \Delta_2(A) = 0 \qquad \Delta_3(A) = 0 \implies \text{Няма екстремум.}\\
\Delta_1(B) = 0 \qquad \Delta_2(B) = -1 \qquad \Delta_3(B) = 0 \implies \text{Няма екстремум.}\\
\Delta_1(C) = 0 \qquad \Delta_2(C) = 0 \qquad \Delta_3(C) = 0 \implies \text{Няма екстремум.}\\
\Delta_1(D) = 0 \qquad \Delta_2(D) = 0 \qquad \Delta_3(D) = 0 \implies \text{Няма екстремум.}\\
\Delta_1(E) = 0 \qquad \Delta_2(E) = 0 \qquad \Delta_3(E) = 0 \implies \text{Няма екстремум.}\\
\Delta_1(F) = 0 \qquad \Delta_2(F) = -1 \qquad \Delta_3(F) = 0 \implies \text{Няма екстремум.}\\
\text{Панева кво стааа?!}
\end{gather*}

\end{itemize}

\subsection*{Задача 11}

\begin{itemize}
\item $u = x^2 + y^2+ z^2 +2x + 4y -6z $, aко $ x^2 + y^2+ z^2 = 14$
\begin{gather*}
F = x^2 + y^2+ z^2 +2x + 4y -6z + \lambda( x^2 + y^2+ z^2 - 14) \quad \lambda \neq 0\\
F'_x = 2\lambda x + 2x + 2 \qquad F'_y = 2\lambda y + 2y + 4 \qquad F'_z = 2\lambda z + 2z -6 \\
\begin{array}{|l@{}}
2\lambda x + 2x + 2 = 0 \\
2\lambda y + 2y + 4  = 0 \\
2\lambda z + 2z -6 = 0 \\
x^2 + y^2+ z^2 = 14
\end{array}\Leftrightarrow 
\begin{array}{|l@{}}
x = -1 \land x = 1 \\
y = -2 \land y = 2 \\
z = 3 \land z = -3 \\
\lambda = 0 \land \lambda = -2 
\end{array} \implies A(-1,-2,3,0), \, B(1,2,-3,-2) 
\end{gather*}

\begin{gather*}
F''_{xx} = F''_{yy} = F''_{zz} = 2\lambda + 2 \\
F''_{xy} = F''_{xz} = F''_{yx} = F''_{yz} = F''_{zx} = F''_{zy} = 0 \\
\Delta_1 = F''_{xx} = 2\lambda + 2 \\
 \Delta_2 = \begin{vmatrix} F''_{xx} & F''_{xy} \\ F''_{yx} & F''_{yy} \end{vmatrix} = \begin{vmatrix} 2\lambda + 2 & 0 \\0 & 2\lambda + 2  \end{vmatrix} = (2\lambda + 2)^2\\
\Delta_3 = \begin{vmatrix} F''_{xx} & F''_{xy} & F''_{xz}\\ F''_{yx} & F''_{yy} & F''_{yz} \\ F''_{zx} & F''_{zy} & F''_{zz}\end{vmatrix} = \begin{vmatrix} 2\lambda + 2 & 0 & 0 \\ 0 & 2\lambda + 2 & 0 \\ 0 & 0 & 2\lambda + 2 \end{vmatrix} = (2\lambda + 2)^3 
\end{gather*}

\begin{gather*}
\Delta_1(A) = 2> 0 \qquad \Delta_2(A) = 4>0 \qquad \Delta_3(A) = 8 >0 \implies \\
u_{min} = u(-1,-2,3) = -14\\
\Delta_1(B) = -2 <0 \qquad \Delta_2(B) = 4>0 \qquad \Delta_3(B) = -8<0 \implies \\
u_{max} = u(1,2,-3) = 42
\end{gather*}

\item $u = x^2 + y^2 + z^2 + 2x + 4y $, ако $x^2 + y^2 = 20$
\begin{gather*}
F = x^2 + y^2 + z^2 + 2x + 4y + \lambda(x^2 + y^2 - 20)\quad \lambda \neq 0\\
F'_x = 2\lambda x + 2x + 2 \qquad F'_y = 2\lambda y + 2y + 4 \qquad F'_z =  2z \\
\begin{array}{|l@{}}
2\lambda x + 2x + 2 = 0 \\
2\lambda y + 2y + 4  = 0 \\
2z = 0 \\
x^2 + y^2= 20
\end{array}\Leftrightarrow 
\begin{array}{|l@{}}
x = 2 \land x = -2 \\
y = 4 \land y = -4 \\
z =0\\
\lambda = -\frac{3}{2} \land \lambda = -\frac{1}{2} 
\end{array} \implies A \left(2,4,0, -\frac{3}{2} \right), \, B \left(-2,-4,0, -\frac{1}{2} \right) 
\end{gather*}

\begin{gather*}
F''_{xx} = F''_{yy} = 2\lambda + 2 \qquad F''_{zz} = 2 \\
F''_{xy} = F''_{xz} = F''_{yx} = F''_{yz} = F''_{zx} = F''_{zy} = 0 \\
\Delta_1 = F''_{xx} = 2\lambda + 2 \\
 \Delta_2 = \begin{vmatrix} F''_{xx} & F''_{xy} \\ F''_{yx} & F''_{yy} \end{vmatrix} = \begin{vmatrix} 2\lambda + 2 & 0 \\0 & 2\lambda + 2  \end{vmatrix} = (2\lambda + 2)^2\\
\Delta_3 = \begin{vmatrix} F''_{xx} & F''_{xy} & F''_{xz}\\ F''_{yx} & F''_{yy} & F''_{yz} \\ F''_{zx} & F''_{zy} & F''_{zz}\end{vmatrix} = \begin{vmatrix} 2\lambda + 2 & 0 & 0 \\ 0 & 2\lambda + 2 & 0 \\ 0 & 0 & 2 \end{vmatrix} = (2\lambda + 2)(4\lambda + 4) 
\end{gather*}

\begin{gather*}
\Delta_1(A) = -1 \qquad \Delta_2(A) = 1>0 \qquad \Delta_3(A) = 2>0 \implies \text{Няма екстремум.}\\
\Delta_1(B) = 1>0 \qquad \Delta_2(B) = 1>0 \qquad \Delta_3(B) = 2>0 \implies u_{min} = u(-2,-4,0) = 0
\end{gather*}

\item $u = x^2 + y^2+ z^2 +6x - 2y + 4z $, ако $x^2 + y^2+ z^2 = 56$ 
\begin{gather*}
F = x^2 + y^2+ z^2 +6x - 2y + 4z + \lambda(x^2 + y^2+ z^2 - 56) \quad \lambda \neq 0\\
F'_x = 2\lambda x + 2x + 6 \qquad F'_y = 2\lambda y + 2y - 2 \qquad F'_z =  2\lambda z + 2z + 4 \\
\begin{array}{|l@{}}
2\lambda x + 2x + 6 = 0 \\
2\lambda y + 2y - 2 = 0 \\
2\lambda z + 2z + 4 = 0\\
x^2 + y^2+ z^2 = 56
\end{array}\Leftrightarrow 
\begin{array}{|l@{}}
x = 6 \land x = -6 \\
y = -2 \land y = 2 \\
z =4 \land z = -4 \\
\lambda = -\frac{3}{2} \land \lambda = -\frac{1}{2} 
\end{array} \implies A \left(6,-2,4, -\frac{3}{2} \right), \, B \left(-6,2,-4, -\frac{1}{2} \right) 
\end{gather*}

\begin{gather*}
F''_{xx} = F''_{yy} = F''_{zz} = 2\lambda + 2 \\
F''_{xy} = F''_{xz} = F''_{yx} = F''_{yz} = F''_{zx} = F''_{zy} = 0 \\
\Delta_1 = F''_{xx} = 2\lambda + 2 \\
 \Delta_2 = \begin{vmatrix} F''_{xx} & F''_{xy} \\ F''_{yx} & F''_{yy} \end{vmatrix} = \begin{vmatrix} 2\lambda + 2 & 0 \\0 & 2\lambda + 2  \end{vmatrix} = (2\lambda + 2)^2\\
\Delta_3 = \begin{vmatrix} F''_{xx} & F''_{xy} & F''_{xz}\\ F''_{yx} & F''_{yy} & F''_{yz} \\ F''_{zx} & F''_{zy} & F''_{zz}\end{vmatrix} = \begin{vmatrix} 2\lambda + 2 & 0 & 0 \\ 0 & 2\lambda + 2 & 0 \\ 0 & 0 & 2\lambda + 2 \end{vmatrix} = (2\lambda + 2)^3
\end{gather*}

\begin{gather*}
\Delta_1(A) = -1 < 0 \qquad \Delta_2(A) = 1>0 \qquad \Delta_3(A) = -1<0 \implies \\
u_{max} = u(6,-2,4) = 112\\
\Delta_1(B) = 1>0 \qquad \Delta_2(B) = 1>0 \qquad \Delta_3(B) = 1>0 \implies \\
u_{min} = u(-6,2,-4) = 0\\
\end{gather*}

\end{itemize}

\subsection*{Задача 12}

\begin{itemize}
\item $u = x^2 + y^2 - 12x + 16y$, aко $ x^2 + y^2 \leq 25, x^2 + y^2 \leq 400, x^2 + y^2 \leq 100$
\item $u = x^2 + y^2+ z^2 +2x + 4y -6 $, ако $x^2 + y^2 + z^2 \leq 9$
\item $u = x^2 + 2y^2+ 3z^2$, ако $x^2 + y^2+ z^2 \leq 100$ 
\end{itemize}


\newpage 
\section{Упражнение към лекция 9}

\subsection{Задачи}

\subsection*{Задача 1}
Да се пресметнат интегралите
\begin{itemize}
\item $I = \iint\limits_D xy \,dx \ dy, \qquad D: \begin{cases} 0 \leq x \leq 1 \\ 0 \leq y \leq 2 \end{cases}$
\item $I = \iint\limits _D xy \,dx \ dy$, ако D e oградена от $\begin{cases} xy = 1 \\ x+y = \frac{5}{2} \end{cases}$
\item $I = \iint\limits _D \,dx \ dy$, ако D e оградена от кривите $D: \begin{cases} 4y = x^2 - 4x \\ x - y - 3 = 0 \end{cases}$
\end{itemize}

\subsection*{Задача 2}
Да се определят границите на интегриране и да се пресметне интеграла
$$\iint\limits_D (x^2 + y^2) \, dx \ dy, \qquad D: \begin{cases} y = x , \quad y =2 \\ y = x + 2, \quad y = 6 \end{cases}$$

\subsection*{Задача 3}
Да се пресметне интеграла
$$\iint\limits_D (x+y) \, dx \ dy, \qquad \partial D: \begin{cases} y^2 = 2x \\ x+y = 4 \\ x+y = 12 \end{cases}$$

\subsection*{Задача 4}
Да се пресметне интеграла
$$\iint\limits_D (x+y) \, dx \ dy$$
Където D е триъгълник $\triangle ABO$ с върхове $A(1,0),\, B = (1,1),\, O = (0,0)$

\newpage
\subsection{Решения}

\subsection*{Задача 1}
\begin{itemize}
\item $I = \iint\limits_D xy \,dx \ dy, \qquad D: \begin{cases} 0 \leq x \leq 1 \\ 0 \leq y \leq 2 \end{cases}$
\begin{gather*}
I = \int\limits_0 ^1 x \int\limits_0 ^2 y \, dy \ dx   \
\int\limits_0 ^2 y \, dy = \frac{y^2}{2} \Big|_0 ^2 = 2 \\
I = 2 \int\limits_0 ^1 x \, dx = 2\frac{x^2}{2} \Big|_0 ^1 = 1
\end{gather*}

\item $I = \iint\limits _D xy \,dx \ dy$, ако D e oградена от $\begin{cases} xy = 1 \\ x+y = \frac{5}{2} \end{cases}$
\begin{gather*}
\begin{array}{|l@{}}
xy = 1 \\
x+y = \frac{5}{2}
\end{array} \Leftrightarrow
\begin{array}{|l@{}}
y = \frac{5}{2} - x \\
x \left( \frac{5}{2} - x \right) = 1 
\end{array}\implies 
D: \begin{cases} 
\frac{1}{2} \leq x \leq 2 \\
\frac{1}{x} \leq y \leq \frac{5}{2} - x
\end{cases}\\
I = \int\limits_{\frac{1}{2}} ^2 x \int\limits_{\frac{1}{x}} ^{\frac{5}{2} - x} y \, dy \ dx \\
\int\limits_{\frac{1}{x}} ^{\frac{5}{2} - x} y \, dy  = 
\frac{y^2}{2} \Big|_{\frac{1}{x}} ^{\frac{5}{2} - x} = 
\frac{1}{2}\left[ \left( \frac{5}{2} - x\right)^2 - \frac{1}{x^2} \right] \\
I = \frac{1}{2} \int\limits_{\frac{1}{2}} ^2 x \left[ \left( \frac{5}{2} - x\right)^2 - \frac{1}{x^2} \right] \, dx = \frac{1}{2} \int\limits_{\frac{1}{2}} ^2 x \left( \frac{25}{4} - 5x + x^2 - \frac{1}{x^2} \right) \, dx = \\
= \frac{1}{2} \int\limits_{\frac{1}{2}} ^2  \left( \frac{25}{4}x - 5x^2 + x^3 - \frac{1}{x} \right) \, dx = 
\frac{1}{2} \left[ \frac{x^4}{4} - \frac{5x^3}{3} + \frac{25x^2}{8} - \ln|x| \right] \Big|_{\frac{1}{2}} ^2 = \frac{165}{128} - \ln(2)
\end{gather*}

\item $I = \iint\limits _D \,dx \ dy$, ако D e оградена от кривите $D: \begin{cases} 4y = x^2 - 4x \\ x - y - 3 = 0 \end{cases}$
\begin{gather*}
\begin{array}{|l@{}}
4y = x^2 - 4x \\ 
x - y - 3 = 0
\end{array} \Leftrightarrow
\begin{array}{|l@{}}
y = x - 3 \\
4(x-3) = x^2 - 4x
\end{array}\implies D: \begin{cases} 
2 \leq x \leq 6 \\
\frac{x^2}{4}- x \leq y \leq x - 3 
\end{cases}\\
I = \int\limits_2 ^6 \int\limits_{\frac{x^2}{4} - x} ^{x - 3} \, dy \ dx  \\
\int\limits_{\frac{x^2}{4} - x} ^{x - 3} \, dy  =
 y \Big|_{\frac{x^2}{4} - x} ^{x - 3} = 
\left[ x- 3 - \left( \frac{x^2}{4} - x\right) \right] \\
I = \int\limits_2 ^6 \left[ x- 3 - \left( \frac{x^2}{4} - x\right) \right] \, dx = 
\int\limits_2 ^6 \left[ 2x- 3 - \frac{x^2}{4} \right] \, dx = 
\left[ x^2- 3x - \frac{x^3}{12} \right] \Big|_2 ^6 = \frac{8}{3}
\end{gather*}

\end{itemize}
\subsection*{Задача 2}
$$I = \iint\limits_D (x^2 + y^2) \, dx \ dy, \qquad D: \begin{cases} y = x , \quad y =2 \\ y = x + 2, \quad y = 6 \end{cases}$$
\begin{gather*}
D_1: \begin{cases} 
0\leq x \leq 2 \\
2 \leq y \leq x +2
\end{cases} \qquad 
D_2: \begin{cases} 
2\leq x \leq 4 \\
x \leq y \leq x +2
\end{cases} \qquad 
D_3: \begin{cases} 
4\leq x \leq 6 \\
x \leq y \leq 6
\end{cases} \\
I_1 = \int\limits_0 ^2 x^2 \int\limits_{2} ^{x+2} y^2\, dy \ dx  \qquad
I_2 = \int\limits_2 ^4 x^2 \int\limits_{x} ^{x+2} y^2 \, dy\ dx  \qquad  
I_3 = \int\limits_4 ^6 x^2 \int\limits_{x} ^{6} y^2\, dy \ dx 
\end{gather*}

\begin{gather*}
I_1 = \int\limits_0 ^2 x^2 \int\limits_{2} ^{x+2} y^2\, dy \ dx\\
\int\limits_{2} ^{x+2} y^2\, dy  \frac{y^3}{3} \Big|_{2} ^{x+2} = \frac{(x+2)^3}{3} - \frac{8}{3}\\
I_1 = \int\limits_0 ^2 x^2 \left(\frac{(x+2)^3}{3} - \frac{8}{3} \right) =  
\int\limits_0 ^2 \left(\frac{x^2(x+2)^3}{3} - \frac{8x^2}{3} \right) = \frac{56}{3} 
\end{gather*}

\begin{gather*}
I_2 = \int\limits_2 ^4 x^2 \int\limits_{x} ^{x+2} y^2 \, dy\ dx \\
\int\limits_{x} ^{x+2} y^2\, dy  \frac{y^3}{3} \Big|_{x} ^{x+2} = \frac{(x+2)^3}{3} - \frac{x^3}{3}\\
I_2 = \int\limits_2 ^4 x^2 \left(\frac{(x+2)^3}{3} - \frac{x^3}{3} \right) =  
\int\limits_0 ^2 \left(\frac{x^2(x+2)^3}{3} - \frac{x^5}{3} \right) = 104 
\end{gather*}

\begin{gather*}
I_3 = \int\limits_4 ^6 x^2 \int\limits_{x} ^{6} y^2\, dy \ dx \\
\int\limits_{x} ^{6} y^2\, dy = \frac{y^3}{3} \Big|_{x} ^{6} = 72 - \frac{x^3}{3} \\
I_3 = \int\limits_4 ^6 x^2 \left( 72 - \frac{x^3}{3}\right) \, dx =  
\int\limits_4 ^6 \left( 72x^2 - \frac{x^5}{3}\right) \, dx = \frac{304}{3} \\
I = I_1 + I_2 + I_3 = \frac{56}{3} + 104 + \frac{304}{3} = 224
\end{gather*}

\subsection*{Задача 3}
$$\iint\limits_D (x+y) \, dx \ dy, \qquad \partial D: \begin{cases} y^2 = 2x \\ x+y = 4 \\ x+y = 12 \end{cases}$$
\begin{gather*}
\begin{array}{|l@{}}
y^2 = 2x \\
x+y = 4
\end{array} \Leftrightarrow
\begin{array}{|l@{}}
y = 4 - x \\
(4-x)^2 = 2x
\end{array} \implies D_1 =  \begin{cases} 2 \leq x \leq 8 \\ 4-x \leq y \leq \sqrt{2x} \end{cases} \\
\begin{array}{|l@{}}
y^2 = 2x \\
x+y = 12
\end{array} \Leftrightarrow
\begin{array}{|l@{}}
y = 12 - x \\
(12-x)^2 = 2x
\end{array} \implies D_2 =  \begin{cases} 8 \leq x \leq 18 \\ -\sqrt{2x} \leq y \leq 12-x \end{cases} \\
I_1 = \int\limits_2 ^8 \int\limits_{4-x} ^{\sqrt{2x}} (x+y)\, dy \ dx \qquad 
I_2 = \int\limits_8 ^{18} \int\limits_{-\sqrt{2x}} ^{12-x} (x+y)\, dy \ dx
\end{gather*}

\begin{gather*}
I_1 = \int\limits_2 ^8 \int\limits_{4-x} ^{\sqrt{2x}} (x+y)\, dy \ dx   \\
\int\limits_{4-x} ^{\sqrt{2x}} (x+y)\, dy  = 
xy + \frac{y^2}{2} \Big|_{4-x} ^{\sqrt{2x}} = 
\sqrt{2} x^{\frac{3}{2}} + \frac{x^2}{2} + x - 8 \\
I_1 = \int\limits_2 ^8  \sqrt{2} x^{\frac{3}{2}} + \frac{x^2}{2} + x - 8 \, dx  = \frac{826}{5} \\
I_2 = \int\limits_8 ^{18} \int\limits_{-\sqrt{2x}} ^{12-x} (x+y)\, dy \ dx \\
\int\limits_{-\sqrt{2x}} ^{12-x} (x+y)\, dy  = 
xy + \frac{y^2}{2} \Big|_{-\sqrt{2x}} ^{12-x} =
\sqrt{2} x^{\frac{3}{2}} - \frac{x^2}{2} - x + 72 \\
I_2 = \int\limits_8 ^{18} \sqrt{2} x^{\frac{3}{2}} - \frac{x^2}{2} - x + 72 \,dx = \frac{5678}{15}\\
I = I_1 + I_2 = \frac{8156}{15}
\end{gather*}

\subsection*{Задача 4}
$$\iint\limits_D (x+y) \, dx \ dy$$
Където D е триъгълник $\triangle ABO$ с върхове $A(1,0),\, B = (1,1),\, O = (0,0)$

\begin{gather*}
D = \begin{cases} 0 \leq x \leq 1 \\ 0\leq y \leq x \end{cases} \\
I = \int\limits_0 ^1 \int\limits_{0} ^{x} (x+y)\, dy \ dx\\
\int\limits_{0} ^{x} (x+y)\, dy =
 xy + \frac{y^2}{2} \Big|_{0} ^{x} = \frac{3x^2}{2} \\
I = \frac{3}{2}  \int\limits_0 ^1 x^2 \,dx = 
\frac{x^3}{2}\Big|_0 ^1 = \frac{1}{2} - 0 = \frac{1}{2}
\end{gather*}

\newpage 
\section{Упражнение към лекция 10}

\subsection{Задачи}

\newpage
\subsection{Решения}

\newpage 
\section{Упражнение към лекция 11}

\subsection{Задачи}

\subsection*{Задача 1}
Да се пресметнат интегралите 
\begin{itemize}
\item $I = \iint\limits_D \arctan \frac{y}{x} \, dx \ dy, \qquad 
D = \begin{cases} 1 \leq x^2 + y^2 \leq 9 \\ y \geq \frac{x}{\sqrt{3}} \qquad y \leq \sqrt{3}x \end{cases} $
\item $I = \iint\limits_D \, dx \ dy, \qquad
 D = \{ (x,y) : x^2 +y^2 \leq 25 \}$ 
\item $I = \iint\limits_D \, dx \ dy, \qquad$ 
D е оградена от лемниската $(x^2 + y^2)^2 = 2a^2 xy$ 
\item $I = \iint\limits_D\sqrt{1 + \frac{x^2}{9} + \frac{y^2}{16}} \, dx \ dy , \qquad
D: \frac{x^2}{9} + \frac{y^2}{16} \leq 1$
\end{itemize}

\subsection*{Задача 2}
Да се пресметнат интегралите 
\begin{itemize}
\item $I = \iint\limits_D \, dx \ dy, \qquad 
\partial D: \begin{cases} y^2 = 2x, & xy=1 \\ y^2 = 3x, & xy=2 \end{cases}$
\item $I = \iint\limits_D (2x - y)\, dx \ dy, \qquad 
\partial D: \begin{cases} x+y = 1, & x+y=2 \\ 2x-y = 1, & 2x-y = 3 \end{cases}$
\end{itemize}

\newpage
\subsection{Решения}

\subsection*{Задача 1}
\begin{itemize}
\item $I = \iint\limits_D \arctan \frac{y}{x} \, dx \ dy, \qquad 
D = \begin{cases} 1 \leq x^2 + y^2 \leq 9 \\ y \geq \frac{x}{\sqrt{3}} \qquad y \leq \sqrt{3}x \end{cases} $
\begin{gather*}
\text{Полярна смяна} \\
x = \rho \cos \varphi \qquad y = \rho \sin \varphi \qquad (\rho > 0, \qquad 0< \varphi < 2\pi ) \\
\text{За допълнителните условия имаме} \\
1 \leq x^2 + y^2  = \rho^2 \cos^2 \varphi +\rho^2 \sin^2 \varphi \leq 9 
\implies  1 \leq \rho^2 \leq 9 \implies
 1 \leq \rho \leq 3 \\
y = \frac{x}{\sqrt{3}} \text{ е права сключваща ъгъл $\frac{\pi}{6}$ с положителната посока на $Ox$, a }\\
y = \sqrt{3}x - \frac{\pi}{3} \implies \\
\tan \varphi \geq \frac{1}{\sqrt{3}} \qquad \tan \varphi \leq \sqrt{3} \\
\varphi \in \left[\frac{\pi}{6} , \frac{\pi}{2}\right) \qquad \varphi \in \left(- \frac{\pi}{2} ,\frac{\pi}{3}\right] \implies
\varphi \in \left[\frac{\pi}{6}, \frac{\pi}{3} \right] \\
\arctan \frac{y}{x} =  
\arctan \frac{\rho \sin \varphi}{\rho \cos \varphi} = 
\arctan \tan \varphi = \varphi \\
\Delta = \rho \implies dxdy = \rho d\rho d\varphi\\
I = \int\limits_{\frac{\pi}{6}} ^{\frac{\pi}{3}} \int\limits_1 ^3 \varphi \cdot \rho \, d\rho \ d\varphi = \int\limits_{\frac{\pi}{6}} ^{\frac{\pi}{3}} \varphi \int\limits_1 ^3 \rho \, d\rho \ d\varphi \\
\int\limits_1 ^3 \rho \, d\rho = \frac{\rho^2}{2} \Big|_1 ^3 = \frac{1}{2} (9-1) = 4 \\
I = 4 \int\limits_{\frac{\pi}{6}} ^{\frac{\pi}{3}} \varphi = 
4\frac{\varphi^2}{2} \Big|_{\frac{\pi}{6}} ^{\frac{\pi}{3}} = 
2 \left(\frac{\pi^2}{9} - \frac{\pi^2}{36}\right) = 2\frac{3\pi^2}{36} = \frac{\pi^2}{6}
\end{gather*}

\item $I = \iint\limits_D \, dx \ dy, \qquad
 D = \{ (x,y) : x^2 +y^2 \leq 25 \}$ 
\begin{gather*}
\text{Полярна смяна} \\
x = \rho \cos \varphi \qquad y = \rho \sin \varphi \qquad (\rho > 0, \qquad 0< \varphi < 2\pi ) \\
\text{За допълнителните условия имаме} \\
x^2 + y^2  = \rho^2 \cos^2 \varphi +\rho^2 \sin^2 \varphi \leq 25 \implies 
0 \leq \rho^2 \leq 25 \implies
 0 \leq \rho \leq 5 \\
\text{За $\varphi$ не се появяват допълнителни ограничения} \\
\text{Якобианът} = \Delta = \rho \\
I = \int\limits_0 ^{2\pi} \int\limits_0 ^5 \rho \, d\rho \ d\varphi\\
\int\limits_0 ^5 \rho \, d \rho = 
\frac{\rho^2}{2} \Big|_0 ^5 = \frac{25}{2} \\
I = \frac{25}{2} \int\limits_0 ^{2\pi} \, d\varphi = 
\frac{25}{2} \cdot 2\pi = 25\pi 
\end{gather*}

\item $I = \iint\limits_D \, dx \ dy, \qquad$ 
D е оградена от лемниската $(x^2 + y^2)^2 = 2a^2 xy$ 
\begin{gather*}
\text{Полярна смяна} \\
x = \rho \cos \varphi \qquad y = \rho \sin \varphi \qquad (\rho > 0, \qquad 0< \varphi < 2\pi ) \\
xy>0 \implies \cos \varphi > 0 \qquad \sin \varphi > 0 \\
\varphi \in \left(0, \frac{\pi}{2} \right) \qquad  \varphi \in \left(\frac{\pi}{2}, \frac{3\pi}{2} \right)
%0 \leq \rho \leq a\sqrt{\sin 2 \varphi}
\end{gather*}

\item $I = \iint\limits_D\sqrt{1 + \frac{x^2}{9} + \frac{y^2}{16}} \, dx \ dy , \qquad
D: \frac{x^2}{9} + \frac{y^2}{16} \leq 1$
\begin{gather*}
\text{Полярна смяна} \\
x = 3\rho \cos \varphi \qquad y = 4\rho \sin \varphi \qquad (\rho > 0, \qquad 0< \varphi < 2\pi )\\
\text{Якобианът на смяната:} \Delta = 12\rho \\
\sqrt{1 + \frac{x^2}{9} + \frac{y^2}{16}} = 
\sqrt{1 + \frac{9\rho^2 \cos^2 \varphi}{9} + \frac{16\rho^2 \sin^2 \varphi}{16}} = 
\sqrt{1 +\rho^2 (\cos^2 \varphi+\sin^2 \varphi)} \implies \\
\sqrt{1 + \frac{x^2}{9} + \frac{y^2}{16}} = \sqrt{1 +\rho^2} \\
\frac{9\rho^2 \cos^2 \varphi}{9} + \frac{16\rho^2 \sin^2 \varphi}{16} \leq 1 \implies \rho^2 \leq 1 \implies 0 \leq \rho \leq 1 \\
I = 12 \int\limits_0 ^{2\pi}  \int\limits_0 ^{1} \sqrt{1 +\rho^2} \, d\rho \ d\varphi\\
\int\limits_0 ^{1} \sqrt{1 +\rho^2} \, d\rho = 
\frac{1}{2}\int\limits_0 ^{1} (1 +\rho^2)^{\frac{1}{2}} \, d (\rho^2 + 1) =
\frac{1}{2} \cdot \frac{(\rho^2+1)^{\frac{3}{2}}}{\frac{3}{2}} \Big|_0 ^{1} =
\frac{1}{2} \cdot \frac{2^{\frac{3}{2}} - 1 }{\frac{3}{2}} \\
I = \frac{12}{2} \cdot \frac{2}{3} \int\limits_0 ^{2\pi} 2^{\frac{3}{2}} - 1 \, d\varphi = 
\frac{12}{3} (2\sqrt{2} - 1) \cdot 2\pi = \frac{8\pi}{3} (2\sqrt{2} - 1)
\end{gather*}
\end{itemize}

\subsection*{Задача 2}
\begin{itemize}
\item $I = \iint\limits_D \, dx \ dy, \qquad 
\partial D: \begin{cases} y^2 = 2x, & xy=1 \\ y^2 = 3x, & xy=2 \end{cases}$
\begin{gather*}
\text{Нека} \\
u = \frac{y^2}{x} \qquad v = xy \\
2 \leq u \leq 3 \qquad 1 \leq v \leq 2 \\
\text{Якобианът на смяната е}:
\Delta = \frac{D(x,y)}{D(u,v)} = \frac{1}{\frac{D(u,v)}{D(x,y)}}\\
\text{Освен това}
\frac{D(u,v)}{\partial (x,y)} = \begin{vmatrix} u'_x & u'_y \\ v'_x & v'_y \end{vmatrix} =  \begin{vmatrix} - \frac{y^2}{x^2} &\frac{2y}{x}\\ y & x \end{vmatrix} = - \frac{y^2}{x} - \frac{2y^2}{x} =  - \frac{3y^2}{x} = 3u \\
\Delta = - \frac{1}{3u} \implies |\Delta| = \frac{1}{3u} 
\end{gather*}
\begin{gather*}
I = \int\limits_1 ^2 \int\limits_2 ^3 \frac{1}{3u} \, du \ dv \\
\int\limits_2 ^3 \frac{1}{3u} \, du  = 
\frac{1}{3} \int\limits_2 ^3 \frac{1}{u} = 
\frac{1}{3} \ln u \Big|_2 ^3  = \frac{1}{3} \ln \frac{3}{2} \\
I = \int\limits_1 ^2 \frac{1}{3} \ln \frac{3}{2} \, dv = 
\frac{1}{3} \ln \frac{3}{2} \int\limits_1 ^2 \, dv  = 
\frac{1}{3} \ln \frac{3}{2} v\Big|_1 ^2 = 
\frac{1}{3} \ln \frac{3}{2}
\end{gather*}
\item $I = \iint\limits_D (2x - y)\, dx \ dy, \qquad 
\partial D: \begin{cases} x+y = 1, & x+y=2 \\ 2x-y = 1, & 2x-y = 3 \end{cases}$
\begin{gather*}
\text{Нека} \\
u = x+y \qquad v = 2x-y \\
1 \leq u \leq 2 \qquad 1 \leq v \leq 3 \\
\text{Якобианът на смяната е}:
\Delta = \frac{D(x,y)}{D(u,v)} = \frac{1}{\frac{D(u,v)}{D(x,y)}}\\
\text{Освен това}
\frac{D(u,v)}{\partial (x,y)} = \begin{vmatrix} u'_x & u'_y \\ v'_x & v'_y \end{vmatrix} =  \begin{vmatrix} 1 & 1\\ 2 & -1 \end{vmatrix} =-3 \\
\Delta = - \frac{1}{3} \implies |\Delta| = \frac{1}{3} 
\end{gather*}
\begin{gather*}
I = \int\limits_1 ^2 \int\limits_1 ^3 \frac{1}{3}v \, dv \ du \\
\int\limits_1 ^3 \frac{1}{3}v \, dv  =
\frac{1}{3} \int\limits_1 ^3 v \, dv =
\frac{1}{3} \frac{v^2}{2} \Big|_1 ^3 = 
\frac{1}{3} \left( \frac{9}{2} - \frac{1}{2}\right) = \frac{4}{3} \\
I = \int\limits_1 ^2  \frac{4}{3} \, du = 
\frac{4}{3} \int\limits_1 ^2  \, du = 
\frac{4}{3} u\Big|_1 ^2 = \frac{4}{3}
\end{gather*}
\end{itemize}

\newpage 
\section{Упражнение към лекция 12}

\subsection{Задачи}

\subsection*{Задача 1}
Пресметнете интегралите 
\begin{itemize}
\item $\iint\limits_{T} (x^2 + y^2 ) \ dx \ dy \ dz, 
T: \begin{cases} x^2 + y^2 = 2z \\ z = 2 \end{cases}$
\item $\iint\limits_{T} (x^2 + y^2 ) \ dx \ dy \ dz, 
T: \begin{cases} z= 6-x^2 - y^2 \\ z^2 = x^2 + y^2 \\ z \geq 0 \end{cases}$
\end{itemize}

\subsection*{Задача 2}
Да се пресметне обемът на тялото T
\begin{itemize}
\item $T: x^2 + y^2 + z^2 \leq R^2$
\item $T: \frac{x^2}{a^2} + \frac{y^2}{b^2} +  \frac{z^2}{c^2} \leq 1$
\end{itemize}
$$R>0 \qquad a>0 \qquad b>0 \qquad c > 0 \qquad  a,b,c,R = const$$

\subsection*{Задача 3}
Да се намери лицето на частта от повърхнината 
$$S: x^2 + y^2 = 2az, \qquad a>0$$
заключена в цилиндъра 
$$(x^2 + y^2)^2 = 2a^2xy$$

\newpage
\subsection{Решения}
\subsection*{Задача 1}
\begin{itemize}
\item $\iint\limits_{T} (x^2 + y^2 ) \ dx \ dy \ dz, 
T: \begin{cases} x^2 + y^2 = 2z \\ z = 2 \end{cases}$ \\
\begin{gather*}
D: \begin{cases} x^2 + y^2 = 2z \\ z = 2 \end{cases} \implies D: x^2 +y^2 \leq 4 \\
\text{Цилиндрична смяна} \\
x = \rho \cos \varphi \qquad y = \rho \sin \varphi \qquad z = z  \implies |\Delta| = \rho \\
\rho^2 \cos^2 \varphi + \rho^2 \sin^2 \varphi \leq 4 \implies \rho^2 \leq 4 \implies 0 \leq \rho \leq 2 \\
\frac{1}{2} (x^2 + y^2) \leq z \leq 2 \implies \frac{\rho^2}{2} \leq z \leq 2 \\
I = \int\limits_0 ^{2\pi} \int\limits_0 ^2 \int\limits_{\frac{\rho^2}{2}} ^2 \rho^2 \cdot \rho \ dz \ d \rho \ d \varphi =
\int\limits_0 ^{2\pi} \int\limits_0 ^2 \int\limits_{\frac{\rho^2}{2}} ^2 \rho^3 \ dz \ d \rho \ d \varphi
\end{gather*}
\begin{gather*}
\int\limits_{\frac{\rho^2}{2}} ^2 \rho^3 \ dz = 
\rho^3 \int\limits_{\frac{\rho^2}{2}} ^2 \rho^3 \ dz = 
\rho^3 \cdot z \Big|_{\frac{\rho^2}{2}} ^2 = 
\rho^3  \cdot \left( 2 - \frac{\rho^2}{2} \right) = 2\rho^3 - \frac{\rho^5}{2} \\
I = \int\limits_0 ^{2\pi} \int\limits_0 ^2 2\rho^3 - \frac{\rho^5}{2} \ d \rho \ d \varphi 
\end{gather*}
\begin{gather*}
\int\limits_0 ^2 2\rho^3 - \frac{\rho^5}{2} \ d \rho = 
\int\limits_0 ^2 2\rho^3 \ d \rho - \int\limits_0 ^2 \frac{\rho^5}{2} \ d \rho = \\
\frac{2\rho^4}{4} \Big|_0 ^2 - \frac{\rho^6}{12} \Big|_0 ^2 = 
\frac{2 \cdot 16}{4} - \frac{64}{12} =  8 - \frac{16}{3} = \frac{24 - 16}{3} = \frac{8}{3} \\
I = \int\limits_0 ^{2\pi}  \frac{8}{3} \ d \varphi  =  
\frac{8}{3} \int\limits_0 ^{2\pi} \ d \varphi = 
\frac{8}{3} \varphi \Big|_0 ^{2\pi} = \frac{16 \pi}{3}
\end{gather*}

\item $\iint\limits_{T} (x^2 + y^2 ) \ dx \ dy \ dz, 
T: \begin{cases} z= 6-x^2 - y^2 \\ z^2 = x^2 + y^2 \\ z \geq 0 \end{cases}$\\
\begin{gather*}
\text{Цилиндрична смяна} \\
x = \rho \cos \varphi \qquad y = \rho \sin \varphi \qquad z = z  \implies |\Delta| = \rho \\
\begin{array}{|l@{}}
z = 6-x^2-y^2 \\
z = \sqrt{x^2+y^2} 
\end{array} \implies 
6-x^2-y^2 = \sqrt{x^2+y^2} \\
\implies 6 - \rho^2 = \rho \implies \rho = -3, \rho > 0 , \rho = 2 \implies \\
\begin{cases} 0 \leq \rho \leq 2 \\ 0 \leq \varphi \leq 2\pi \end{cases} \\
I = \int\limits_0 ^{2\pi} \int\limits_0 ^2 \int\limits_{\rho} ^{6 - \rho^2} \rho^3  \ dz \ d\rho \ d\varphi \\
\int\limits_\rho ^{6 - \rho^2} \rho^3  \ dz = 6\rho^3 - \rho^5 - \rho^4 \\
I = \int\limits_0 ^{2\pi} \int\limits_0 ^2 6\rho^3 - \rho^5 - \rho^4 \ d\rho \ d\varphi \\
\int\limits_0 ^2 6\rho^3 - \rho^5 - \rho^4 \ d\rho  = 
\frac{6\rho^4}{4}\Big|_0 ^2 - \frac{\rho^6}{6}\Big|_0 ^2 - \frac{\rho^5}{5}\Big|_0 ^2 = 
\frac{104}{15} \\
I = \int\limits_0 ^{2\pi} \frac{104}{15} \ d\varphi =  \frac{104}{15} \int\limits_0 ^{2\pi} \ d\varphi = \frac{208\pi}{15}
\end{gather*}
\end{itemize}

\newpage
\subsection*{Задача 2}
$$R>0 \qquad a>0 \qquad b>0 \qquad c > 0 \qquad  a,b,c,R = const$$
$$V_T = \iiint\limits_{T} \ dx \ dy \ dz $$
\begin{itemize}
\item $T: x^2 + y^2 + z^2 \leq R^2$\\
\begin{gather*}
\text{Сферична смяна} \\
x = \rho \cos \varphi \sin \theta \qquad 
y = \rho \sin \varphi \sin \theta \qquad  
z = \rho \cos \theta \qquad 
|\Delta| = \rho^2 \sin \theta \\
\rho^2 \cos^2 \varphi \sin^2 \theta + \rho^2 \sin^2 \varphi \sin^2 \theta + \rho^2 \cos^2 \theta \leq R^2 \\
\rho^2 \sin^2 \theta (\cos^2 \varphi + \sin^2 \varphi) + \rho^2 \cos^2 \theta \leq R^2 \\
\rho^2 \sin^2 \theta + \rho^2 \cos^2 \theta \leq R^2 \implies \rho^2 \leq R^2 \implies 0 \leq \rho \leq R \\
V_T = \int\limits_0 ^{2\pi} \int\limits_0 ^{R} \int\limits_0 ^{\pi} \rho^2 \sin \theta \ d \theta \ d \rho \ d \varphi \\
\int\limits_0 ^{\pi} \rho^2 \sin \theta \ d \theta = 
\rho^2 \int\limits_0 ^{\pi} \sin \theta \ d \theta = 
\rho^2 (- \cos \theta)\Big|_0 ^{\pi} = 2 \rho^2 \\
V_T =  \int\limits_0 ^{2\pi} \int\limits_0 ^{R}  2 \rho^2 \ d \rho \ d \varphi =
2 \int\limits_0 ^{2\pi} \int\limits_0 ^{R} \rho^2 \ d \rho \ d \varphi \\
\int\limits_0 ^{R} \rho^2 \ d \rho = \frac{\rho^3}{3} \Big|_0 ^{R} = \frac{R^3}{3} \\
2 \int\limits_0 ^{2\pi}  \frac{R^3}{3} \ d \varphi = 
\frac{2R^3}{3} \int\limits_0 ^{2\pi} \ d \varphi = \frac{2R^3}{3} \varphi \Big|_0 ^{2\pi} = \frac{4\pi R^3}{3}
\end{gather*}

\item $T: \frac{x^2}{a^2} + \frac{y^2}{b^2} +  \frac{z^2}{c^2} \leq 1$\\
\begin{gather*}
\text{Сферична смяна} \\
x = a\rho \cos \varphi \sin \theta \qquad
 y = b\rho \sin \varphi \sin \theta \qquad  
z = c\rho \cos \theta \qquad 
|\Delta| = abc\rho^2\sin\theta \\
\frac{a^2\rho^2 \cos^2 \varphi \sin^2 \theta}{a^2} + \frac{b^2\rho^2 \sin^2 \varphi \sin^2 \theta }{b^2} + \frac{c^2 \rho^2 \cos^2 \theta}{c^2} \leq 1 \\
\rho^2 \cos^2 \varphi \sin^2 \theta + \rho^2 \sin^2 \varphi \sin^2 \theta + \rho^2 \cos^2 \theta \leq  1 \\
\rho^2 \sin^2 \theta (\cos^2 \varphi + \sin^2 \varphi) + \rho^2 \cos^2 \theta \leq 1 \\
\rho^2 \sin^2 \theta + \rho^2 \cos^2 \theta \leq R^2 \implies \rho^2 \leq 1 \implies 0 \leq \rho \leq 1 \\
V_T = \int\limits_0 ^{2\pi} \int\limits_0 ^{1} \int\limits_0 ^{\pi} abc\rho^2 \sin \theta \ d \theta \ d \rho \ d \varphi = 
abc \int\limits_0 ^{2\pi} \int\limits_0 ^{1} \int\limits_0 ^{\pi} \rho^2 \sin \theta \ d \theta \ d \rho \ d \varphi \\
\int\limits_0 ^{\pi} \rho^2 \sin \theta \ d \theta = 
\rho^2 \int\limits_0 ^{\pi} \sin \theta \ d \theta = 
\rho^2 (- \cos \theta)\Big|_0 ^{\pi} = 2 \rho^2 \\
V_T = abc \int\limits_0 ^{2\pi} \int\limits_0 ^{1}  2 \rho^2 \ d \rho \ d \varphi =
2 abc \int\limits_0 ^{2\pi} \int\limits_0 ^{1} \rho^2 \ d \rho \ d \varphi \\
\int\limits_0 ^{1} \rho^2 \ d \rho = \frac{\rho^3}{3} \Big|_0 ^{1} = \frac{1}{3} \\
2 abc \int\limits_0 ^{2\pi}  \frac{1}{3} \ d \varphi = 
\frac{2abc}{3} \int\limits_0 ^{2\pi} \ d \varphi = \frac{2abc}{3} \varphi \Big|_0 ^{2\pi} = \frac{4}{3} \pi abc 
\end{gather*}
\end{itemize}

\subsection*{Задача 3}
$$S: x^2 + y^2 = 2az, \qquad a>0$$
$$(x^2 + y^2)^2 = 2a^2xy$$
\begin{gather*}
\sigma = \iint\limits_{D} \sqrt{1 + (z'_x)^2 + (z'_y)^2} \ dx \ dy \\
D: (x^2 + y^2)^2 \leq 2a^2xy \\
S: z = \frac{1}{2a^2} (x^2 + y^2) \implies z'_x = \frac{x}{a}, z'_y = \frac{y}{a} \\
1 + (z'_x)^2 + (z'_y)^2 = \frac{a^2 + x^2 + y^2}{a^2} \implies
\sigma = \frac{4}{a} \iint\limits_{D} \sqrt{a^2 + x^2 + y^2} \ dx \ dy 
\end{gather*}
\begin{gather*}
\text{Полярна смяна} \\
x = \rho \cos \varphi \qquad y = \rho \sin \varphi \implies \\
0 \leq \varphi \leq \frac{\pi}{4} \qquad 0 \leq \rho \leq a\sqrt{2\sin\varphi \cos\varphi} = a\sqrt{\sin 2\varphi}\\
\sigma = \frac{4}{a} \int\limits_0 ^{\frac{\pi}{4}} \int\limits_0 ^{a\sqrt{\sin 2\varphi}} \sqrt{a^2 + \rho^2} \rho \ d\rho \ d \varphi  = \frac{4}{a2} \int\limits_0 ^{\frac{\pi}{4}} \int\limits_0 ^{a\sqrt{\sin 2\varphi}} (a^2 + \rho^2)^\frac{1}{2} \ d\rho^2  \ d \varphi  = \\
\frac{2}{a} \int\limits_0 ^{\frac{\pi}{4}} \frac{(a^2 + \rho^2 )^\frac{3}{2}}{\frac{3}{2}} \Big|_0 ^{a\sqrt{\sin 2\varphi}}  \ d \varphi =
\frac{4}{3a} \int\limits_0 ^{\frac{\pi}{4}} (a^2 + a^2 \sin 2\varphi)^\frac{3}{2} - (a^2)^\frac{3}{2}  \ d \varphi = \\
\frac{4a^3}{3a} \int\limits_0 ^{\frac{\pi}{4}} (1+ \sin 2\varphi)^\frac{3}{2} - 1 \ d \varphi \\
1+ \sin 2\varphi = \sin^2 \varphi + \cos^2 \varphi + 2\sin\varphi \cos\varphi = (\sin\varphi + \cos\varphi)^2 \\
\sigma = \frac{4a^2}{3} \int\limits_0 ^{\frac{\pi}{4}} ((\sin\varphi + \cos\varphi)^2)^\frac{3}{2} - 1 \ d \varphi = \frac{4a^2}{3} \int\limits_0 ^{\frac{\pi}{4}} (\sin\varphi + \cos\varphi)^3 - 1 \ d \varphi = \\
\frac{4a^2}{3} \int\limits_0 ^{\frac{\pi}{4}} \sin^3 \varphi + 3\sin^2\varphi \cos\varphi + 3\sin\varphi \cos^2\varphi + \cos^3 \varphi - 1 \ d \varphi \\
\sigma = \frac{4a^2}{3} 
\left[ 
\int\limits_0 ^{\frac{\pi}{4}} \left( \sin^2\varphi \sin\varphi + 3\sin^2\varphi\cos\varphi\right) \ d \varphi+ 
\int\limits_0 ^{\frac{\pi}{4}} \left( \cos^2 \varphi \cos \varphi + 3 \cos^2 \varphi \sin\varphi - 1 \right)  \ d \varphi
\right] \\
\int\limits_0 ^{\frac{\pi}{4}} \left( \sin^2\varphi \sin\varphi + 3\sin^2\varphi\cos\varphi\right) \ d \varphi = 
\int\limits_0 ^{\frac{\pi}{4}} \left( (1 - \cos^2 \varphi) \sin\varphi + 3\sin^2\varphi\cos\varphi\right) \ d \varphi = \\
\int\limits_0 ^{\frac{\pi}{4}} \left( \sin\varphi - \cos^2 \varphi \sin\varphi + 3\sin^2\varphi\cos\varphi\right) \ d \varphi \\
\int\limits_0 ^{\frac{\pi}{4}} \left( \cos^2 \varphi \cos \varphi + 3 \cos^2 \varphi \sin\varphi - 1 \right)  \ d \varphi = 
\int\limits_0 ^{\frac{\pi}{4}} \left( (1 - \sin^2 \varphi) \cos \varphi + 3 \cos^2 \varphi \sin\varphi - 1 \right)  \ d \varphi = \\
\int\limits_0 ^{\frac{\pi}{4}} \left( \cos\varphi - \sin^2\varphi \cos\varphi + 3 \cos^2 \varphi \sin\varphi - 1 \right)  \ d \varphi
\end{gather*}
\begin{gather*}
\sin\varphi - \cos^2 \varphi \sin\varphi + 3\sin^2\varphi\cos\varphi 
+ \cos\varphi - \sin^2\varphi \cos\varphi + 3 \cos^2 \varphi \sin\varphi - 1 = \\
\sin \varphi + \cos \varphi + 2\sin^2 \varphi \cos \varphi + 2 \sin \varphi \cos^2 \varphi - 1 \\
\sigma = \frac{4a^2}{3} \left[ \int\limits_0 ^{\frac{\pi}{4}} 
\sin \varphi + \cos \varphi + 2\sin^2 \varphi \cos \varphi + 2 \sin \varphi \cos^2 \varphi - 1 \ d\varphi \right] = \\
\frac{4a^2}{3} \left[ 
\int\limits_0 ^{\frac{\pi}{4}} \sin \varphi \ d\varphi +
\int\limits_0 ^{\frac{\pi}{4}} \cos \varphi \ d\varphi + 
2\int\limits_0 ^{\frac{\pi}{4}} \sin^2 \varphi \cos \varphi \ d\varphi +
2\int\limits_0 ^{\frac{\pi}{4}} \sin \varphi \cos^2 \varphi \ d\varphi  -
\int\limits_0 ^{\frac{\pi}{4}} 1 \ d\varphi \right] \\
I_1 = \int\limits_0 ^{\frac{\pi}{4}} \sin \varphi \ d\varphi  = 
- \cos \varphi \Big|_0 ^{\frac{\pi}{4}} = -\left(\frac{\sqrt{2}}{2} - 1 \right) = 1 - \frac{\sqrt{2}}{2} \\
I_2 = \int\limits_0 ^{\frac{\pi}{4}} \cos \varphi \ d\varphi = 
\sin \varphi \Big|_0 ^{\frac{\pi}{4}} = \frac{\sqrt{2}}{2} - 0  = \frac{\sqrt{2}}{2} \\
I_3 = \int\limits_0 ^{\frac{\pi}{4}} \sin^2 \varphi \cos \varphi \ d\varphi = 
\frac{\sin^3 \varphi}{3} \Big|_0 ^{\frac{\pi}{4}} = \frac{1}{3} \left[ \left( \frac{\sqrt{2}}{2} \right) ^3 - 0 \right] = 
\frac{1}{3} \cdot \frac{2\sqrt{2}}{8} = \frac{\sqrt{2}}{12} \\
I_4 = \int\limits_0 ^{\frac{\pi}{4}} \sin \varphi \cos^2 \varphi \ d\varphi = 
-\frac{2 \cos^3 \varphi}{3} \Big|_0 ^{\frac{\pi}{4}}  = -\frac{1}{3} \left[ \left( \frac{\sqrt{2}}{2} \right) ^3 - 1 \right] = 
-\frac{\sqrt{2}}{12} + \frac{1}{3} = \frac{1}{3}-\frac{\sqrt{2}}{12} \\
I_5 = \int\limits_0 ^{\frac{\pi}{4}} 1 \ d\varphi = \varphi \Big|_0 ^{\frac{\pi}{4}} = \frac{\pi}{4} \\
\sigma  = \frac{4a^2}{3}  \left[ 1 - \frac{\sqrt{2}}{2}  + \frac{\sqrt{2}}{2} + \frac{2\sqrt{2}}{12} + \frac{2}{3}-\frac{2\sqrt{2}}{12} - \frac{\pi}{4} \right] = 
\frac{4a^2}{3}  \left[ 1 + \frac{2}{3} - \frac{\pi}{4} \right] \\
\sigma = \frac{4a^2}{3}  \left[\frac{5}{3} - \frac{\pi}{4} \right] = \frac{a^2}{9} \left[20 - 3\pi \right]
\end{gather*}


\newpage 
\section{Упражнение към лекция 13}

\subsection{Задачи}

\subsection*{Задача 1}
Да се пресметне 
$$I = \oint\limits_{\Gamma} (x+y) \ ds$$
където $\Gamma$ е затворена начупена линия $OABC, O(0,0), A(1,0), B(0,1)$.

\subsection*{Задача 2}
Да се пресметне 
$$I = \oint\limits_{C} \sqrt{x^2 + y^2} \ ds$$
където 
$$C \equiv \{x = a(\cos t + t \sin t), y = a(\sin t - t\cos t), 0 \leq t \leq 2\pi \}$$
 
\subsection*{Задача 3}
Да се пресметне 
$$I = \int\limits_{C} \frac{1}{x+y} \ ds$$
ако интегрирането е извършено по отреза на правата, свързваща точките $A(2,4)$ и $B(1,3)$
 
\subsection*{Задача 4}
Да се пресметне чрез криволинеен интеграл от първи род лицето на цилиндрична повърхнина
$y^2 = \frac{4}{9}(x-1)^3$ ограничена отдолу от равнината $z=0$, а отгоре от повърхнината $z = 2 - \sqrt{x}$.

\subsection*{Задача 5}
Да се намери стойността на 
$$I = \int\limits_{C}(x^2 + y^2 + z^2) \ ds$$
където C е част от винтовата линия с параметрични уравнения
$$x = a\cos t \quad y = a\sin t \quad z = bt \qquad 0 \leq t \leq 2\pi \qquad a>0,b>0$$

\subsection*{Задача 6}
Да се пресметне
$$\oint\limits_C (x^2-y^2) \ dx + (x^2+y^2) \ dy$$
в положителна посока на описване на елипсата 
$$\frac{x^2}{a^2} + \frac{y^2}{b} = 1$$
започвайки от точката $A(a,0), a>0, b>0$.

\subsection*{Задача 7}
\begin{enumerate}
\item Да се покаже, че ако пътят на интегриране не пресича ординатната ос, то интегралът
$$\int\limits_{(1,2)} ^{(2,1)} \frac{y \ dx - x \ dy}{x^2}$$
не зависи от пътя на интегриране, и да се пресметне. 
\item Да се пресметне стойността на интеграла от същата функция по затворен контур, непресичащ ординатната ос.
\end{enumerate}

\newpage
\subsection{Решения}

\subsection*{Задача 1}
\begin{gather*}
I = \oint\limits_{\Gamma} (x+y) \ ds \qquad \Gamma = AOBO \qquad O(0,0), A(1,0), B(0,1) \\
\Gamma = OA+AB+BO \implies I = \int\limits_{OA} (x+y) \ ds +  \int\limits_{AB} (x+y) \ ds +  \int\limits_{OB} (x+y) \ ds \\
I_1 = \int\limits_{OA} (x+y) \ ds  \\
I_1: \begin{cases} y(t) = 0 \\ x(t) = t \\ 0 \leq t \leq 1 \end{cases} \implies y'(t) = 0 \quad x'(t) = 1 \implies ds = \sqrt{0^2 + 1^2} dt = \sqrt{1} dt \\
I_1 = \int\limits_0 ^1 (t+0)\sqrt{1} \ dt = \int\limits_0 ^1 t \ dt = \frac{t^2}{2} \Big|_0 ^1 = \frac{1}{2} \\
I_2 = \int\limits_{AB} (x+y) \ ds  \\
I_2: AB: x+y=1 \Leftrightarrow AB: y=1-x, 0 \leq x \leq 1 \\
\implies y'=-1 \implies ds = \sqrt{(-1)^2 + 1^2} dx = \sqrt{2} dx \\
I_2 = \sqrt{2} \int\limits_0 ^1 \ dx = \sqrt{2} \\
I_3 = \int\limits_{OB} (x+y) \ ds  \\
I_3: OB: \begin{cases} x(t) = 0 \\ y(t) = t \end{cases} \implies x'(t) = 0 \quad y'(t) = 1 \implies ds = \sqrt{0^2 + 1^2} dt = \sqrt{1} dt \\
I_3 = \int\limits_0 ^1 (0+t)\sqrt{1} \ dt = \int\limits_0 ^1 t \ dt = \frac{t^2}{2} \Big|_0 ^1 = \frac{1}{2} \\
I = \frac{1}{2} + \sqrt{2} + \frac{1}{2} = \sqrt{2} + 1 
\end{gather*}

\subsection*{Задача 2}
\begin{gather*}
I = \oint\limits_{C} \sqrt{x^2 + y^2} \ ds \qquad 
C \equiv \{x = a(\cos t + t \sin t), y = a(\sin t - t\cos t), 0 \leq t \leq 2\pi \} \\
x' = a(-\sin t + \sin t + t\cos t) \implies x' = at\cos t \\
y'= a(- \cos t + \cos t + t\sin t) \implies y' = at\sin t \\
ds = \sqrt{x'^2 + y'^2} dt = \sqrt{a^2t^2 \cos^2 t + a^2t^2 \sin^2 t}dt = \\
\sqrt{a^2t^2(\cos^2 t + \sin^2 t)}dt = \sqrt{a^2t^2}dt = atdt \\
x^2 + y^2 = a^2(\cos t + t \sin t)^2 + a^2(\sin t - t\cos t)^2 = \\
a^2 \left[\cos^2 t + t^2\sin^2t + 2t\sin t \cos t + \sin^2 t + t^2\cos^2 t - 2t\sin t \cos t \right] = \\
a^2 \left[ (\cos^2 t + \sin^2 t) + t^2(\cos^2 t + \sin^2 t)\right] = a^2 \left[1+t^2 \right] \implies \\
\sqrt{x^2+y^2} = a(1+t^2)^{\frac{1}{2}} \\
I = \int\limits_0 ^{2\pi}  a(1+t^2)^{\frac{1}{2}} at \ dt = 
a^2 \int\limits_0 ^{2\pi} (1+t^2)^{\frac{1}{2}} t \ dt = 
a^2 \frac{1}{2} \int\limits_0 ^{2\pi} (1+t^2)^{\frac{1}{2}} d(t^2 + 1) =  \\
\frac{a^2}{2} \cdot \frac{(1+t^2)^{\frac{3}{2}}}{\frac{3}{2}} \Big|_0 ^{2\pi} = 
\frac{a^2}{3} \left[(1+4\pi^2)^{\frac{3}{2}} - 1 \right]
\end{gather*}

\subsection*{Задача 3}
\begin{gather*}
I = \int\limits_{C} \frac{1}{x+y} \ ds \qquad C=AB \quad A(2,4), B(1,3)\\
y-y_1 = k(x-x_1) \implies k = \frac{y_2-y_1}{x_2 - x_1} = \frac{4-3}{2-1} = 1 \implies \\
AB: y-3 = 1(x-1); \qquad x-1 \implies y = 3 \qquad x=2 \implies y=4 \implies \\
AB: y=x+2; 1 \leq x \leq 2 \\
ds = \sqrt{1 + y'^2}dx = \sqrt{1+1}dx = \sqrt{2} dx \\
x+y = x +x+2 = 2x+2 = 2(x+1) \\
I = \int\limits_1 ^2 \frac{\sqrt{2}}{2(x+y)} \ dx =
\frac{\sqrt{2}}{2} \int\limits_1 ^2 \frac{1}{x+1} \ dx =
\frac{\sqrt{2}}{2} \int\limits_1 ^2 \frac{1}{x+1} \ d(x+1) = 
\frac{\sqrt{2}}{2} \ln |x+1| \Big|_1 ^2 = \\
\frac{\sqrt{2}}{2} (\ln 3 - \ln 2) =  
\frac{\sqrt{2}}{2} \ln \left( \frac{3}{2} \right)
\end{gather*}

\subsection*{Задача 4}
\begin{gather*}
\sigma = \int\limits_C z \ dl \qquad 2-\sqrt{x}, z = 0 \qquad C: y^2 = \frac{4}{9} (x-1)^3, x \geq 1 \\
y = \pm \frac{2}{3} (x-1)^{\frac{3}{2}} \implies y' = \pm \frac{2}{3} \cdot \frac{3}{2} (x-1)^{\frac{1}{2}} =
y' = \pm (x-1)^{\frac{1}{2}} \implies y'^2 = x-1 \\
dl = \sqrt{1+x-1}dx = \sqrt{x} dx \\
0 \leq z \leq 2-\sqrt{x} \implies 2-\sqrt{x} \geq 0 \implies x \leq 4 \implies 1\leq x \leq 4\\
\sigma = 2\int\limits_1 ^4 (2-\sqrt{x})\sqrt{x} \ dx = 
2\int\limits_1 ^4 (2\sqrt{x} - x) \ dx = \\
2\left[2 \cdot \frac{x^{\frac{3}{2}}}{\frac{3}{2}} - \frac{x^2}{2} \right] \Big|_1 ^4 = 
2\left[\frac{4}{3} x^{\frac{3}{2}} - \frac{x^2}{2} \right] \Big|_1 ^4 = 
2\left[\frac{4}{3} \left( \sqrt{4^3} - 1 \right)-  \frac{1}{2} \left(16-1 \right) \right] = \\
2\left[\frac{4}{3} \left( 7 \right)- \frac{1}{2} \left( 15 \right) \right] = 
2\left[\frac{28}{3} - \frac{15}{2} \right] = 
2\left[\frac{56-45}{6} \right] = \frac{11}{3}
\end{gather*}

\subsection*{Задача 5}
\begin{gather*}
I = \int\limits_{C}(x^2 + y^2 + z^2) \ ds \\
x = a\cos t \quad y = a\sin t \quad z = bt \qquad 0 \leq t \leq 2\pi \qquad a>0,b>0 \\
x' = -a\sin t \qquad y' = a\cos t \qquad z = b \implies \\
x'^2 + y'^2 + z'^2 = a^2 \sin^2 t + a^2 \cos^2 t + b^2 = a^2+b^2 \implies dl = \sqrt{a^2+b^2}dt \\
x^2 + y^2 + z^2 = a^2 \cos^2 t + a^2 \sin^2 t + b^2 t^2 = a^2 + b^2t^2 \\
I = \int\limits_0 ^{2\pi} (a^2 + b^2t^2)\sqrt{a^2+b^2} \ dt = 
\sqrt{a^2+b^2} \int\limits_0 ^{2\pi} (a^2 + b^2t^2) \ dt = \\
\sqrt{a^2+b^2} \left[ a^2t + \frac{b^2t^3}{3}\right] \Big|_0 ^{2\pi} = 
\sqrt{a^2+b^2} \left[ a^2 2\pi + \frac{b^2(2\pi)^3}{3} \right] = 
\frac{\sqrt{a^2+b^2}}{3} [a^2 2 \cdot3 \pi + 8\pi^3b^2] = \\
\frac{\sqrt{a^2+b^2}}{3} [6\pi a^2 + 8\pi^3 b^2] = 
\frac{2\pi}{3}\sqrt{a^2+b^2} [3a^2 + 4\pi^2 b^2] = 
\end{gather*}

\subsection*{Задача 6}
\begin{gather*}
\oint\limits_C (x^2-y^2) \ dx + (x^2+y^2) \ dy \qquad \frac{x^2}{a^2} + \frac{y^2}{b} = 1\\
C: x =a\cos t \qquad y = b\sin t \qquad  0\leq t \leq 2\pi \\
x^2-y^2 = a^2 \cos^2 t - b^2\sin^2 t \qquad dx = -a\sin t dt \\
(x^2 - y^2) dx = (a^2 \cos^2 t - b^2\sin^2 t)(-a\sin t) dt \\
x^2+y^2 = a^2 \cos^2 t +  b^2\sin^2 t \qquad dy = b\cos t dt \\
(x^2 + y^2) dy = (a^2 \cos^2 t + b^2\sin^2 t)(b\cos t) dt \\
I = \int\limits_0 ^{2\pi} (a^2 \cos^2 t - b^2\sin^2 t)(-a\sin t) \ dt + (a^2 \cos^2 t + b^2\sin^2 t)(b\cos t) \ dt \\
I = \int\limits_0 ^{2\pi} -a^3 \cos^2 t +ab^2\sin^3 t dt + a^2b \cos^3 t + b^3\sin^2 t \ dt \\
I = -a^3 \int\limits_0 ^{2\pi} \cos^2 t \sin t \ dt + 
ab^2 \int\limits_0 ^{2\pi} sin^3 t \ dt + 
a^2b \int\limits_0 ^{2\pi} \cos^3 t \ dt + 
b^3 \int\limits_0 ^{2\pi} \sin^2 t \cos t \ dt \\
I = -a^3 I_1 + ab^2 I_2 +a^2 b I_3 + b^3 I_4 \\
I_1 = \int\limits_0 ^{2\pi} \cos^2 t \sin t \ dt = 
- \int\limits_0 ^{2\pi} \cos^2 t \ d(\cos t) = -\frac{\cos^3 t}{3} \Big|_0 ^{2\pi} = 0 \\
I_2 = \int\limits_0 ^{2\pi} sin^3 t \ dt = \int\limits_0 ^{2\pi} (1-\cos^2 t) \ d(\cos t) = 0 \\
I_3 = \int\limits_0 ^{2\pi} \cos^3 t \ dt  = - \int\limits_0 ^{2\pi} (1- \sin^2 t) \ d(\sin t) = 0 \\
I_4 =  \int\limits_0 ^{2\pi} \sin^2 t \cos t \ dt =  \int\limits_0 ^{2\pi} \sin^2 t  \ d(\sin t) = \frac{\sin^3 t}{3} \Big|_0 ^{2\pi} = 0 \\
I = 0
\end{gather*}

\subsection*{Задача 7}
1.\\
\begin{gather*}
\int\limits_{(1,2)} ^{(2,1)} \frac{y \ dx - x \ dy}{x^2}\\
\text{Кривата не пресича $Oy$}\\
P(x) = \frac{y}{x^2} \qquad Q(x) = - \frac{1}{x} \implies x \neq 0 \\
\frac{\partial P}{\partial y} = \frac{1}{x^2} \qquad \frac{\partial Q}{\partial x} = \frac{1}{x^2} \implies \frac{\partial P}{\partial y} = \frac{\partial Q}{\partial x} \\
1) \text{Интегралът I не зависи от пътя на интегриране} \\
2) \exists u = u(x,y) : u'_x = \frac{y}{x^2} \quad u'_y = - \frac{1}{x}\\
u'_x = \frac{y}{x^2} \implies u(x,y) = \int \frac{y}{x^2} \ dx \implies \\
u(x,y) =  y \int \frac{1}{x^2} \ dx  y \int x^{-2} \ dx = y\frac{x^{-1}}{-1} + \varphi(y) = -\frac{y}{x}+ \varphi(y) \\
u'_y = - \frac{1}{x} \qquad u'_y = - \frac{1}{x} + \varphi'(y) \implies - \frac{1}{x} = - \frac{1}{x} + \varphi'(y) \implies \\
\varphi'(y) = 0 \implies \varphi(y) = const = c \implies \\
u(x,y) = - \frac{y}{x} + c 
I = \int\limits_{(1,2)} ^{(2,1)} \frac{y \ dx - x \ dy}{x^2} = -\frac{y}{x} \Big|_{(1,2)} ^{(2,1)} \implies \\
I = - \left[\frac{2}{1} - \frac{1}{2}\right] = - \frac{3}{2}
\end{gather*}
2. \\
\begin{gather*}
\int\limits_{\gamma} P \ dx + Q \ dy = ? \\
\gamma_1 = ACB \qquad \gamma_2 = ADB \qquad \gamma^- _2 = BDA \\
\int\limits_{\gamma_1} P \ dx + Q \ dy = \int\limits_{\gamma_2} P \ dx + Q \ dy \implies \\
\int\limits_{\gamma_1} P \ dx + Q \ dy - \int\limits_{\gamma_2} P \ dx + Q \ dy = 0 \implies \\ 
\int\limits_{\gamma_1} P \ dx + Q \ dy + \int\limits_{\gamma^- _2} P \ dx + Q \ dy = 0 \implies \\
\int\limits_{ACBDA} P \ dx + Q \ dy = 0 
\end{gather*}

\newpage
\section{Примерен изпит}

\subsection{Задачи}

\subsection*{Задача 1}
Да се провери изпълнено ли е равенството, за такива стойности на независимите променливи, за които частните производни на функцията $z$ съществува
$$\frac{1}{x} z'_x + \frac{1}{y} z'_y = \frac{z}{x^2} \qquad \qquad z = 2x\arcsin(y^2- x^2)$$

\subsection*{Задача 2}
Дадени за функциите и точката
$$f(x,y) = x^4 + y^4 - 14xy - 7x^2 - 7y^2 + 24x + 24y \qquad F(x,y,z) = f(x,y) + e^{2x^2 + 3y^2 + 4z^2} \qquad A(1,1,1)$$
Да се намери
\begin{enumerate}
\item Локалните екстремуми на f и вида им. 
\item $g = grad F(A), \Vert g \Vert, \frac{\partial F(A)}{\partial g}$.
\end{enumerate}

\subsection*{Задача 3}
Нека y e неявна функция, която се дефинира с уравнението $f(x,y) = 0$, където 
$$f(x,y) = \sin(x-y) + e^{xy} - 1 \quad y(0) = 0 $$
Да се намери $y'(0)$.

\subsection*{Задача 4}
Да се пресметне интеграла
$$\iint\limits_D (6x^2 + y) \ dx \ dy$$
където D е оградена от кривите с уравнение $y=2x \text{ и } y = x^2 - x$.

\subsection*{Задача 5}
Пресметнете криволинейния интеграл от първи род
$$\int\limits_C (x^2 + y^2 - 2z) \ dl \qquad C: \{x= 4\cos 2t , y = 4\sin 2t, z = 6t, t \in [0,3\pi] \}$$

\subsection*{Задача 6}
Да се пресметне криволинейния интеграл от втори род
$$\int\limits_C 2x \ dx + 3yx \ dy \qquad C = \{x= 4\cos 2t , y = 3\sin 2t, t \in [0,2\pi] \}$$
\begin{enumerate}
\item непосредствено
\item с формулата на Грийн.
\end{enumerate}
Да се сравнят резултатите. 

\subsection*{Задача 7}
Да се намери лицето на частта от повърхнината 
$$S: z = -\sqrt{x^2 + y^2}$$
заградена от цилиндъра
$$x^2 + y^2 = 36$$

\subsection*{Задача 8}
Намерете обема на тялото
$$T = \{z \leq 2 - x^2 - y^2, z^2 \geq x^2 + y^2 \quad (z \geq 0) \}$$

\newpage
\subsection{Решения}

\subsection*{Задача 1}
\begin{gather*}
\frac{1}{x} z'_x + \frac{1}{y} z'_y = \frac{z}{x^2} \qquad \qquad z = 2x\arcsin(y^2- x^2) \\
z'_x = 2\arcsin(y^2 - x^2) + \frac{2x}{\sqrt{1 + (y^2 - x^2)^2}} \cdot -2x = 
2\arcsin(y^2 - x^2) - \frac{4x^2}{\sqrt{1 + (y^2 - x^2)^2}} = \\
z'_x = \frac{2\sqrt{1 + (y^2 - x^2)^2}\arcsin(y^2 - x^2) - 4x^2}{\sqrt{1 + (y^2 - x^2)^2}} \\
z'_y = 0\arcsin(y^2 - x^2) + \frac{2x}{\sqrt{1 + (y^2 - x^2)^2}} \cdot 2y =  \frac{4xy}{\sqrt{1 + (y^2 - x^2)^2}}\\
\frac{1}{x} z'_x= \frac{\frac{2\sqrt{1 + (y^2 - x^2)^2}\arcsin(y^2 - x^2) - 4x^2}{\sqrt{1 + (y^2 - x^2)^2}}}{x} =
\frac{2\sqrt{1 + (y^2 - x^2)^2}\arcsin(y^2 - x^2) - 4x^2}{\sqrt{1 + (y^2 - x^2)^2}} \cdot \frac{1}{x} = \\
\frac{2\sqrt{1 + (y^2 - x^2)^2}\arcsin(y^2 - x^2) - 4x^2}{x\sqrt{1 + (y^2 - x^2)^2}} \\
\frac{1}{y} z'_y = \frac{1}{y} \cdot \frac{4xy}{\sqrt{1 + (y^2 - x^2)^2}} = \frac{4x}{\sqrt{1 + (y^2 - x^2)^2}} \\
\frac{1}{x} z'_x + \frac{1}{y} z'_y  = \frac{2\sqrt{1 + (y^2 - x^2)^2}\arcsin(y^2 - x^2) - 4x^2}{x\sqrt{1 + (y^2 - x^2)^2}}  + \frac{4x}{\sqrt{1 + (y^2 - x^2)^2}} = \\
\frac{2\sqrt{1 + (y^2 - x^2)^2}\arcsin(y^2 - x^2) - 4x^2 + 4x^2}{x\sqrt{1 + (y^2 - x^2)^2}} = 
\frac{2\sqrt{1 + (y^2 - x^2)^2}\arcsin(y^2 - x^2)}{x\sqrt{1 + (y^2 - x^2)^2}} \\
\frac{1}{x} z'_x + \frac{1}{y} z'_y = \frac{2\arcsin(y^2 - x^2)}{x} \\
\frac{z}{x^2} = \frac{2x\arcsin(y^2- x^2)}{x^2} =  \frac{2\arcsin(y^2- x^2)}{x} \implies \text{Равенството е изпълнено}
\end{gather*}

\subsection*{Задача 2}
$$f(x,y) = x^4 + y^4 - 14xy - 7x^2 - 7y^2 + 24x + 24y \qquad F(x,y,z) = f(x,y) + e^{2x^2 + 3y^2 + 4z^2} \qquad A(1,1,1)$$
\begin{enumerate}
\item Локалните екстремуми на f и вида им. 
\begin{gather*}
f'_x = 4x^3 - 14x - 14y + 24  \qquad f'_y = 4y^3 - 14x - 14y + 24 \\
\begin{array}{|l@{}}
4x^3 - 14x - 14y + 24 = 0 \\
4y^3 - 14x - 14y + 24 = 0 
\end{array}\Leftrightarrow
\begin{array}{|l@{}}
4x^3 - 4y^3 = 0 \\
4y^3 - 14x - 14y + 24 = 0 
\end{array}\\
4x^3 - 4y^3 = 0 \Leftrightarrow x^3 - y^3 = 0 \Leftrightarrow (x-y)(x^2+xy+y^2) = 0 \\
f''_{xx} = 12x^2 - 14 \qquad f''_{yy} = 12y^2 - 14 \qquad f''_{xy} = f''_{yx} = -14 \\
\Delta = f''_{xx}f''_{yy} - f''_{xy}f''_{yx} = (12x^2 - 14)(12y^2 - 14) - 196 \\
\end{gather*}
\item $g = grad F(A), \Vert g \Vert, \frac{\partial F(A)}{\partial g}$.
\end{enumerate}
\subsection*{Задача 3}

\subsection*{Задача 4}
$$I = \iint\limits_D (6x^2 + y) \ dx \ dy \qquad \qquad D: \begin{cases} y=2x \\ y = x^2 - x \end{cases}$$
\begin{gather*}
\begin{array}{|l@{}} y=2x \\ y = x^2 - x \end{array} \Leftrightarrow 2x = x^2 - x \Leftrightarrow x^2 - 3x = 0 \implies 0 \leq x \leq 3 \implies \begin{array}{|l@{}} 0 \leq x \leq 3 \\  x^2 - x \leq y \leq2x \end{array} \\
I = \int\limits_0 ^3 \int\limits_{x^2 - x} ^{2x} 6x^2 + y \ dy \ dx \\
\int\limits_{x^2 - x} ^{2x} 6x^2 + y \ dy = 6x^2y + \frac{1}{2} y^2 \Big|_{x^2 - x} ^{2x} = \\
6x^2\cdot 2x + \frac{4x^2}{2} - 6x^2(x^2-x) - \frac{(x^2-x)^2}{2} = 12x^3 + 2x^2 - 6x^4 + 6x^3 - \frac{x^4 - 2x^3 +x^2}{2} = \\
\frac{24x^3 + 4x^2 - 12x^4 + 12x^3 - x^4 + 2x^3 - x^2}{2} = \frac{-13 x^4 + 38x^3 + 3x^2}{2} = -\frac{13}{2} x^4 + 19x^3 + \frac{3}{2} x^2 \\
I = \int\limits_0 ^3 -\frac{13}{2} x^4 + 19x^3 + \frac{3}{2} x^2  \ dx = -\frac{13x^5}{2 \cdot 5} + \frac{19x^4}{4} + \frac{3x^3}{2 \cdot 3} = - \frac{13}{10} x^5 + \frac{19}{4} x^4 + \frac{1}{2} x^3 \Big|_0 ^3 \\
I = \frac{(-13\cdot 2 \cdot 3^5) + (19 \cdot 5 \cdot 3^4) + (1 \cdot 10 \cdot 3^3)}{20} = \frac{1647}{20}
\end{gather*}

\subsection*{Задача 5}
$$I = \int\limits_C (x^2 + y^2 - 2z) \ dl \qquad C: \{x= 4\cos 2t , y = 4\sin 2t, z = 6t, t \in [0,3\pi] \}$$
\begin{gather*}
I = \int\limits_C f(x,y,z) \ dl \qquad C: \{x = \varphi(t), y=\psi(t), z=\chi(t), t \in [t_1, t_2] \} \implies \\
I = \int\limits_{t_1} ^{t_2} f(\varphi(t), \psi(t), \chi(t)) \sqrt{\varphi'^2(t), \psi'^2(t), \chi'^2(t)} \ dt  \\
\varphi' = -8\sin 2t \qquad \psi' = 8\cos 2t \qquad \chi' = 6 \\
\varphi'^2 + \psi'^2 + z'^2 = 64\sin^2 2t + 64 \cos^2 2t + 36 = 64 (\sin^2 2t + \cos^2 2t) + 36 = 64 + 36 = 100 \\
\sqrt{\varphi'^2 + \psi'^2 + \chi'^2} = 10 \\
dl = \sqrt{\varphi'^2 + \psi'^2 + \chi'^2} dt = 10 dt \\
x^2 + y^2 - 2z = 16\cos^2 2t + 16 \sin^2 2t - 12t = 16(\cos^2 2t + \sin^2 2t) - 12t = 16 - 12t \\
I = \int\limits_0 ^{3\pi} (16 - 12t) \cdot 10 \ dt = \int\limits_0 ^{3\pi} 160 - 120t \ dt = 160t - 60t^2 \Big|_0 ^{3\pi} = 160\cdot 3\pi - 60 \cdot 9\pi^2 = \\
480\pi - 540 \pi^2 
\end{gather*}

\subsection*{Задача 6}
$$I= \int\limits_C 2x \ dx + 3yx \ dy \qquad C = \{x= 4\cos 2t , y = 3\sin 2t, t \in [0,2\pi] \}$$
\begin{enumerate}
\item непосредствено
\begin{gather*}
P(t) = 8\cos 2t \quad Q(t) = 36 \cos 2t \sin 2t  \quad x' = -8\sin 2t \quad y' = 6\cos 2t \implies \\
I = \int\limits_0 ^{2\pi} Px' + Qy' \ dt = \int\limits_0 ^{2\pi} -64\cos 2t \sin 2t + 216 \cos^2 2t \sin 2t  \ dt  = \\
\int\limits_0 ^{2\pi} -64\cos 2t \sin 2t \ dt + \int\limits_0 ^{2\pi} 216 \cos^2 2t \sin 2t  \ dt = \\
- \frac{64}{2} \int\limits_0 ^{2\pi} \cos 2t\ d(\cos 2t) + \frac{216}{2} \int\limits_0 ^{2\pi}  \cos^2 2t \ d(\cos 2t) = \\
-32 \left(- \frac{\cos^2 2t}{2} \right) + 108 \left(- \frac{\cos^3 2t}{3} \right) \Big|_0 ^{2\pi}= \\
16\cos^2 4\pi - 36 \cos^3 4\pi - 16\cos^2 0 - 36 \cos^3 0 = 0\\ 
\end{gather*}
\item с формулата на Грийн.
\begin{gather*}
P = 2x \quad Q = 3yx \\
\frac{\partial P}{\partial y} = P'_y = 0  \qquad \frac{\partial Q}{\partial x} = Q'_x = 3y \\
I = \iint\limits_D \left( \frac{\partial Q}{\partial x} - \frac{\partial P}{\partial y} \right) \ dx \ dy = \iint\limits_D \left( Q'_x  - P'_y \right) \ dx \ dy = \iint\limits_D 3y \ dx \ dy \\
\end{gather*}
\end{enumerate}
Да се сравнят резултатите. 

\subsection*{Задача 7}
Да се намери лицето на частта от повърхнината 
$$S: z = -\sqrt{x^2 + y^2}$$
заградена от цилиндъра
$$x^2 + y^2 = 36$$
\begin{gather*}
\sigma = \iint\limits_C  \sqrt{1 + z'^2 _x + z'^2 _y} \ dx \ dy \qquad  C: \{ x^2 + y^2 = 36 \} \\
z'_x = - \frac{x}{\sqrt{x^2 + y^2}} \qquad z'_y = - \frac{y}{\sqrt{x^2 + y^2}} \\
1 + z'^2 _x + z'^2 _y = 1 + \frac{x^2}{x^2 + y^2} + \frac{y^2}{x^2 + y^2} = 
\frac{x^2 + y^2 +x^2 + y^2 }{x^2 + y^2} = \frac{2(x^2+y^2)}{x^2 + y^2} = 2 \\
\sqrt{1 + z'^2 _x + z'^2 _y} = \sqrt{2} \\
\sigma = \iint\limits_C \sqrt{2} \ dx \ dy \\
\text{Полярна смяна: }
x = a\cos(t) \qquad y = a \sin t \qquad |\Delta| = a \qquad  a\geq 0, t \in [0,2\pi] \\
C = a^2 \cos^2 t + a^2 \sin^2 t = 36 \Leftrightarrow a^2 = 36 \implies a \in [0,6] \\ 
\sigma = \int\limits_0 ^{2\pi} \int\limits_0 ^6 \sqrt{2} a \ da \ dt \qquad 
\int\limits_0 ^6 \sqrt{2} a \ da = \sqrt{2}\frac{a^2}{2} \Big|_0 ^6 = \frac{36}{2} \sqrt{2} = 18\sqrt{2} \\
\sigma = 18\sqrt{2}  \int\limits_0 ^{2\pi} \ dt = 18\sqrt{2} \cdot  2\pi = 36\sqrt{2} \pi 
\end{gather*} 

\subsection*{Задача 8}
Намерете обема на тялото
$$T : z \leq 2 - x^2 - y^2, z^2 \geq x^2 + y^2 \quad (z \geq 0) $$
\begin{gather*}
V_T = \iiint\limits_T \ dx \ dy \ dz \\
\begin{array}{|l@{}}
z \leq 2 - x^2 - y^2 \\ z^2 \geq x^2 + y^2 \\ z \geq 0 
\end{array} \\
\text{Цилиндрична смяна} \\
x = t \cos a \quad y = t \sin a  \quad  z = z  \quad  (t\geq 0, a \in [0,2\pi]) \\
\implies |\Delta| = t \\ 
2 - x^2 - y^2 = \sqrt{x^2 + y^2} \Leftrightarrow 2-t^2 = t \\
t^2 + t -2 = 0 \Leftrightarrow (t+2)(t-1) = 0 \implies t = 1 \implies t \in [0,1]  \qquad z \in [0,2-t^2]\\ 
\int\limits_0 ^{2\pi} \int\limits_0 ^1 \int\limits_0 ^{2-t^2} t \ dz \ dt \ da  = \int\limits_0 ^{2\pi} \int\limits_0 ^1  t(2-t^2) \ dt \ da = \int\limits_0 ^{2\pi} \frac{2t^2}{2} - \frac{t^4}{4} \Big|_0 ^1 \ da =  \\
\int\limits_0 ^{2\pi} \frac{3}{4} \Big|_0 ^1 \ da  = \frac{3\cdot 2\pi}{4} = \frac{3}{2} \pi \\
\end{gather*} 
2ра опция
\begin{gather*}
z^2 \geq x^2 + y^2 \implies  z \geq \pm \sqrt{x^2 + y^2 }, z \geq 0 \implies z \geq \sqrt{x^2 + y^2} \implies \\
\sqrt{x^2 + y^2} \leq z \leq 2 - x^2 - y^2 \\
D: \{ (x,y) , \sqrt{x^2 + y^2} \leq 2-x^2 - y^2 \} \\
\text{Цилиндрична смяна} \\
x = t \cos a \quad y = t \sin a  \quad  z = z  \quad  (t\geq 0, a \in [0,2\pi]) \implies |\Delta| = t \\ 
2 - x^2 - y^2 \leq \sqrt{x^2 + y^2} \Leftrightarrow 2-t^2 = t \\
t^2 + t -2 = 0 \Leftrightarrow (t+2)(t-1) = 0 \implies t = 1 \implies t \in [0,1]  \qquad z \in [t,2-t^2]\\ 
\int\limits_0 ^{2\pi} \int\limits_0 ^1 \int\limits_t ^{2-t^2} t \ dz \ dt \ da  = 
\int\limits_0 ^{2\pi} \int\limits_0 ^1  t(2 - t^2 - t) \ dt \ da = \int\limits_0 ^{2\pi} \frac{2t^2}{2} - \frac{t^4}{4} - \frac{t^3}{3} \Big|_0 ^1 \ da =  \\
\int\limits_0 ^{2\pi} \frac{5}{12} \Big|_0 ^1 \ da  = \frac{5\cdot 2\pi}{12} = \frac{5}{6} \pi \\
\end{gather*} 


































\end{document}