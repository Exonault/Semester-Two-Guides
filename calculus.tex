\documentclass[fleqn]{article}

\usepackage[utf8]{inputenc}
\usepackage[bulgarian]{babel}
\usepackage{amsmath}

\newtheorem{theorem}{Tеорема}[subsection]
\newtheorem{corollary}{Следствие}[theorem]
\newtheorem{lemma}[theorem]{Лема}

\newtheorem{definition}{Дефиниция}[subsection]

\title{Математически анализ 2}
\author{Exonaut}


\begin{document}
\maketitle
\pagenumbering{gobble}
\newpage
\pagenumbering{arabic}

\tableofcontents
\newpage

\section{Лекция 1: Пространството $R^m$}
\subsection{Няколко важни неравенства}
Нека $a_k $ и $b_k (k = 1, 2, ..., m) $ са реални числа и $m \in N$
\begin{theorem}[Неравенство на Коши-Шварц]
В сила е следното неравенство: \\
$$\left( \sum_{k=1}^{m}a_kb_k \right ) ^ 2   \leq  \left( \sum_{k=1}^{m}a_k \right)  \left( \sum_{k=1}^{m}b_k  \right) $$
\end{theorem}
Равенство се достига само когато $a_k$ и $b_k$ са пропорционални:\\
 ($\exists \lambda_0: b_k =\lambda_0 a_k  $)\\
Равенството може да се запише: 
$$\displaystyle\left\lvert \sum_{k=1}^{m}a_kb_k \right \rvert \leq \sqrt{\left( \sum_{k=1}^{m}a_k \right)}  \sqrt{ \left( \sum_{k=1}^{m}b_k  \right)} $$

\begin{theorem}[Неравенство на Минковски]
В сила е следното неравенство: \\
$$\sqrt{ \sum_{k=1}^{m} (a_k + b_k)^2}   \leq \sqrt{ \sum_{k=1}^{m}a_k^2 } + \sqrt{ \sum_{k=1}^{m}b_k^2 }$$
\end{theorem}
Равенство се достига само когато $a_k$ и $b_k$ са пропорционални.
Общ случай на неравенството на Минковски:
$$\left ( \sum_{k=1}^{m} |a_k + b_k|^p \right)^\frac{1}{p}   \leq\left ( \sum_{k=1}^{m}|a_k|^p\right)^\frac{1}{p} + \left ( \sum_{k=1}^{m} |b_k|^p \right)^\frac{1}{p}   ( p\geq 1)$$

\begin{theorem}
В сила е следното неравенство: \\
$$ |a_k + b_k|  \leq \sqrt{ \sum_{k=1}^{m} (a_k + b_k)^2} \leq   \sum_{k=1}^{m}|a_k - b_k| $$
\end{theorem}

\subsection{Видове крайно мерни пространства}

\subsubsection{Линейно(Векторно) пространство}
	\begin{definition}
Нека L е линейно(векторно) пространство над полето $R$. В него има въведени две операции: събиране и умножение на вектор с число.
		\begin{enumerate}
 			\item $x,y \in L \implies z = x+ y \in L $
			\item $x \in L, \lambda \in R \implies z = \lambda x \in L$
		\end{enumerate}
\end{definition}

\subsubsection{Евклидово пространство}
\begin{definition}
Крайномерното пространство L се нарича евклидово, ако в него е въведено скаларно произведение, т.е за всеки два елемента $x,y \in L$ може да се съпостави реално число $(x,y)$, удовлетворяващо свойствата за линейност, симетричност и положителна определеност.
	\begin{enumerate}
 		\item $x,y,z \in L ,\lambda \in R \implies (x+y,z) = (x,z) + (y,z); (\lambda x,y) = \lambda(x,y)$
		\item $x,y \in L \implies (x,y) = (y,x)$
		\item $x \in L, x\neq 0 \implies (x,x) > 0 $
	\end{enumerate}
\end{definition}

\subsubsection{Метрично пространство}
\begin{definition}
Крайномерното пространство L се нарича метрично, ако в него е въведено разстояние (метрика) $\rho$, т.е за два елемента $x,y \in L$ може да се съпостави неотрицателно число  $\rho \geq 0 $  със следните свойства
	\begin{enumerate}
 		\item $\rho(x,x) = 0 ; \rho(x,y) > 0 , x\neq y $
		\item $\rho(x,y) = \rho(y,x)$
		\item $\rho(x,z) \leq \rho(x,y) + \rho(y,z). \forall x,y,z \in L$
	\end{enumerate}
Метрично пространство L с метрика $\rho$ се означава $(L,\rho)$
\end{definition}

\subsubsection{Нормирано пространство}
\begin{definition}
Пространството се нарича нормирано, ако в него е въведена норма $||.||$, т.е $||.||: L \rightarrow R_0^+$ със свойства 
	\begin{enumerate}
 		\item $x = 0  \implies ||x|| = 0, x \neq 0  \implies ||x|| > 0$
		\item $x \in L, \lambda \in R \implies ||\lambda x|| = |\lambda|  ||x||$
		\item $x,y \in L \implies ||x+y|| \leq |x| + |y|$
	\end{enumerate}
\end{definition}
\begin{theorem}
Ако L е нормирано пространство с дадена норма $|| . ||$, то L е метрично пространство, т.е равенството $\rho(x,y) = ||x - y||$ дефинира  разстоянието в L
\end{theorem}

\subsection{Пространството $R^m$ - дефиниция и основни свойства}

\begin{definition}
Множеството от наредени m-торки $a = (a_1, a_2, ... , a_m)$ от реални числа. Числата $a_1, a_2, ... , a_m$ се наричат съответно първа, втора, ..., m-та кордината на а.\\
\\
 Ако имаме $a = (a_1, a_2, ... , a_m), b = (b_1, b_2, ... , b_m), ; \lambda \in R $ то
	\begin{enumerate}
		\item $a+b =(a_1, a_2, ... , a_m) +  (b_1, b_2, ... , b_m) = (a_1 + b_1, a_2+b_2, ... , a_m + b_m) \in R^m$
		\item $ \lambda a =(\lambda a_1,\lambda a_2, ... , \lambda a_m) \in R^m $\\
	\end{enumerate}
\end{definition}
\subsubsection{Скаларно произведение}
Скаларно произведение се дефинира : $$ (a,b) = \left( \sum_{k=1}^{m}a_kb_k \right )$$\\
С така въведено скаларно произведение пространството $R^m$ се превръща в евклидово.
\subsubsection{Норма и метрика}
С равенството:
$$||a|| := \sqrt{ \sum_{k=1}^{m}(a_k)^2  }$$ \\
се въвежда норма в $R^m$.
\\
 Нормата генерира метрика в $R^m$ с формула: 
$$\rho(a,b) := ||a-b|| = \sqrt{ \sum_{k=1}^{m} (a_k - b_k)^2} $$
\subsubsection{Скаларен квадрат}
Скаларен квадрат: $a^2 = (a,a) = \sum _{k=1}^{m}a_k^2$
\subsubsection{Неравенство на Коши-Шварц, чрез скаларен квадрат}
Коши-Шварц чрез скаларен квадрат: $(a,b)^2 \leq a^2b^2$ и $|(a,b)|\leq ||a|| ||b||$
\subsubsection{Неравенство на Минковски, чрез скаларен квадрат}
Неравенство на Минковски чрез скаларен квадрат: $||a+b||\leq ||a|| + ||b||$
\subsection{Точки и множества в $R^m$}

\subsubsection{Паралелепипед}
	\begin{definition}
Множеството\\
$$\Pi (a; \delta_1, \delta_2, ... ,\delta_m) = \{ x \in R^m: -\delta_k < x_k -a_k < \delta_k   \}$$
се нарича отворен паралелепипед в $R^m$ с център точката а.\\
\\
Множеството\\
$$\widetilde\Pi (a; \delta_1, \delta_2, ... ,\delta_m) = \{ x \in R^m: -\delta_k \leq x_k -a_k \leq \delta_k   \}$$
се нарича затворен паралелепипед в $R^m$ с център точката а.\\
\\
Ако $\delta_1 = \delta_2 = ... = \delta_m = \delta$, получените множества $\Pi (a; \delta)$ и $\widetilde\Pi (a; \delta)$ се  наричат съответно отворен и затворен куб в $R^m$ с център а.
 	\end{definition}
\subsubsection{Сфера и кълбо}
	\begin{definition}
Нека числото r > 0. Множеството \\
$$B(a;r) = \{ x | x \in R^m, \rho(x,a) <r \} = \{ x| x \in R^m, || x-a || < r \}$$\\
се нарича отворено кълбо в $R^m$ с център a и радиус r, множеството \\
$$\widetilde B(a;r) = \{ x | x \in R^m, \rho(x,a) \leq r \} = \{ x| x \in R^m, || x-a || \leq r \}$$\\
се нарича затворено кълбо в $R^m$ с център a и радиус r, а множеството \\
$$S(a;r) = \{ x | x \in R^m, \rho(x,a) = r \} = \{ x| x \in R^m, || x-a || = r \}$$\\
се нарича сфера в $R^m$ с център a и радиус r, а множеството \\
	\end{definition}

	\begin{definition}
Точката а се нарича
\begin{itemize}
	\item вътрешна за множеството А, ако съществува отворено кълбо $B(a,\varepsilon): B(a,\varepsilon) \subset A$
	\item външна за А, ако съществува $B(a,\varepsilon): B(a,\varepsilon) \subset R^m \backslash  A$
	\item контурна за А, ако за всяко $\varepsilon > 0: B(a,\varepsilon) \cap A \neq \O $ и \\ $ B(a,\varepsilon) \cap (R^m \backslash  A) \neq \O$
	\item изолирана ако съществува $\varepsilon > 0: B(a,\varepsilon) \cap A = \{ a \}$
\end{itemize}
	\end{definition}

\begin{definition}
Множеството $A \subset R^m $ се нарича 
\begin{itemize}
	\item отворено, ако всяка негова точка е вътрешна 
	\item затворено, ако неговото допълнение  $R^m \backslash A$ е отворено
\end{itemize}
	\end{definition}

\begin{definition}
Околност на дадена точка $a \in R^m $ се нарича всяко отворено множество, което я съдържа. Означава се с $U_a$.
\end{definition}

\begin{definition}
Точка а се нарича точка на сгъстяване на множеството $A \subset R^m $, ако всяка нейна околност $U_a$ съдържа поне една точка на А, различна от а, т.е $U_a \cap (A\backslash \{ a \} \neq \O $
\end{definition}

\begin{definition}
Величината \\
$$d = d(A) = \sup_{a',a'' \in A} \rho(a';a'')$$\\
се нарича диаметър на множеството  $A \subset R^m $.
\end{definition}

\begin{definition}
Множеството  $A \subset R^m $ се нарича ограничено, ако съшествува кълбо(с краен радиус), което го съдържа.
\end{definition}

\begin{definition}
Множеството  $A \subset R^m $ се нарича компактно, ако А е затворено и ограничено.
\end{definition}

\begin{definition}
Множеството   $x = (x_1, x_2, ... , x_m) \in R^m$, чийто кординати са непрекъснати функции $x_k = x_k(t) (k = 1, 2, ..., m)$, дефинирани върху даден интервал [a,b] се нарича непрекъсната крива в $R^m$. t се нарича параметър на кривата.\\
Точките $x(a) = (x_1(a), x_2(a), ... , x_m(a))$ и $x(b) = (x_1(b), x_2(b), ... , x_m(b))$ се наричат начало и край на дадената крива. Ако $x(a) = x(b)$ кривата е затворена
\end{definition}

\begin{definition}
Нека $x^0 = (x_1^0, x_2^0, ... , x_m^0) \in R^m$ и $\alpha_1, \alpha_2, ... , \alpha_m$ са фиксирани числа за които $\sum_{k=1}^{m}\alpha_k > 0$. Множеството от точки $x = (x_1, x_2, ... , x_m)$ чиито кординати се представят във вида \\
$$x_k = x_k^0 + \alpha_kt, k = 1, 2, ..., m , -\infty < t <\infty $$\\
се нарича права линия в пространството $R^m$, минаваща през точка $x^0$ по направление $(\alpha_1, \alpha_2, ... , \alpha_m).$
\end{definition}

\begin{definition}
Множеството  $A \subset R^m $ се нарича свързано,  ако за всеки две негови точки съществува непрекъсната крива $\gamma$, която ги свързва и $\gamma \subset A$.
\end{definition}

\begin{definition}
Множеството  $A \subset R^m $ се нарича област, ако е отворено и свързано. Ако е и затворено, то се нарича затворена област. 
\end{definition}

\begin{definition}
Област, всеки две точки на която могат да се съединят с отсечка, изцяло лежаща в нея, се нарича изпъкнала област. 
\end{definition}

\begin{definition}
Областа $A \subset R^m $ се нарича звездообразна област, отностно точката $x^0 \in A$, ако за вскяка точка $x \in A$ отсечката $[x^0,x]$ лежи изцяло в А. 
\end{definition}

\subsection{Редици от точки в $R^m$}

\begin{definition}
Редицата $\{ x^{(n)} \}_{n = 1} ^ \infty = \{x_1 ^{(n)}, x_2 ^{(n)}, ... , x_m^{(n)}\} $ се нарича редица от точки в $R^m$, а редицата $\{ x_k ^{(n)}\}_{n=1} ^ \infty ( k = 1 \div m) $ - к-та кординатна редица. За по кратко редицата $\{ x^{(n)} \}_{n = 1} ^ \infty$ се означава $\{ x^{(n)} \}$
\end{definition}

\begin{definition}
Редицата $\{ y^{(l)} \}_{l = 1} ^ \infty$ се нарича поредица на редицата $\{ x^{(n)} \}$ и се означава:\\
$\{ x^{(n_l)} \}, l = 1, 2, ..., $ или $\{ x^{(n_l)} \}_{l = 1} ^ \infty$\\
ако за всяко l съществува такова $n_l$, че $y^{(l)} = x^{(n_l)}$, при това, ако $l' < l'' $, то $n_{l'}<n_{l''}$.
\end{definition}

\begin{definition}
Редицата $\{ x^{(n)} \}$ се нарича сходяща към точка $a \in R^m$ (граница на редицата), ако за всяко $\varepsilon > 0$ съществува такова $N_0 > 0$, че за всяко $n > N_0$ е изпълено неравенството $\rho(x^{(n)};a) = ||x^{(n)} - a|| < \varepsilon$. Ако редицата няма граница, се нарича разходяща. 
\end{definition}


\begin{definition}
Точката $a \in R^m$ се нарича точка на сгъстяване на редицата $\{ x^{(n)} \}$, ако всяка нейна околност съдържа безброй много членове на редицата. 
\end{definition}

\begin{theorem}
Нека $x^{(n)} \in R^m$ за $n \in N$ и точката $a \in R^m$. Тогава 
$$(\{x^{(n)}\} \to a )\iff (x_k ^{(n)} \to a_k, k = 1 \div m)$$\\
T.e редицата има граница точката а, тогава и само тогава когато всяка от кординатите на редици $\{x_k^{(n)}\}$ има граница съответната кордината $a_k$ на точката а
\end{theorem}

\begin{theorem}[Критерий на Коши]
Нека $x^{(n)} \in R^m$ за $n \in N$. Редицата $x^{(n)}$ е сходяща тогава и само тогава когато за всяко $\varepsilon > 0$ съществува такова число $N_0 > 0$, че при всяко $n \in N, n > N_0$ и всяко $p \in N$ е изпълено $\rho(x^{(n+p)}, x^{(n)}) = ||x^{(n+p)} - x^{(n)} || < \varepsilon$
\end{theorem}

\begin{definition}
Редицата $\{x^{(n)}\}$ се нарича ограничена , ако съществува кълбо ( с краен радиус), което съдържа всичките ѝ членове. 
\end{definition}


\begin{theorem}[Болцано-Вайерщрас]
От всяка ограничена редица в пространството $R^m$ може да се избере сходяща подредица. 
\end{theorem}


\begin{definition}
Всяко множество $A \subset R^m $ се нарича компактно, ако от всяка редица $\{x^{(n)}\}, x^{(n)} \in A$, може да се избере сходяща подредица $\{x_k ^{(n)}\}$ с граница принадлежаща на А
\end{definition}

\section{Лекция 2: Функция на няколко променливи. Граница и непрекъснатост}

\section{Лекция 3: Частни производни. Диференцируемост на функция на две и повече променливи }
















































\end{document}
