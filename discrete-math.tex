\documentclass[fleqn]{article}

\usepackage[utf8]{inputenc}
\usepackage[bulgarian]{babel}
\usepackage{amsmath}
\usepackage{amssymb}
\usepackage{booktabs}

\newtheorem{example}{Пример}[subsection]
\newtheorem{definition}{Дефиниция}[subsection]
\newtheorem{axiom}{Аксиома}[subsection]

\title{Дискретна математика}
\author{Exonaut}


\begin{document}
\maketitle
\pagenumbering{gobble}

\newpage
\pagenumbering{arabic}

\tableofcontents
\newpage

\section{Лекция 1: Логика и логически оператори}

\subsection{Дефиниции}

\begin{definition}
\textbf{Логиката} е система, базирана на съждения
\end{definition}

\begin{definition}
\textbf{Съждението} е твърдение което може да бъде истина или лъжа (но не и двете едновеременно).\\
Следователно резултатът от едно съждение може да бъде истина(И) или ако то е вярно или лъжа (Л), ако е грешно.
\end{definition}

\begin{definition}
Съжденията, които не съдържат в себе си други съждения, се наричат \textbf{прости}.
\end{definition}

\begin{definition}
Едно и няколко съждения могат да бъдат обединени в едно единствено \textbf{комбинирано съждение}, посредством логически оператори. 
\end{definition}

\begin{definition}
\textbf{Таблица на истинност} се нарича таблица, в която се изброяват всички възможни комбинации от стойности  на отделните променливи в съждението, както и съответните стойности на функцията. 
\end{definition}

\begin{definition}
Две \textbf{съждения са еквиваленти}, ако имат една и съща таблица на истинност или следват едно от друго вследствие прилагани основни закони за преобразуване. 
\end{definition}
\newpage
\subsection{Логически оператори}

\subsubsection{Отрицание(NOT)}
Означава се със знака $\neg$\\
Функция на една променлива с таблица на истинност: 
\begin{table}[htp]
  \begin{center}
    \begin{tabular}{c|c} 
      \textbf{p} & \textbf{$\neg$p} \\
      \hline
      T & F \\
      F & T \\
    \end{tabular}
  \end{center}
\end{table}

\subsubsection{И, Конюнкция (AND)}
Означава се със знака $\land$\\
Функция на две променливи с таблица на истинност: 
\begin{table}[htp]
  \begin{center}
    \begin{tabular}{c|c|c} 
      \textbf{p} & \textbf{q}  & \textbf{p $\land$ q} \\
      \hline
	F & F & F \\
	F & T & F \\
	T & F & F \\
 	T & T & T \\
    \end{tabular}
  \end{center}
\end{table}

\subsubsection{ИЛИ, Дизюнкция (OR)}
Означава се със знака $\lor$\\
Функция на две променливи с таблица на истинност: 
\begin{table}[htp]
  \begin{center}
    \begin{tabular}{c|c|c} 
      \textbf{p} & \textbf{q}  & \textbf{p $\lor$ q} \\
      \hline
	F & F & F \\
	F & T & T \\
	T & F & T \\
 	T & T & T \\
    \end{tabular}
  \end{center}
\end{table}

\newpage

\subsubsection{Сума по модул 2, изключващо или (XOR)}
Означава се със знака $\otimes$\\
Функция на две променливи с таблица на истинност: 
\begin{table}[htp]
  \begin{center}
    \begin{tabular}{c|c|c} 
      \textbf{p} & \textbf{q}  & \textbf{p $\otimes$ q} \\
      \hline
	F & F & F \\
	F & T & T \\
	T & F & T \\
 	T & T & F \\
    \end{tabular}
  \end{center}
\end{table}

\subsubsection{Импликация, следствие }
Означава се със знака $\rightarrow$\\
Функция на две променливи с таблица на истинност: 
\begin{table}[htp]
  \begin{center}
    \begin{tabular}{c|c|c} 
      \textbf{p} & \textbf{q}  & \textbf{p $\rightarrow$ q} \\
      \hline
	F & F & T \\
	F & T & T \\
	T & F & F \\
 	T & T & T \\
    \end{tabular}
  \end{center}
\end{table}

\subsubsection{Двупосочно следствие}
Означава се със знака $\Leftrightarrow$\\
Функция на две променливи с таблица на истинност: 
\begin{table}[htp]
  \begin{center}
    \begin{tabular}{c|c|c} 
      \textbf{p} & \textbf{q}  & \textbf{p $\Leftrightarrow$ q} \\
      \hline
	F & F & T \\
	F & T & F \\
	T & F & F \\
 	T & T & T \\
    \end{tabular}
  \end{center}
\end{table}

\newpage

\subsection{Закони за еквивалентни преобразувания}

\begin{itemize}
	\item Закон за идентичност : $p \land T \equiv p , p \lor F \equiv p$
	\item Закон за доминиране : $p \lor T \equiv T , p \land F \equiv F$
	\item Закон за пълна идентичност : $p \land p \equiv p , p \lor p \equiv p$
	\item Закон за двойно отрицание : $\neg(\neg p) \equiv p $
	\item Комутативен закон : $p \land q \equiv q \land p | p \lor q \equiv q \lor p$
	\item Асоциативен закон : $p \land (q \land r) \equiv (p \land q ) \land r | p \lor (q \lor r) \equiv (p \lor q ) \lor r$
	\item Дистрибутивен закон : $p \land (q \lor r) \equiv (p \land q ) \lor (p \land r ) | p \lor (q \land r) \equiv (p \lor q ) \land (p \lor r )  $
	\item Закони на Де Морган : $\neg (p \land q) \equiv (\neg p) \lor (\neg q) | \neg (p \lor q) \equiv (\neg p) \land (\neg q) $
	\item Закон за импликацията : $p \rightarrow q \equiv \neg p \lor q$
	\item Закон за тривиалната тавтология:  $p \lor \neg p \equiv T | p \land \neg p \equiv F$
	\item Закон за тривиалното опровержение : $(p \Leftrightarrow q) \equiv \neg (p \otimes q), \neg (p \Leftrightarrow q) \equiv (p \otimes q) $
\end{itemize}

\subsection{Предикатни функции и предикати}

\begin{definition}[Предикатна функция]
Предикатна функция е твърдение, което съдържа една или повече променливи. \\
Ако на дадена променлива е присвоена стойност, се казва, че е известна.\\
Предикатна функция става съждение, ако всички нейни аргументи са известни. \\
\end{definition}

\begin{definition}[Предикат]
Нека е дадена предикатна функция $Q(x,y,z)$.\\
Свойството Q, с което се задава връзката между променливите x, y, z се нарича предикат.
\end{definition}

\subsection{Квантор}
Нека P(x) е предикатна функция.

\begin{definition}[Квантор за общност]
За твърдения от вида: \\
За \textbf{всяко} x, P(x) e истина/лъжа.\\
се записва : \\
$\forall x P(x)$ "за всяко х P(x)"\\
Знака за общност е $\forall$.
\end{definition}

\begin{definition}[Квантор за съществуване]
За твърдения от вида: \\
Съществува \textbf{такова} x, \textbf{за което} P(x) e \textbf{истина/лъжа}.\\
се записва : \\
$\exists x P(x)$ "съществува х, такова че P(x) е истина/лъжа" или "Съществува поне едно x, за което P(x) е истина/лъжа"\\
Знака за общност е $\exists$.
\end{definition}

\subsection{Закони на Де Морган за квантори}

\begin{itemize}
	\item $\neg (\forall P(x)) \equiv \exists x(\neg P(x))$
	\item $\neg (\exists P(x)) \equiv \forall x(\neg P(x))$
\end{itemize}

\section{Лекция 2: Математически доказателства}

\subsection{Теория на доказателствата}
Теорията на доказателствата се използва за определяне на верността на дадени математически аргументи и за конструирането им. 

\begin{definition}[Дедукция]
Дедукция във философията означава извеждане на особеното и единичното от общото, както и схващане на единичния случай на базата на всеобщ закон. \\
В кибернетиката означава извеждане на твърдения от други твърдения с помощта на логически заключения.
\end{definition}

\subsection{Терминология}

\begin{definition}[Аксиома]
Аксиомата е базово допускане, което не е необходимо да се доказва. 
Аксиомите са твърдения, които са истина, или твърдения които се приемат за истинам но не могат да бъдат доказани. 
\end{definition}

\begin{definition}[Доказателство]
Доказателството се използва, за да се докаже, че дадено твърдение е истина. То се състои от последователност от твърдения, които формират \textbf{аргумент}.
\end{definition}

\begin{definition}[Теорема]
Теорема е твърдение, чиято истинност се доказва. \\
Теоремата се състои от две части: условия(хипотези) и  извод(заключения). \\
Коректното доказателство (по дедукция) се състои в това да се установи: 
\begin{itemize}
	\item Ако условията са изпълнени, то извода е истина.
	\item Ако съждението "условия $\rightarrow$ извод" е тавология.
\end{itemize}
Често липват елементи на логическа връзка, която може да бъде запълнена с допълнителни условия и аксиоми и съждения, свързани помежду си посредством подходящи правила за изводи. 
\end{definition}

\begin{definition}[Лема]
Лемата е проста теорема, което се използва като междинен резултат за доказване на друга теорема. 
\end{definition}

\begin{definition}[Слествие]
Следствието е резултат, който директно следва от съответната теорема
\end{definition}

\begin{definition}[Допускане]
Допускането е твърдение, чийто резултат е неизвестен. Веднъж доказано, то се превръща в теорема. 
\end{definition}

\subsection{Правила за извод}
$\therefore$ - знак за следователно\\
Всичко преди знака са хипотези. 
\begin{itemize}
	\item Събиране(Addition) \\ $p (q) \\ \therefore p\lor q$
	\item Опростяване (Simplification) \\$p\land q \\ \therefore p(q)$
	\item Конюнкция (Conjunction) \\$p \\ q\\ \therefore p\land q$
	\item Закон за безразличие(Modus ponens) \\ $p \\ p \rightarrow q \\ \therefore q$
	\item Моdus tollens \\$\neg q \\ p \rightarrow q \\ \therefore \neg p$
	\item Хипотетичен силогизъм (Hypothetical syllogism) \\$ p \rightarrow q \\ q \rightarrow r \\ \therefore p \rightarrow r $
	\item Дизюнктивен силогизъм \\ $p\lor q \\ \neg p \\ \therefore q$
\end{itemize}

\subsection{Аргументи}

\begin{definition}[Аргумент]
Аргументът се състои от една или няколко хипотези и заключение. \\
Аргумент е валиден, ако всичките му хипотези са истина и заключението също е истина. \\
Ако някоя от хипотезите е лъжа, дори валиден аргумент може да води до некоректно заключение. \\
\\
Доказателство: трябва да се докаже че твърдението "хипотези $\rightarrow$ заключение" е истина, като се използват правила за извод. 
\end{definition}

\begin{example}
"Ако $101$ е кратно на $3$ , то $101^2$ се дели на $9$." \\
Въпреки че аргументът е валиден, заключението е некоректно (невярно), защото едната от хипотезите(“$101$ е кратно на $3$”)е лъжа. \\
Ако в аргумента $101$ се замести с $102$, ще се получи коректно заключение “$102^2$ е кратно на $9$”.
\\ 
Нека p: "101 e кратно на 3" и q: "$101^2$ е кратно на $9$". Toгава имаме следния случай
\\
$p
\\ p \rightarrow q 
\\ \therefore q$
\\Обаче p е лъжа, следователно q е некоректно заключение. 
\end{example}

\subsection{Правила за извод при използване на квантори и предикати}

\begin{itemize}
	\item Универсалноследствие(Universal instantiation)
\\
$\forall x P(x) 
\\
\therefore P(c)$ ако $c \in U$
	\item Универсално обобщение(Universal generalization)
\\
$ P(c)$ за някое $c \in U$ 
\\
$\therefore\forall x P(x)$

	\item Частичноследствие(Existential instantiation)
\\
$\exists x P(x) 
\\
\therefore P(c)$ за някой елемент $c \in U$

	\item Частично обобщение(Existential generalization)
\\
$P(c)$ за някой елемент $c \in U$
\\
$\therefore \exists x P(x)  $
\end{itemize}

\subsection{Доказателство на теореми}
\begin{itemize}
	\item Директно доказателство \\
Импликацията "$p \rightarrow q$" може да бъде доказана чрез доказване на твърдението:\\
 “Ако p е истина , то q също е истина.”
	\item Индиректно доказателство \\
Импликацията "$p \rightarrow q$" е еквивалентна на следния контра-пример "$\neg q \rightarrow \neg p$". Следователно, доказателството на изходната импликация "$p \rightarrow q$" се свежда до доказване на твърдението: “Ако q е лъжа, то и p също е лъжа.
\end{itemize}

\section{Лекция 3: Теория на множества}

\subsection{Парадокс на Ръсел}
Някои множества (класове) съдържат себе си а други не.\\
Ръсел нарича множествата класове.\\
Класът от всички класове е клас, който се съдържа (принадлежи)на себе си. Празният клас не принадлежи на себе си. Да допуснем, че може да създадем клас от всички класове, като празния например, които не съдържат себе си.\\
Парадоксът възниква при въпроса, дали този клас принадлежи на себе си. \\
Множеството от всички множества, които не съдържат себе си !? \\
$M = \{ A | A \notin A \}$\\
\begin{itemize}
	\item Множество – първично понятие
	\item Принадлежност на елемент към множеството – първично понятие \\
$x\in A \rightarrow$ Елементът x принадлежи на A.
	\item Свойство – понятие от логиката и се приема за първично. \\
P – свойство. Ако даден предмет x го притежава се записва P(x)
\end{itemize}

\begin{definition}[Множества]
Множества се дефинират като се изброяват елементите му 
$$A = \{ a_1, a_2, ..., a_n\}$$
за безкрайни множества е по - удобно като се използва свойство за принадлежност 
$$R = \{x | P(x) \} $$
\end{definition}

\begin{axiom}
Ако го има свойството P, то съществува множеството на всички обекти, които имат това свойство P.
\end{axiom}

\begin{axiom}
Две множества са равни, ако съдържат еднакви елементи.
\end{axiom}

\begin{definition}[Нормално множество]
Нормално множество е множество, което не принадлежи на себе си $A \notin A$
\end{definition}

\subsection{Множества}
\begin{itemize}
	\item Множество е неподредена съвокупност от нула или повече различни обекти (наречени елементи). 
 	\item $a \in A$ - "а е елемент на множеството А" 
	\item $a \notin A$ - "а не е елемент на А"
	\item $A = \{ a_1, a_2, ..., a_n\}$ - "A се състои от $a_1, a_2, ..., a_n$"
	\item Подреждането на елементите не е от значение.
	\item Няма значение ако някой елемент се повтаря. 
\end{itemize}

\begin{example}

\begin{itemize}
	\item $A = \varnothing$ - празно множество 
	\item $A = \{z\}, z \in A, z \neq \{z\}$
	\item $A =\{ \{b,c \}, \{c, x, d \} \}$ - множество от множества 
	\item $A = \{\{ x, y \}\}, \{x,y\} \in A, \{x,y\} \neq \{\{x,y\}\}$
	\item $A = \{x | P(x) \}$
\end{itemize}

\end{example}

\subsubsection{Равенство на множества}

\begin{definition}
Множествата A и B са равни тогава и само тогава, когато съдържат едни и същи елементи.
\end{definition}

\begin{example}
$A = \{9, 2, 7, 3\}, B = {7, 9, -3, 2} : A = B$\\
$A$= $\{$куче, котка, кон$\}$, $B$ =$\{$котка, кон, катерица, куче$\}$ : $A\neq B$ \\
$A$ = $\{$куче, котка, кон$\}$, $B$ = $\{$котка, кон, куче, куче$\}$  : $A = B$
\end{example}

\subsubsection{Подмножества}
$A \subseteq B$ - "A e подмножество на B"\\
$A \subseteq B$ тогава и само тогава, когато всеки елемент на А е и елемент на В\\
Формален запис:
$$A \subseteq B \Leftrightarrow \forall x (x \in A \rightarrow x \in B )$$
\\
Правила
\begin{itemize}
	\item $A = B \Leftrightarrow (A \subseteq B) \land (B \subseteq A)$
	\item $((A \subseteq B)) \land (B \subseteq C)) \Rightarrow (A \subseteq C) $
	\item $\varnothing \subseteq A$ за всяко множество А (но $\varnothing \in A$ не е вярно за всяко множество А)
	\item $А \subseteq A$ за всяко множество А
\end{itemize}

\subsubsection{Собствени подмножества}
$A \subset B$ - "A е собствено подмножество на B"\\
$A \subseteq B$ и $A \neq B$\\
Формален запис:
$$A \subset B \Leftrightarrow \forall x (x \in A \rightarrow x \in B ) \land \exists x (x \in B \land x \notin A)$$
или
$$A \subset B \Leftrightarrow \forall x (x \in A \rightarrow x \in B ) \land \neg \forall x (x \in B \rightarrow x \in A)$$

\subsubsection{Число на кардиналност, мощност(брой елементи)  на множество}
Ако множеството S съдържа n различни елемента, $n \in N$, Множеството S е крайно множество с кардиналност (мощност) n.\\
Мощност: $|P(A)| = 2^{|A|}$ 
\begin{example}
$A = \{Mercedes, BMW, Porsche\}, |A| = 3 $\\
$B = \{1, \{2, 3\}, \{4, 5\}, 6\}, |B| = 4 $\\
$C = \varnothing, |C| = 0 $ \\
$D = \{x \in N | x \leq 7000\}, |D| = 7001$\\
$E = \{x \in N | x \geq 7000\}, |E| = \infty$\\
\end{example}

\subsubsection{Множество от всички подмножества на дадено множество}
$P(A)$ "множеството от всички подмножества на A" (записва се като $2^A$)\\
$P(A) = \{B | B\subseteq A \}$

\subsubsection{Декартово произведение}
\begin{definition}[Декартово произведение ]
Декартово произведение на две множества А и В е ново множество $A \times B$, което се дефинира като
$$A \times B = \{(a,b) | a\in A \land b \in B \}$$
т.е декартовото произведение е множество от наредени двойки. 
\end{definition}

\begin{example}
$A = \{x, y\}, B = \{a, b, c\}$\\
$A \times B = \{(x, a), (x, b), (x, c), (y, a), (y, b), (y, c)\}$
\end{example}
 
\begin{itemize}
	\item $A \times \varnothing = \varnothing$
	\item $\varnothing \times A = \varnothing$
	\item За непразни множества A иB : $A\neq B \Leftrightarrow A \times B \neq B \times A$
	\item $|A \times B| = |A|\cdot |B|$
\end{itemize}

\subsection {Oперации на множества} 
 
\subsubsection{Обединение}
$A \cup B = \{x | x\in A \land x \in B\}$
\begin{example}
$A = \{a, b\}, B = \{b, c, d\}$\\
$A \cup B = \{a, b, c, d\}$
\end{example}

\subsubsection{Сечение}
$A \cap B = \{x | x\in A \lor x \in B\}$\\
Несвързани множества: Нямат общи елементи , $A \cap B = \varnothing$

\begin{example}
$A = \{a, b\}, B = \{b, c, d\}$\\
$A \cap B = \{b\}$
\end{example}

\subsubsection{Разлика}

$A \setminus B = \{x | x \in A \land x \notin B \}$
\begin{example}
$A = \{a, b\}, B = \{b, c, d\}$\\
$A \setminus B = \{a\}$
\end{example}

\subsubsection{Допълнение}
$A^c \equiv  \overline{A} = U - A$\\ 
U - универсално множество. 

\begin{example}
$U = N$\\
$B = \{250, 251, 252, ...\}$\\
$\overline{B} = \{0, 1, 2, ..., 248, 249\}$
\end{example}

\subsection{Идентичност на законите за логическо преобразуване и законите за операции с множества}

\begin{itemize}
	\item Закон за идентичност : \\
$A \cup \varnothing = A \\
 A \cap U = A$
	\item Закон за доминиране :  \\
$A \cup U = A \\
 A \cap \varnothing = A$
	\item Закон за пълна идентичност : \\
$A \cup A = A \\
 A \cap A = A$
	\item Закон за двойно отрицание : \\
$\overline{\overline{A}} \equiv (A^c)^c  = A$
	\item Комутативен закон :\\ 
$A \cup B = B \cup A \\ 
A \cap B = B \cap A $
	\item Асоциативен закон :\\ 
$A \cup (B \cup C) = (A \cup B) \cup C \\ 
A \cap (B \cap C) = (A \cap B) \cap C $
	\item Дистрибутивен закон : \\
$A \cup (B \cap C) = (A \cup B) \cap (A \cup C) \\
 A \cap (B \cup C) = (A \cap B) \cup (A \cap C) $
	\item Закони на Де Морган : \\
$\overline{A \cup B} = \overline{A} \cap \overline{B} \\
\overline{A \cap B} = \overline{A} \cup \overline{B}$
	\item Закон за поглъщането : \\
$A \cup (A \cap B) = A \\
 A \cap (A \cup B) = A $
	\item Закон за допълнението: \\
$A \cup \overline{A} = U \\
 A \cap \overline{A} = \varnothing $  
\end{itemize}
 
\section{Лекция 4: Релации}








\end{document}