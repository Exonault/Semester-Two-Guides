\documentclass[fleqn]{article}

\usepackage[utf8]{inputenc}
\usepackage[bulgarian]{babel}
\usepackage{amsmath}
\usepackage{booktabs}


\newtheorem{example}{Пример}[subsection]

\newtheorem{definition}{Дефиниция}[subsection]



\title{Дискретна математика}
\author{Exonaut}


\begin{document}
\maketitle
\pagenumbering{gobble}
\newpage
\pagenumbering{arabic}

\tableofcontents
\newpage

\section{Лекция 1: Логика и логически оператори}

\subsection{Дефиниции}

\begin{definition}
\textbf{Логиката} е система, базирана на съждения
\end{definition}

\begin{definition}
\textbf{Съждението} е твърдение което може да бъде истина или лъжа (но не и двете едновеременно).\\
Следователно резултатът от едно съждение може да бъде истина(И) или ако то е вярно или лъжа (Л), ако е грешно.
\end{definition}

\begin{definition}
Съжденията, които не съдържат в себе си други съждения, се наричат \textbf{прости}.
\end{definition}

\begin{definition}
Едно и няколко съждения могат да бъдат обединени в едно единствено \textbf{комбинирано съждение}, посредством логически оператори. 
\end{definition}

\begin{definition}
\textbf{Таблица на истинност} се нарича таблица, в която се изброяват всички възможни комбинации от стойности  на отделните променливи в съждението, както и съответните стойности на функцията. 
\end{definition}

\begin{definition}
Две \textbf{съждения са еквиваленти}, ако имат една и съща таблица на истинност или следват едно от друго вследствие прилагани основни закони за преобразуване. 
\end{definition}
\newpage
\subsection{Логически оператори}

\subsubsection{Отрицание(NOT)}
Означава се със знака $\neg$\\
Функция на една променлива с таблица на истинност: 
\begin{table}[htp]
  \begin{center}
    \begin{tabular}{c|c} 
      \textbf{p} & \textbf{$\neg$p} \\
      \hline
      T & F \\
      F & T \\
    \end{tabular}
  \end{center}
\end{table}

\subsubsection{И, Конюнкция (AND)}
Означава се със знака $\land$\\
Функция на две променливи с таблица на истинност: 
\begin{table}[htp]
  \begin{center}
    \begin{tabular}{c|c|c} 
      \textbf{p} & \textbf{q}  & \textbf{p $\land$ q} \\
      \hline
	F & F & F \\
	F & T & F \\
	T & F & F \\
 	T & T & T \\
    \end{tabular}
  \end{center}
\end{table}

\subsubsection{ИЛИ, Дизюнкция (OR)}
Означава се със знака $\lor$\\
Функция на две променливи с таблица на истинност: 
\begin{table}[htp]
  \begin{center}
    \begin{tabular}{c|c|c} 
      \textbf{p} & \textbf{q}  & \textbf{p $\lor$ q} \\
      \hline
	F & F & F \\
	F & T & T \\
	T & F & T \\
 	T & T & T \\
    \end{tabular}
  \end{center}
\end{table}

\newpage

\subsubsection{Сума по модул 2, изключващо или (XOR)}
Означава се със знака $\otimes$\\
Функция на две променливи с таблица на истинност: 
\begin{table}[htp]
  \begin{center}
    \begin{tabular}{c|c|c} 
      \textbf{p} & \textbf{q}  & \textbf{p $\otimes$ q} \\
      \hline
	F & F & F \\
	F & T & T \\
	T & F & T \\
 	T & T & F \\
    \end{tabular}
  \end{center}
\end{table}

\subsubsection{Импликация, следствие }
Означава се със знака $\rightarrow$\\
Функция на две променливи с таблица на истинност: 
\begin{table}[htp]
  \begin{center}
    \begin{tabular}{c|c|c} 
      \textbf{p} & \textbf{q}  & \textbf{p $\rightarrow$ q} \\
      \hline
	F & F & T \\
	F & T & T \\
	T & F & F \\
 	T & T & T \\
    \end{tabular}
  \end{center}
\end{table}

\subsubsection{Двупосочно следствие}
Означава се със знака $\Leftrightarrow$\\
Функция на две променливи с таблица на истинност: 
\begin{table}[htp]
  \begin{center}
    \begin{tabular}{c|c|c} 
      \textbf{p} & \textbf{q}  & \textbf{p $\Leftrightarrow$ q} \\
      \hline
	F & F & T \\
	F & T & F \\
	T & F & F \\
 	T & T & T \\
    \end{tabular}
  \end{center}
\end{table}

\newpage

\subsection{Закони за еквивалентни преобразувания}

\begin{itemize}
	\item Закон за идентичност : $p \land T \equiv p , p \lor F \equiv p$
	\item Закон за доминиране : $p \lor T \equiv T , p \land F \equiv F$
	\item Закон за пълна идентичност : $p \land p \equiv p , p \lor p \equiv p$
	\item Закон за двойно отрицание : $\neg(\neg p) \equiv p $
	\item Комутативен закон : $p \land q \equiv q \land p | p \lor q \equiv q \lor p$
	\item Асоциативен закон : $p \land (q \land r) \equiv (p \land q ) \land r | p \lor (q \lor r) \equiv (p \lor q ) \lor r$
	\item Дистрибутивен закон : $p \land (q \lor r) \equiv (p \land q ) \lor (p \land r ) | p \lor (q \land r) \equiv (p \lor q ) \land (p \lor r )  $
	\item Закони на де Морган : $\neg (p \land q) \equiv (\neg p) \lor (\neg q) | \neg (p \lor q) \equiv (\neg p) \land (\neg q) $
	\item Закон за импликацията : $p \rightarrow q \equiv \neg p \lor q$
	\item Закон за тривиалната тавтология:  $p \lor \neq p \equiv T | p \land \neq p \equiv F$
	\item Закон за тривиалното опровержение : $(p \Leftrightarrow q) \equiv \neg (p \otimes q), \neg (p \Leftrightarrow q) \equiv (p \otimes q) $

\end{itemize}

\section{Лекция 2:}
































\end{document}